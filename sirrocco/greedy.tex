% !TEX root = sirocco-main.tex

\section{Finding dominators greedily}

Before we continue, let us pause for a moment to realize what we
have already achieved towards a good solution for the dominating set
problem. To illustrate the achievements, we consider a modified
greedy algorithm to solve the dominating set problem on classes with
bounded $\nabla_1$.

Given a graph $G$, in a preprocessing step,
we compute the set
of all vertices $v$ whose neighborhood $N(v)$ cannot be covered
by at most $2\nabla_1(G)$ vertices different from $v$. According
to \cref{lem:neighborhood-dom1}, almost all vertices do not
have this property and we can safely add all vertices with the property
to the output dominating set. We remove these vertices from the
graph and mark their neighbors as dominated. In the following, we call
the number of non-dominated neighbors of a vertex $v$ the \emph{residual
degree} of $v$, and slightly abusing notation, by $N(v)$
we denote the non-dominated neighbors of $v$.

We now iteratively consider the remaining vertices in the following
greedy procedure. Let $v$ be a vertex of maximum residual degree.
Assume first that~$N(v)$ is large TODO: fill numbers.
We compute all $(\alpha, q,\ell,K)$-pseudo-covers of $N(v)$.
According to \cref{lem:num-pseudo-covers}, this number is
bounded $2(4\nabla_1(G)/\alpha)^K$. As every
$(\alpha, q,\ell,K)$-pseudo-cover has at most $K$ vertices,
the total number of vertices in this collection is bounded by
$c=2K(4\nabla_1(G)/\alpha)^K$. We add all these vertices
to a dominating set, remove them, mark their neighbors as
dominated and update the residual degrees.

We claim that after at most $3\gamma$ steps this process terminates
because it does not find vertices with large residual degree anymore.
The reason is that according to \cref{lem:neighborhood-dom2},
in all but at most $2\gamma$ many steps we add a vertex from the
optimal dominating set $D$ to our dominating set. This implies in particular,
that up to this point we have added at most $3c\gamma$ vertices
to the dominating set.

However, when we have reached the situation that the residual degree
of every vertex is at most $x$, in particular every remaining vertex of
$D$ has residual degree at most $x$. Hence, we know that the resulting
graph has at most $(x+1)\gamma$ non-dominated vertices. By adding all
non-dominated vertices (iteratively if we like) we hence compute a $(3+3c+x)$-approximation of a minimum dominating set.

The same idea is the key approach to the modified greedy algorithms
on graphs with bounded arboricity and graphs that exclude some biclique~\cite{jones2017parameterized,siebertz2019greedy}. Unfortunately,
the same approach cannot be directly carried out in parallel. It could be
the case that all pseudo-covers contain the same vertex from the
optimal dominating set $D$ and a further dummy vertex. When adding
all pseudo-covers in parallel, we would in this case collect an unbounded
number of dummy vertices, which would not result in a good
approximation. The truly beautiful contribution of Czygrinow et al.\ is
an iterative selection procedure of vertices from the pseudo-covers
that works in parallel.
