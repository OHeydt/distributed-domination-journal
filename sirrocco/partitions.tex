% !TEX root = sirocco-main.tex

\section{Finding dominators in parallel}

According to \cref{lem:neighborhood-dom1}, the
neighborhood $N(v)$ of almost every vertex $v$
can be dominated by a few vertices different from $v$ and we added
the exceptions to the output dominating set. After this step, we
were hence allowed to assume that the neighborhood $N(v)$
of every vertex $v$ is dominated by a few vertices different from $v$.
In fact, according to \cref{lem:neighborhood-dom1}, the neighborhood $N(v)$ of almost every vertex $v$ can
be dominated by a few vertices from $D$, however, we cannot
determine locally which vertices do not have this property. Let us
fix for those vertices $v$ whose neighborhood cannot be dominated
by $2\nabla_1(G)$ vertices of $D$ different from $v$ arbitrary
$2\nabla_1(G)$ vertices different from $v$ that dominate $N(v)$.
We collect all of these dominating vertices together with the vertices
from $D$ in a set $D'$. According to \cref{lem:neighborhood-dom1} and \cref{lem:neighborhood-dom2}, $D'$ contains at most
$(1+4\nabla_1(G))\gamma$ vertices.

By construction of $D'$ we find for every vertex $v$ a cover of
$N(v)$ consisting of at most $2\nabla_1(G)$ vertices from $D'$.
We now consider all $W=N(v)$ of size at least ... Then according
to \cref{lem:num-pseudo-covers}, the number of
$(\alpha,q,\ell,K)$-pseudo-covers is bounded by
$2(4\nabla_1(G)/\alpha)^K$. Furthermore, within those
pseudo-covers there are enough vertices of $D'$ that cover
all but $q$ vertices of $N(v)$.

\smallskip
Intuitively, we now proceed as follows. Every vertex $v$, instead of
choosing itself for the dominating set, considers all vertices from
the pseudo-covers of~$N(v)$ as a dominator for its neighborhood.
The pseudo-covers induce a partition of~$N(v)$ by forming
intersections of the neighborhoods of all the elements from the
pseudo-covers. For all these intersections, instead of directly
choosing a vertex~$w$ from a pseudo-cover, we again postpone
the decision and consider all vertices from the pseudo-covers of
$N(w)$ as possible dominators. This process is iterated again
and again, and by forming more and more refined intersections,
in $s$ rounds of the construction we construct a $K_{s,t}$
subgraph for some value of $s$. Recall that we observed that
if \mbox{$\nabla_0(G)< t/2$}, then $G$ excludes $K_{t,t}$ as
a subgraph. This fact is used to conclude that the process
must terminate after at most $t$ rounds. \textcolor{red}{TODO:
remark in the prelims that $\nabla_0\leq \nabla_1$. TODO:
Motivate bounded expansion more and give examples of bounded
expansion that do not exclude topological minors.} Furthermore,
it will turn out that the iterative process will at some point
consider a vertex from $D'$. This will allow us to bound the
number of vertices that are considered in the very last step
of the selection process. Choosing these last vertices as dominators
will hence be possible as their number is small, and this choice
will leave us with a graph of bounded degree. As in the greedy
algorithm we add all remaining vertices to the dominating set
and obtain a constant factor approximation of a minimum
dominating set.

\smallskip
Let us now formalize this approach. We slightly diverge from the
presentation of Czygrinow et al.\ to simplify the arguments and
algorithm. For families $\Aa, \Bb$ of sets, define
$\Aa\bigcap\Bb$ as the intersection closure of all
sets from $\Aa$ and $\Bb$ and $\Aa\bigcap_{\geq q}\Bb$
as $(\Aa\bigcap\Bb)\setminus \{X~:~|X|<q\}$. In other words,
$\Aa\bigcap_{\geq q}\Bb$ contains all possible intersections
$A_1\cap A_\ell\cap B_1\cap B_k$ for $A_1,\ldots, A_\ell\in \Aa,
B_1,\ldots, B_k\in \Bb$ of size at least $q$.

Slightly abusing notation, we write $\bigcup \Tt(v)$
for the set of all vertices that appear in some $S\in \Tt(v)$.
Let \[\Nn(v)\coloneqq \{(N(v)\cap N(x))\setminus \bigcup\Tt(v)~:~x\in \Tt(v)\}.\]
Hence, $\Nn(v)$ contains all intersections of $N(v)$ with
$N(x)$ for $x$ in some pseudo-cover from $\Tt(v)$, where
from the intersection the elements of $\bigcup\Tt(v)$ have
been removed. Let
$\Rr'(v)\coloneqq \Nn(v)\bigcap_{\geq q}\Nn(v)$. Finally,
let \[\Rr(v) \coloneqq \{X~:~X\text{ is $\subseteq$-minimal in }\Rr'(v)\}.\]

In other
words, we form all possible intersections $N(x_1)\cap N(x_2)\cap\ldots
\cap N(x_i)$ for $x_1,\ldots, x_i\in \bigcup \Tt(v)$ and remove
all vertices that are not in $N(v)$ and remove all vertices that
are in $\bigcup\Tt(v)$. Then we keep only those sets $X$ with at
least $q$
elements for which there is no other intersection $Y\subsetneq X$
with $|Y|\geq q$.

%
%Following Czygrinow et al.\ we call
%$\Tt(v)$ the set of all $(\alpha,q,\ell,K)$-pseudo-covers of $N(v)$.
%For $S=(x_1,\ldots, x_m)\in \Tt(v)$ define the following partition
%$\Pp_S=\{W_0,W_1,\ldots, W_m\}$ of $N(v)$. Let
%\begin{itemize}
%\item $W_1\coloneqq N(v)\cap N[x_1]$,
%\item $W_i\coloneqq (N(v)\cap N[x_i])\setminus \bigcup_{j<i} W_j$ for $i>1$ and
%\item $W_0\coloneqq N(v)\setminus \bigcup_{j\leq m}N[x_j]$.
%\end{itemize}
%Since $S$ is an $(\alpha,q,\ell,K)$-pseudo-cover of $N(v)$, we have $|W_0|\leq q$. Let
%$\Qq(v)$ be the minimal partition that refines the partitions
%$\Pp_S$ over all $(\alpha,q,\ell,K)$-pseudo-covers $S$ from $\Tt(v)$.

%\textcolor{red}{TODO: good figure}

\textcolor{red}{I don't know if we need this next lemma
or if we should actually
cut out the small intersections.}
\begin{lemma}
Let $X,Y\in \Rr(v)$. Then $|X\cap Y|<q$.
\end{lemma}
\begin{proof}
If $|X\cap Y|\geq q$, then $X$ and $Y$ are not $\subseteq$-minimal
in~$\Rr'(v)$.
\end{proof}

\textcolor{red}{TODO: plug in the number $c$ for $|\bigcup\Tt(v)|$.}

\begin{lemma}
If $G$ excludes $K_{q,t}$ as a subgraph, then
$|\Rr(v)|\leq c^t$.
\end{lemma}
\begin{proof}
Each element $X$ in $\Rr(v)$ is defined as the intersection
$N(x_1)\cap\ldots\cap N(x_i)\cap (N(v)\setminus \bigcup\Tt(v))$
for some $x_1,\ldots, x_i\in \bigcup\Tt(v)$. Furthermore,
$|X|\geq q$, so the vertices of $X$ together with $x_1, \ldots, x_i$
form a $K_{q,i}$ subgraph, which implies that $i\leq t$.
\end{proof}


%
%Slightly diverging from the presentation of Czygrinow et al.,
%we let $V_0$ be the union of the classes from $\Qq(v)$ that have
%size at most $q$ and let $V_1,\ldots, V_s$ denote the remaining classes.
%Then $\{V_0,V_1,\ldots, V_s\}$ is a partition of $N(v)$.
%We let $\Rr(v)\coloneqq \{V_1,\ldots, V_s\}$.
%
%\textcolor{red}{TODO: good figure}

\textcolor{red}{$q$ should actually be replaced by $\min\{q,c\}$.}

\begin{lemma}
If $G$ excludes $K_{q,t}$ as a subgraph, then at most $m\coloneqq q+c+c^{t+q}q$ vertices of $N(v)$ are not covered by $\Rr(v)$.
\end{lemma}
\begin{proof}
At most $q$ vertices of $N(v)$ are not covered at all by the
neighborhoods $N(x)$ for some $x\in \bigcup\Tt(v)$. We remove
$c=|\bigcup\Tt(v)|$ vertices by hand, which are not covered.
All other vertices are not covered because they lie in some
intersection $X=N(x_1)\cap\ldots\cap N(x_i)\cap (N(v)
\setminus \bigcup\Tt(v))$ of size smaller than $q$. We can choose
$x_1,\ldots, x_j$ for $j\leq t$ among $c$ vertices
until the intersection size drops
below $q$, as otherwise we find a $K_{q,t}$ subgraph as in the
proof of the previous lemma. When making the remaining
choices, the intersection gets smaller by at least one by every
choice we make, leaving us with at most $q$ choices among $c$
vertices.
\end{proof}


Motivated by our discussion of iteratively considering covering
vertices as better dominators, we define the following sets.
For a set $U\subseteq V(G)$ let \[\Tt(U)\coloneqq \bigcup_{v\in U}\Tt(v).\]
For a set $\Ss$ of pseudo-covers, again with a slight
abuse of notation, we define \[\Tt(\Ss)\coloneqq \Tt(\bigcup \Ss).\]
We now define
\[\Tt^{(1)}(U)\coloneqq \Tt(U)\]
and  for $i>1$
\[\Tt^{(i)}(U)\coloneqq \Tt(\Tt^{(i-1)}(U)).\]
Finally, $\Tt^{(\leq k)}(U)\coloneqq \bigcup_{1\leq i\leq k}\Tt^{(i)}(U)$.

\begin{lemma}
For all $i$ we have $\Tt^{(i)}(D')\leq c^i|D'|$.
\end{lemma}

We now consider the following process. Let $v=v_1\in V(G)$ and let
$\Rr_1\coloneqq \Rr(v_1)$ be as defined above.
For $i\geq 2$ we now choose some $v_i\in \bigcup\Tt(v_{i-1})$
different from $v_1,\ldots, v_{i-1}$ and let
$\Rr_i\coloneqq \Rr_{i-1}\bigcap_{\geq q}\Rr(v_i)$.
We terminate this construction
if we have constructed a sequence $v_1,\ldots,v_i$ and for all
possible choices of $v_{i+1}$ different from $v_1,\ldots, v_i$ we have
$\Rr_{i+1}=\emptyset$, that is, if all intersections of $\Rr_i$ with
$\Rr(v_{i+1})$ have size smaller than $q$.

\begin{center}
\fbox{\begin{minipage}{10cm}
We now define $\hat{D}$ as the set consisting of all vertices
$v_i$ such that there exists a maximal sequence $v_1,\ldots, v_i$
as constructed above.
\end{minipage}}
\end{center}

In the next lemmas we will prove that $\hat{D}$ is small
and that $G-N[\hat D]$ contains only vertices of bounded degree.
We can then argue as in the proof of the greedy algorithm that
also choosing all remaining vertices gives us in altogether a
small dominating set.

\smallskip
We state the first lemma for graphs excluding a complete
bipartite graph $K_{s,t}$ to make the argument as clear as possible.
Recall that we are working on graphs with $\nabla_1$ bounded
and that we have that $G$ excludes $K_{t,t}$ as a subgraph when
\mbox{$\nabla_0(G)< t/2$}.

\begin{lemma}
Let $v_1,\ldots, v_i$ be a maximal sequence constructed as above.
If $G$ excludes $K_{q,t}$ as a subgraph, then $i<t$.
\end{lemma}
\begin{proof}
\textcolor{red}{TODO: } We are otherwise constructing a $K_{q,t}$ subgraph.
\end{proof}


\begin{lemma}
If $G$ excludes $K_{q,t}$ as a subgraph, then
$\hat D\subseteq \Tt^{(\leq t)}(D')$.
\end{lemma}
\begin{proof}
By assumption, for every $v\in V(G)$ we have that $N(v)$ is
covered by at most $2\nabla_1(G)$ vertices of $D'$. By
\cref{lem:cover-to-pseudo-cover}, $N(v)$ hence has
a pseudo-cover $S$ with only vertices from $D'$. Every set
$\Rr(v_j)=\{V_1,\ldots, V_s\}$ therefore has the property that
$V_i\subseteq N(d)$ for some $d\in D'$. Hence, in every
step $j$ we can choose $v_j=d\in D'$, unless it has been chosen
already before and is therefore not available anymore. In this
case however, all choices for $v_j$ lie in $\Tt^{(\leq t)}(D')$.
\end{proof}


\begin{lemma}
$G-N[\hat D]$ is a graph of maximum degree ...
\end{lemma}
\begin{proof}
TODO
\end{proof}
