% !TEX root = sirocco-main.tex

\section{Finding dominators}


Recall that by \cref{lem:neighborhood-dom2} for every $v\in V(G)$ we can
cover $N(v)$ with at most~$k$ vertices from $D'$.
To first gain an intuitive understanding of the second phase of the
algorithm, where we construct a set $D_2\subseteq V(G)$, let us consider the
following iterative procedure.

Fix some $v\in V(G)$. Let $v_1=v$ and $B_1\coloneqq N(v)$
and assume $|B_1|\geq k^{t-1}(2t{-}1)$. We consider $s$
vertices $v_1,v_2\ldots, v_s$ as follows.
Choose as~$v_2$ an arbitrary vertex different from $v_1$
that dominates at least $k^{t-2}(2t-1)$ vertices of~$B_1$, that is,
a vertex that satisfies $|N[v_2]\cap B_1|\geq k^{t-2}(2t-1)$.
Note that any vertex~$v_2$ that dominate a $1/k$-fraction of $B_1$ can
be such vertex, i.e.~it is enough for $v_2$ to be $\alpha$-strong for $B_1$.

Let $B_2\coloneqq N(v_2)\cap B_1$. Observe that we consider
the open neighborhood of $v_2$ here, hence $B_2$ does
not contain $v_2$. Hence, $|B_2|\geq k^{t-2}(2t-1)-1\geq
k^{t-2}(2t-2)$.
We continue to choose vertices $v_3,\ldots$ inductively
just as above. That is, if the vertices $v_1,\ldots,v_i$ and sets
$B_1,\ldots, B_i\subseteq V(G)$ have been defined, we choose
the next vertex $v_{i+1}$ as an arbitrary vertex not in $\{v_1,\ldots,
v_i\}$ that dominates
at least $k^{t-i-1}(2t-i)$ vertices of $B_i$, that is, a vertex with
$|N[v_{i+1}]\cap B_i| \ge k^{t-i-1}(2t-i)$ and let
$B_{i+1}\coloneqq N(v_{i+1})\cap B_i$, of size at least
$k^{t-i-1}(2t-i-1)$.



\begin{lemma}\label{lem:sequence}
Assume $|N(v)|\geq k^{t-1}\cdot (2t-1)$. Let $v_1,\ldots, v_s$
be a maximal sequence obtained as above. Then $s<t$ and
$D'\cap \{v_1,\ldots, v_s\}\neq \emptyset$.
\end{lemma}
\begin{proof}
  Assume that we can compute a sequence $v_1,v_2,\ldots,v_t$. By definition,
  every $v_i$ is connected to every vertices of $B_t$.
  For every $1\leq i\leq t$ we have $|B_i|\geq k^{t-i}(2t-i)$
  and therefore
  $|B_t|\ge t$. This shows that the two sets
  $\{v_1,\ldots,v_t\}$ and $B_t$ form a $K_{t,t}$ as a subgraph of $G$.
  Since this is not possible, the process must stop having performed at
  most $t-1$ rounds.

  We now turn to the second claim of the lemma. Assume that
  $v_1,v_2,\ldots,v_s$ is a maximal sequence for some $s < t$. We assume
  $v_1\not\in \hat{D}$, otherwise, we are done, as $\hat{D}\subseteq D'$.
  Because $s <t$,
  we have that $B_s$ is not empty. Because $B_s \subseteq B_1 = N(v_1)$, we have
  that $B_s$ can be dominated with at most~$k$ elements of~$D$ (by definition of $\hat{D}$), and in particular by at most $k$ elements of $D'$. Therefore,
  there must be an element~$v$ of~$D'$ that dominates a $1/k$ fraction of $B_s$.
  If $v$ was not one of the $v_1,\ldots,v_s$, we could have continued the
  sequence by defining $v_{s+1} \coloneqq v$.
  Since the sequence is maximal, $v$ must be one of the $v_1,\ldots,v_s$,
  which leads to $D'\cap \{v_1,\ldots, v_s\}\neq\emptyset$.

%
%  By assumption, $B_1$ is dominated by at most $k$
%  vertices of $D'$. Hence, unless $v_1$ is in $B'$, we can find
%  a vertex $v_2\in V(G)$ that dominates at least
%  a $1/k$ fraction of $B_1$, that is, at least $k^{t-1}\cdot 2t$ vertices
%  of $B_1$. Let $B_2\coloneqq N(v_2)\cap B_1$,
%  hence $|B_2|\geq k^{t-1}\cdot 2t-1\geq k^{t-1}\cdot (2t-1)$.
%  Observe that we subtracted $1$, as $v_2$ is not in $B_2$.
%
%  Assume that neither $v_1$ nor $v_2$ are in $D'$. We repeat the same argument
%  as above for $B_2$. Because $B_2$ is dominated by at most $k$ vertices
%  of $D'$, there exists a vertex $v_3\in V(G)$ that
%  dominates at least a $1/k$ fraction of $B_2$, that is, at least
%  $k^{t-2}\cdot (2t-1)$ vertices of~$B_2$. Let $B_3\coloneqq
%  (N(v_3)\cap B_2)$,
%  hence $|B_3|\geq k^{t-2}\cdot (2t-1)-1\geq k^{t-2}\cdot (2t-2)$.
%  We repeat the argument for $v_4,\ldots,v_{\ell}$ and $B_4,\ldots, B_{\ell}$, each $B_i$
%  for $1\leq i\leq \ell<t$ of size at
%  least $k^{t-i}\cdot (2t-i)$, ending with a set~$B_{\ell}$ of
%  size at least $k\cdot (t+1)$.
%
%  Hence, assuming that $v_1,\ldots, v_\ell\not\in D'$, we have $\ell=t-1$
%  and there must
%  exist a vertex~$v$ with
%  $|N(v)\cap B_\ell|\geq t+1$. Fix any subset $B=\{w_1,\ldots, w_t\}$
%  of $N(v)\cap B_\ell$ of size exactly $t$. Then the vertices
%  $v,v_1,\ldots, v_{t-1}$ and the vertices $w_1,\ldots, w_t$
%  form a subgraph $K_{t,t}$, contradicting that
%  such a subgraph does not exist in $G$. Hence, one of
%  $v_1,\ldots, v_\ell$ must be contained in $D'$.
\end{proof}

We aim to carry out this iterative process in parallel
for all vertices \mbox{$v\in V(G)$} with a sufficiently large neighborhood.
Of course, in the process
we cannot tell when we have encountered the element of~$D'$.
Hence, from the constructed vertices $v_1,\ldots, v_s$
we will simply choose the element~$v_s$ into the dominating set.
Unfortunately, this approach alone can give us arbitrarily large
dominating sets, as we can have many choices for the vertices
$v_i$, while already $v_1$ was possibly optimal.
We address this issue by restricting
the possible choices for the vertices~$v_i$.

\begin{definition}\label{def:dom-sequence}
  For any vertex $v\in V(G)$, a {\em $k$-dominating-sequence} of $v$ is a sequence
  $(v_1,\ldots,v_s)$ for which we can define sets $B_1,\ldots,B_s$ such that:
  \begin{itemize}
    \item $v_1=v$, $B_1 \subseteq N(v_1)$,
    \item for every $i\le s$ we have $B_{i} \subseteq N(v_{i})\cap B_{i-1}$,
    \item $|B_{i}|\geq k^{t-i}(2t-i+(t-i)q)$
    \item and for every $i\le s$ we have $v_i\in \Pp(v_{i-1})$.
  \end{itemize}
  A $k$-dominating-sequence $(v_1,\ldots,v_s)$ is {\em maximal} if there is no
  vertex $u$ such that $(v_1,\ldots,v_s,u)$ is a $k$-dominating-sequence.
\end{definition}

%Remember that $t$ is an integer such that $G$ excludes a $K_{t,t}$ subgraph,
%and that $2\nabla_0(G)+1$ is such an integer. Also remember that we
%defined $K=k$,
%$\alpha = 1/k$, $\ell = 4k^3$ and $q=4k^4$, where $k=2\nabla_1(G)$.
Note that this definition requires $|N(v)|\ge k^{t-1}(2t-1+(t-1)q)$. For a
vertex~$v$ with a not sufficiently large neighborhood, there are no
$k$-dominating-sequences of $v$.
We show two main properties of these dominating-sequences.
First, \cref{lem:max-dom-sequence} shows that a maximal dominating sequence must
encounter $D'$ at some point. Second, with \cref{lem:shape-sequences,lem:small-D-hat,lem:inclusion-D-hat},
we show that collecting all ``end points'' of all maximal dominating sequences
results in a set $D_2$ of size linear in the size of $D'$. While $D'$ cannot be computed, we
can compute $D_2$.
%
% For $v\in V(G)$ consider the following modified construction of
% $v_1,\ldots, v_s$. Every $v_i$ is chosen such that it has
% an intersection $|N(v_i)\cap B_{i-1}|\geq k^{t-i}(2t-i+(t-i)q)$.
% Furthermore, we
% restrict the choice of $v_i$ to the set of vertices that appear
% in some $(\alpha,q,\ell,K)$-pseudo-cover in $\Tt(v_{i-1})$.
% We observe that this process still works as desired.

\begin{lemma}\label{lem:max-dom-sequence}
Let $v$ be a vertex %with $|N(v)|\geq k^{t-1}\cdot (2t-1+(t-1)q)$,
and let
$(v_1,\ldots, v_s)$ be a maximal $k$-dominating-sequence of $v$. Then $s<t$ and
$D'\cap \{v_1,\ldots, v_s\}\neq \emptyset$.
\end{lemma}
\begin{proof}
  The statement $s<t$ is proved exactly as for \cref{lem:sequence}.\\
  To prove the second statement we assume, in order to reach
  a contradiction, that $D'\cap\{ v_1,\ldots, v_s\}=\emptyset$.
  We have that $B_s \subseteq N(v_s)$, and remember that $N(v_s)$ can be
  dominated by at most $k$ elements of $D'$. By \cref{lem:cover-to-pseudo-cover},
  we can derive a pseudo-cover $S=(u_1,\ldots,u_m)$ of
  $N(v_s)$, where $m\le k$ and every $u_i$ is an element of $D'$.
  Let $X$ denote the set (of size at most $q$) of vertices not covered by $S$.
  As $S$ contains at most $k$ vertices there must exist a
  vertex $u$ in~$S$ that covers at least a $1/k$ fraction of
  $B_s\setminus X$.
  By construction, we have that $|B_s| \ge k^{t-s}\cdot(2t-s+(t-s)q)\ge k(t+q)$
  because $s<t$. Therefore $|B_s\setminus X| \ge k$ and we have
  $$|N[u]\cap B_{s}|\geq \frac{|B_s|-q}{k}
  \geq\frac{k^{t-s}(2t-s+(t-s)q) -q}{k},$$
  hence
  $$|N[u]\cap B_{s}| \geq\frac{k^{t-s}(2t-s+(t-s-1)q)}{k} \geq k^{t-s-1}(2t-s+(t-s-1)q),$$
  and therefore
  $$ |N(u)\cap B_{s}| \geq |N[u]\cap B_{s}|-1 \geq k^{t-s-1}(2t-s-1+(t-s-1)q).$$
  So we can continue the sequence $(v_1,\ldots,v_s)$ by defining
  $v_{s+1}\coloneqq u$. In conclusion if $(v_1,\ldots,v_s)$ is a maximal
  sequence, it contains an element of $D'$.
%  \alex{end of the proof}
%The statement $s<t$ is proved exactly as above. To show that
%$D'\cap \{v_1,\ldots, v_s\}\neq
%\emptyset$ holds, we show that in each step $i>1$ we have the
%possibility of choosing some vertex that appears in some
%$(\alpha,q,\ell,k)$-pseudo-cover $S\in \Tt(v_{i-1})$.
%As $B_i\subseteq N(v_{i-1})$ and $|B_i|\geq k^{t-i}(2t-i+(t-i)q)$,
%all but at most $q$ vertices of $B_i$ must be covered by $S$.
%Let $X$ denote the set of vertices not covered by $S$.
%As $S$ contains at most $k$ vertices there must exist a
%vertex in $S$ that covers at least a $1/k$ fraction of
%$B_i\setminus X$. Now we have
%$(k^{t-i}(2t-i+(t-i)q)-q)/k-1\geq k^{t-i-1}(2t-i-1+(t-i-1)q)$,
%so the recursion can continue as desired.
\end{proof}


The goal of this modified procedure is first to ensure that every maximal
sequence contains an element of $D'$ and second, to make sure that there are not
to many possible $v_s$ (which are the elements that we pick for the dominating set).
%Intuitively, picking $v_{i+1}$ as an element in some pseudo-cover of $N(v_i)$
%ensure that there are only few possible final picks $v_s$ (linear in $|D'|$).
%
This is illustrated in the following example and formalized right after that.

\begin{example}\label{ex:sequence}
  Consider the case of planar graphs. Since these graph exclude $K_{3,3}$,
  i.e.~$t=3$, we have that every maximal sequence is of length 1 or 2.
  For every $v$ of sufficiently large neighborhood we consider every
  maximal $k$-dominating-sequence $(v_1,v_s)$ of $v$.
  We then add $v_s$ to the set $D_2$. We want to show that $|D_2|$ is
  linearly bounded by $|D'|$ and hence by $\gamma(G)$.

  If $s=1$, then we have $v_s\in D'$ and we are good.

  If $s=2$, we have two possibilities. If $v_2$ is in $D'$, we are good.
  If however, $v_2$ is not in $D'$, then $v_1$ is in
  $D'$. Additionally, $v_2$ is in some pseudo-cover $S$ of $v_1$,
  i.e.~$v_2\in \Pp(v_1)$.

  By \cref{cor:nb-dominators}, we have $|\Pp(v_1)|\le (2k)^{2k+1}$ (and
  in fact this number is much smaller in the case of planar graphs).
  Therefore we have  $|D_2| \le ((2k)^{2k+1}+1)|D'|$.
\end{example}

We generalize the ideas of \cref{ex:sequence}, by explaining what
a ``few possible choices''  in the discussion before \cref{def:dom-sequence}
means.

\begin{lemma}\label{lem:shape-sequences}
  For any maximal $k$-dominating-sequence $(v_1,\ldots,v_s)$,
  and for any $i\le s-1$, we have that
  \begin{itemize}
    \item $v_{i+1}\in \Pp(v_i)$,
    \item $|N(v_i)|\ge \ell$, and
    \item $|\Pp(v_i)|\le (2k)^{2k+1}$.
  \end{itemize}
\end{lemma}
\begin{proof}
  By construction $v_{i+1}\in \Pp(v_i)$, furthermore, $v_i$ dominates at least
  $B_i$, and
  $|B_{i}|\geq k^{t-i}(2t-i+(t-i)q) \ge q >\ell$.
  We conclude with~\cref{cor:nb-dominators}.
\end{proof}

We now for every $v\in V(G)$ compute all maximal $k$-dominating-sequences
starting with $v$.
Obviously, as every $v_i$ in any $k$-dominating-sequences of $v$ dominates some
neighbors of $G$, we can locally compute these steps after having
learned the $2$-neighborhood $N_2[v]$ of every vertex in two rounds
in the LOCAL model of computation.

For a set $W\subseteq V(G)$ we write $\Pp(W) = \bigcup_{v\in W}\Pp(v)$.
Remember that the definition of $\Pp(v)$ requires that $|N(v)|>\ell$. We simply
extend the notation with $\Pp(v)=\emptyset$ if $|N(v)|\le \ell$.
We now define
\[\Pp^{(1)}(W)\coloneqq \Pp(W)\]
additionally, for $1<i <t$
\[\Pp^{(i)}(W)\coloneqq \Pp(\Pp^{(i-1)}(W))\]
and finally, for every $1\le i \le t$
\[\Pp^{(\leq i)}(W)\coloneqq \bigcup_{1\leq j\leq i}\Pp^{(j)}(W).\]


  % for the set of all vertices that appear in some $S\in \Tt(v)$.
  % For a set $U\subseteq V(G)$ let \[\Tt(U)\coloneqq \bigcup_{v\in U}\Tt(v).\]
  % For a set $\Ss$ of pseudo-covers, again with a slight
  % abuse of notation, we define \[\Tt(\Ss)\coloneqq \Tt(\bigcup \Ss).\]
  % We now define
  % \[\Tt^{(1)}(U)\coloneqq \Tt(U)\]
  % and  for $i>1$
  % \[\Tt^{(i)}(U)\coloneqq \Tt(\Tt^{(i-1)}(U)).\]
  % Finally, $\Tt^{(\leq k)}(U)\coloneqq \bigcup_{1\leq i\leq k}\Tt^{(i)}(U)$.



Using \cref{lem:shape-sequences}, for every
$k$-dominating-sequence $(v_1,\ldots,v_s)$ we have that
$v_s \in \Pp^{(\le k)}(v_1)$.
More generally, for every $i\le s$, we have that $v_s\in \Pp^{(\le t)}(v_i)$.


\begin{tcolorbox}
We define $D_2$ as the set of all $u\in V(G)$ such that there is some vertex
$v\in V(G)$, and some maximal $k$-dominating-sequence $(v_1,\ldots,v_s)$ of $v$
with $u=v_s$.
\end{tcolorbox}


This leads to the following lemma.
\begin{lemma}\label{lem:inclusion-D-hat}
  $D_2 \subseteq \Pp^{(\le t)}(D')$.
\end{lemma}
\begin{proof}
  This uses the observation made above the statement of this lemma, together
  with \cref{lem:max-dom-sequence}.
\end{proof}
Note that while we don't know how to compute $D'$, this section explained how
to compute $D_2$ in 2 rounds with the LOCAL model of computation.

\begin{lemma}\label{lem:small-D-hat}
  $|D_2| \le (2k)^{t(2k+1)} \cdot|D'|$
\end{lemma}
\begin{proof}
  \cref{cor:nb-dominators} gives us that $|\Pp(v)|\le 2k(2k^2)^k$ for every
  $v\in V(G)$ with $|N(v)|> \ell$.
  %This with \cref{lem:shape-sequences} implies that
  As $\Pp(W) \le \sum\limits_{v\in W} |\Pp(v)|$,
  we have $P(W)\le |W|\cdot (2k)^{2k+1}$.
  A naive induction yields that for every $i\le t$,
  \[ |\Pp^{(\le i)}(W)|\leq c^i|W|, \] where $c=(2k)^{2k+1}$.
  Hence with this and \cref{lem:inclusion-D-hat} we have
  \[|D_2| \le (2k)^{t(2k+1)} \cdot|D'|\]
\end{proof}
