% !TEX root = sirocco-main.tex

\section{Covers and pseudo-covers}

Intuitively, the vertices from a cover of a set $W$ can
take different roles. A few vertices of a cover may cover almost the
complete set $W$, while a few others are only there to cover what was
left over. The key observation of Czygrinow et al.\ is that in classes that
exclude some~$K_{t,t}$ as a subgraph, there can only be few of such
high degree covering vertices, while there can be arbitrarily many vertices
that cover at most $t-1$ vertices of~$W$ (the same vertices can be covered
over and over again). This observation can be applied recursively and is
distilled into the following two definitions. Recall that by the processing
carried out in the first phase of the algorithm
we know that every neighborhood $N(v)$ can be covered by $k=2\nabla_1$ vertices different from $v$.
% For the rest of this article we fix
% $\alpha\coloneqq 1/k$, $\ell=8\nabla_1/\alpha^2+1=4k^3+1$ and $q=4k^4$.
% These constants will be needed in the following definitions of pseudo-covers
% and domination sequences.
We recall all fixed parameters for easy to find reference.

\bigskip
\begin{tcolorbox}
\begin{tabular}{l l}
- $G$ & :~ fixed graph. \\
- $\gamma$ & :~ $\gamma(G)$.\\
- $\nabla_1$ & :~ $\nabla_1(G)$. \\
- $t$ & :~ smallest integer such that $G$
excludes $K_{t,t}$ as a subgraph\\
- $D_1$ & :~ defined and computed in \cref{lem:neighborhood-dom1}.\\
- $D$ & :~ fixed dominating set of $G$ of size $\gamma$ (not computed).\\
- $\hat{D}$ & :~ defined in \cref{lem:neighborhood-dom2} (not computed).\\
- $D'$ & :~ $D \cup \hat{D}$ (not computed).
\end{tabular}
\end{tcolorbox}

% For the rest of this article we fix
% $\alpha\coloneqq 1/k$, $\ell=8\nabla_1/\alpha^2+1=4k^3+1$ and $q=4k^4$.
% These constants will be needed in the following definitions of pseudo-covers
% and domination sequences.
% We also fix the following constants to follow the presentation
% of~\cite{czygrinow2018distributed}.

Following the presentation of~\cite{czygrinow2018distributed}, we name and fix
these constants for the rest of this article.

\begin{tcolorbox}
\begin{tabular}{l l}
- $k$       & $\coloneqq~ 2\nabla_1$.\\
- $\alpha$  & $\coloneqq~ 1/k$.\\
- $\ell$    & $\coloneqq~ 8\nabla_1/\alpha^2+1=4k^3+1$.\\
- $q$       & $\coloneqq~ 4k^4$.
\end{tabular}
\end{tcolorbox}


\begin{definition}
A vertex $z\in V(G)$ is \emph{$\alpha$-strong} for a vertex set $W\subseteq V(G)$ if $|N[z]\cap W|\geq \alpha|W|$.
\end{definition}

The following is the key definition by Czygrinow et al.~\cite{czygrinow2018distributed}.

\begin{definition}
A \emph{pseudo-cover} (with parameters $\alpha, \ell, q, k$)
of a set $W\subseteq V(G)$ is a
sequence $(v_1,\ldots, v_m)$ of vertices
such that  for every $i$ we have
\begin{itemize}
\item $|W\setminus \bigcup_{j\le m}N[v_j]|\leq q$,
\item $v_i$ is $\alpha$-strong for $W\setminus\bigcup_{j<i}N[v_j]$,
\item $|N[v_i]\cap (W\setminus\bigcup_{j<i} N[v_j])|\geq \ell$,
\item $m\leq k$.
\end{itemize}
\end{definition}
Intuitively, all but at most $q$ elements of the set $W$ are covered by the $(v_i)_{i\le m}$.
Additionally, each element of the pseudo-cover dominates both an
$\alpha$-fraction of what remains to be dominated, and at least $\ell$ elements.
Note that with our choice of constants, if there are more than $q$ vertices not
covered yet, any vertex that covers an $\alpha$-fraction of what remains also
covers at least $\ell$ elements.


The next lemma shows how to derive the existence of pseudo-covers from
the existence of covers.

\begin{lemma}\label{lem:cover-to-pseudo-cover}
Let $W\subseteq V(G)$ be of size at least
$q$ and let~$Z$ be a cover of $W$ with~$k$ elements.
There exists an ordering of the vertices of $Z$ as $z_1,\ldots, z_k$
and $m\leq k$ such that $(z_1,\ldots, z_m)$ is a pseudo-cover of $W$.
\end{lemma}
\begin{proof}
 We build the order greedily by induction. We order the elements by neighborhood size, while removing the neighborhoods of the previously ordered vertices. More precisely, assume that $(z_1,\ldots,z_i)$ have been defined for \mbox{some~$i\ge 0$}. We then define $z_{i+1}$ as the element that maximizes $|N[z] \cap (W \setminus \bigcup_{j\le i}N[z_j])|$.

 Once we have ordered all vertices of $Z$, we define $m$ as the maximal integer not larger than $k$ such that for every $i \le m$ we have:
 \begin{itemize}
   \item $z_i$ is $\alpha$-strong for $W\setminus\bigcup_{j<i}N[z_j]$, and
   \item $|N[z_i] \cap (W \setminus \bigcup_{j\le i}N[z_j])| \ge \ell$.
 \end{itemize}

This made sure that $(z_1,\ldots, z_m)$ satisfies the last 3 properties of a
 pseudo-cover of $W$. It only remains to check the first one.
 To do so, we define $W' \coloneq W \setminus\bigcup_{i\le m}N[z_i]$. We want to prove
 that $|W'| \le q$. Note that because $Z$ covers~$W$, if $m=k$ we
 have $W'=\emptyset$ and we are done. We can therefore assume
 that $m<k$ and $W'\neq \emptyset$. Since $Z$ is a cover of $W$,
 we also know that $(z_{m+1},\ldots z_k)$ is a cover of $W'$,
 therefore there is an element in $(z_{m+1},\ldots z_k)$ that
 dominates at least a $1 / k$ fraction of $W'$. Thanks to the
 previously defined order, we know that~$z_{m+1}$ is such element.
 Since $\alpha = 1/k$, it follows that~$z_{m+1}$ is $\alpha$-strong
 for $W'$.
 This, together with the definition of $m$, we have that $|N[z_i] \cap (W \setminus \bigcup_{j\le i}N[z_j])| < \ell$ meaning that $|N[z_{m+1}] \cap W'| < \ell$. This implies that $|W'|/k < \ell$. And since $\ell = q/k$, we have $|W'|<q$.
%
  Hence, $(z_1,\ldots, z_m)$ is a pseudo-cover of $W$.
\end{proof}

%We will apply the pseudo-covers to neighborhoods
%of vertices.

While there can exist unboundedly many covers for a set $W\subseteq V(G)$,
the nice observation of Czygrinow et al.\ was that the number of
pseudo-covers is bounded whenever the input graph excludes some
biclique $K_{s,t}$ as a subgraph. We do not state the result in this
generality, as it leads to enormous constants. Instead, we focus on the
case where small covers exist, that is, on the case where~$\nabla_1(G)$
is bounded and optimize the constants for this case.

\begin{lemma}\label{lem:num-high-degree}
Let $W\subseteq V(G)$ of size at least $8\nabla_1 / \alpha^2$.
Then there are at most $4\nabla_1/\alpha$ vertices that are
$\alpha$-strong for $W$.
\end{lemma}
\begin{proof}
  Assume that there is such a set $W$ with at least $c\coloneq 4\nabla_1 /\alpha$ vertices that are $\alpha$-strong for $W$.
  We build a $1$-shallow minor $H$ of the graph $G$ with $|W|$ branch sets.
  Each branch set is either a single element of $W$, or a pair $\{w,a\}$, where $w$ is in $W$ and $a$ is an $\alpha$-strong vertex for $W$, connected to $w$, and that is not in $W$. This is obtained by iteratively contracting one edge of an $\alpha$-strong vertex with a vertex of $W$. This is possible because $\alpha |W|>c$, so during the process and for any \mbox{$\alpha$-strong} vertex we can find a connected vertex in $W$ that is not part of any contraction.

  Once this is done, we have that $|V_H|=|W|$. For the edges, each of the \mbox{$\alpha$}-strong vertices can account for $\alpha |W|$ many edges. We need to subtract~$c^2$ from the total as we do not count twice an edge between two strong vertices. Therefore $|E_H| \ge c\alpha |W| - c^2$. Note also that because $|W| \ge 8\nabla_1 / \alpha^2$, we have that $ 2\nabla_1\ge (4\nabla_1)^2 / (\alpha^2|W|)$. All of this together leads to:

  $$\frac{|E_H|}{|V_H|} \ge \frac{c\alpha |W| - c^2}{|W|} \ge 4 \nabla_1 - \frac{(4\nabla_1)^2}{\alpha^2|W|} \ge 4\nabla_1 - 2\nabla_1 > \nabla_1 $$

  This contradicts the definition of $\nabla_1$.
\end{proof}

This leads quickly to a bound on the number of pseudo-covers.

\begin{lemma}\label{lem:num-pseudo-covers}
For every $W\subseteq V(G)$ of size at least $\ell$, the number of pseudo-covers is bounded by $2(4\nabla_1(G)/\alpha)^k$.
\end{lemma}

The proof of the lemma is exactly as the proof of Lemma~7 in the
presentation of Czygrinow et al.~\cite{czygrinow2018distributed},
we therefore refrain from repeating it here.

%
%Having previous decided to fix $K$ and $\alpha$, \cref{lem:num-pseudo-covers}
%yields the following handy corollary.
%\begin{corollary}\label{cor:num-pseudo-covers}
%  Let $G$ be a graph. Fixe $k\coloneq 2\nabla_1(G)$, and then define
%  $\alpha = 1/k$, $K=k$, $\ell =4k^3+1$, and $q:= 4k^4$.
%  Then for every $W\subseteq V(G)$ of size at least $\ell$, the number of
%  $(\alpha,q,\ell,K)$-pseudo-covers of $W$
%  is bounded by $2(2k^2)^k$.
%\end{corollary}

% \textcolor{blue}{
% \begin{example} Let $G$ be a planar graph. We can strongly optimize
% the construction. For $\alpha>0, \ell\geq 8, q\geq 0$ and
% $K\leq 6$ the number of vertices appearing in any $(\alpha, q,\ell, K)$-pseudo-cover
% of $N(v)$ for a vertex $v$ is at most $7$.
% The neighborhood $N(v)$ is dominated by $v$ and also by
% at most $6$ vertices from $D'$. Then there does not exist
% a further vertex that dominates at least $8$ vertices of
% $N(v)$ as $G$ is planar. PROOF+PICTURE.
% \end{example}
% }

\begin{tcolorbox}
We write $\Tt(v)$ for the set of all pseudo-covers
of $N(v)$ and $\Pp(v)$ for the set of all vertices that appear in a
pseudo-cover of $N(v)$.
\end{tcolorbox}



\begin{corollary}\label{cor:nb-dominators}
For every $v\in V(G)$ with $|N(v)|> \ell$, we have
  $|\Tt(v)|\le 2(2k^2)^k$ and $|\Pp(v)|\le 2k(2k^2)^k\le (2k)^{2k+1}$.
\end{corollary}
