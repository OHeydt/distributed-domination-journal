% !TEX root = sirocco-main.tex

\section{Phase 3: LP-based approximation}\label{sec:LP}

\subsection{LP-based approximation}
In the light of \cref{lem:res-degree}, we could now simply choose
$D_3$ as the set of elements not in $N[D_1\cup D_2]$.  We would get a
constant factor approximation, but not a very good one. Instead,
we now proceed with an LP-based approximation. The dominating
set problem can be formulated as an integer linear program~(ILP).
Note that it remains to dominate the set $R$, which leads to the
following ILP.

\[
  \begin{array}{l l l}
    \text{Minimize }    & \sum_{v\in V}x_v &\\
    \text{Subject to }\quad & \sum_{u\in N[v]}x_u \ge 1 \quad &\forall v\in R \\
                            & x_v\in \{0,1\}                   &\forall v \in V\\
  \end{array}
\]

By relaxing the condition that $x_v\in \{0,1\}$ to $x_v\in [0,1]\subseteq \mathbb{R}$,
we obtain the corresponding linear program (LP). By a result of
Bansal and Umboh~\cite{bansal2017tight} one can obtain a constant
factor approximation of a dominating set from a solution to the LP.
The proof can easily be adapted to the problem of approximating a
dominating set of the set $R$.

\begin{lemma}
  Let $G$ be a graph that has an orientation with maximum out-degree $d$
  and let $R\subseteq V(G)$. Let $D_R\subseteq V(G)$ be a minimum dominating
  set of $R$.
  Let $\big(x_v\big)_{v\in V(G)}$ be a solution to the
  $R$-dominating set~LP. Let $H:=\{v\in V(G) : x_v\ge 1/(2d+1)\}$ and
  let $U:=\{v\in R : v\not\in N[H]\}$. Then $H\cup U$ dominates~$R$
  and has size at most $(2d+1)\cdot |D_R|$.
\end{lemma}

Observe that when given the solution $\big(x_v\big)_{v\in V(G)}$ to the
$R$-dominating set~LP the lemma gives rise to a simple LOCAL algorithm.
First select all vertices $v$ with $x_v\geq 1/(2d+1)$ into a dominating set
and mark all their neighbors as dominated. Then select all non-dominated
vertices of $R$ into the dominating set. Clearly, $H\cup U$ is a dominating
set of $R$. The rest of this section is devoted to the proof of the claimed
approximation factor. The proof follows the presentation of Bansal and
Umboh~\cite{bansal2017tight} with the improved bounds of Dvo\v{r}\'ak~\cite{dvovrak2019distance}. As every solution to the ILP is also a solution
to the LP we have $\sum_{v\in V(G)}x_v\leq |D_R|$.

Consider an orientation of $G$ such that the neighborhood of each vertex
$v$ is decomposed into $N^{in}(v)$ and $N^{out}(v)$, where $|N^{out}(v)|\le d$.

\begin{claim}
  For every vertex $v\in U$, we have
  $\big(\sum_{u\in N^{in}(v)}x_u \big)\ge d/(2d+1)$.
\end{claim}
\begin{proof}
  As $v$ is not in $H$, $x_v<1/(2d+1)$. As $v$ is not in $N(H)$, for every
  vertex $u\in N^{out}(v)$ we have $x_u<1/(2d+1)$. As $|N^{out}(v)|\le d$,
  and by the first LP condition
  $\big(\sum_{u\in N^{in}(v)}x_u \big)\ge 1- \frac{1}{2d+1} - \frac{d}{2d+1}
  \ge \frac{d}{2d+1}$.
\end{proof}

We can now bound the size of $U$ and $H$
\begin{claim}
  $|H\cup U| \le (2d+1)\sum_{v\in V}x_v$.
\end{claim}
\begin{proof}
  First, observe that
  $|H|\le (2d+1)\sum_{v\in H}\frac{1}{2d+1}
  \le (2d+1)\sum_{v\in H}(x_v)$.
  %
  Then observe that\linebreak
  %
  $|U|\le \frac{2d+1}{d}\cdot\sum_{v\in U} \frac{d}{2d+1}
  \le \frac{2d+1}{d}\sum_{v\in U}\sum_{u\in N^{in}(v)}x_u
  \le \frac{2d+1}{d}\sum_{u\in N^{in}(U)} (d\cdot x_u)
  \le (2d+1) \sum_{u\in N^{in}(U)} x_u$.

  By definition of $U$, we have that $N(U)$ and $H$ are disjoint, this also
  holds for~$H$ and $N^{in}(U)$ hence $|H\cup U| \le (2d+1)\sum_{v\in V}x_v\leq (2d+1)|D_R|$.
\end{proof}

\subsection{Solving LPs locally}

As shown by Kuhn et al.~\cite{kuhn2006price} we can locally approximate
general covering LPs, in particular the above $R$-dominating set LP,
when the maximum
degree of the graph is bounded. More precisely, they show how to compute
a $\Delta^{1/k}$-approximation in $\Oof(k^2)$ rounds. Assuming for a
moment that $\Delta$ is bounded by an absolute constant we can
choose $k$ such that $\Delta^{1/k}=1+\e$,
hence $k=(\log \Delta)/(\log (1+\e))$, which is a constant depending
only on $\Delta$ and $\e$ in order to compute a
$(1+\e)$-approximation. Hence, it remains to establish the situation
that not only the residual degrees are bounded but that $\Delta$ is
bounded by an absolute constant and to choose $\e$ appropriately.

\smallskip
Recall that we chose $\rho$ such that $|(D_1 \cup D_2)\cap D| =\rho\gamma$.
When $\rho\gamma$ vertices of $D$ were already chosen into
the partial dominating set $D_1\cup D_2$ we have $|D_R|\leq (1-\rho)\gamma$.
With \cref{cor:planar-orientations} we conclude the following corollary.

\begin{cor}\label{cor:LP-approx}
Let $G$ be a graph that has an orientation with maximum out-degree~$d$, let $R\subseteq V(G)$, let~$D_R$ be a
minimum dominating set of $R$, and let $\e>0$. Then we can
compute a set $D'$ of size at most~$(2d+1)(1+\e)|D_R|$ that dominates~$R$ in $\Oof(\log \Delta/(\log (1+\e))$ rounds in the LOCAL
model.

\vspace{1mm}
In particular, for our algorithm when

\vspace{-2mm}
\begin{enumerate}
\item $G$ is planar, then $|D'|\leq 7(1+\e)|D_R|
  \leq 7(1+\e)(1-\rho)\gamma$ and when \smallskip
\item $G$ is planar and triangle-free or outerplanar, then
  $|D'|\leq 5(1+\e)|D_R|\leq 5(1+\e)(1-\rho)\gamma$.
\end{enumerate}
\end{cor}

\subsection{From bounded residual degree to bounded degree}

It remains to establish the situation that the maximum degree $\Delta$
of our graph is bounded. As argued, we have $|R|\leq 31(1-\rho)\gamma$.
As only
the vertices of $R$ need to be
dominated it suffices to keep only the vertices that have a
neighbor in $R$; other vertices are not useful as dominators.
Also, when two vertices $u,v\in V(G)\setminus R$ have exactly
the same neighbors in $R$, that is, $N_R(u)=N_R(v)$,
it suffices to keep one of $u$ and $v$.
Note that we can locally decide whether $N_R(u)=N_R(v)$.
For every set $N\subseteq R$ such that there is a vertex $v$
with $N_R(v)=N$ we choose the one with the lowest identifier
as a representative. We construct the graph $G'$ consisting
of~$R$ and all edges between vertices in $R$ as well as the
set of all representatives and a minimal set of edges such that
$N_R(v)$ is equal in $G$ and $G'$ for all representatives~$v$.
Hence in $G'$ we have $N_R(u)\neq N_R(v)$ for all $u\neq v\in V(G')\setminus R$.
As argued above, every $R$-dominating set in $G$ can
be transformed into an $R$-dominating set of the same size
in $G'$ (by choosing the appropriate representative) and
every $R$-dominating set in $G'$ is an $R$-dominating set in~$G$.
We can hence continue to work with the graph $G'$. In order to
avoid complicated notation we simply assume that $G=G'$.

Note that in general we could have $|V(G)|\in \Omega(2^{|R|})$,
in a planar graph however, $|V(G)|$ is linear in~$|R|$, which is
crucial for our further argumentation.

\begin{lemma}\label{lem:smallX}
  $|V(G)|\le 12|R|$.
\end{lemma}
We follow the presentation of \cite[Lemma~4.3]{gajarsky2017kernelization}.
The presented construction is not build by our algorithm, it simply shows that
our graph $G$ satisfies the above property.
\begin{proof}
  Consider a sequence of graphs $G_0, G_1, \ldots, G_\ell$ such that
  $G_i$ is a $1$-shallow minor of $G$ for all $0 \le i \le \ell$ as follows.
  Set $G_0 = G$, and for $0 \le i\le\ell-1$ define $G_{i+1}$ from $G_i$ by
  choosing a vertex $v\in V (G_i ) \setminus R$ such that $N_R(v)$
  contains two non-adjacent vertices $u, w$ in $G_i$ and contract $\{u,v\}$ into
  the vertex $u$ to obtain~$G_{i+1}$.
  Note that contracting $\{u,v\}$ into $u$ is equivalent to deleting vertex $v$ and
  adding edges between each vertex in $N(v) \setminus u$ and $u$. Note that this
  contraction will only add edges to $R$ and remove vertices from $V(G)\setminus R$.
  Hence, for $0\le i\le\ell$, we maintain $R \subseteq V(G_i)$.

  This process clearly terminates, and $G_{i+1}$ has at least one more edge
  between vertices of $R$ than $G_i$ does. Note that $G_i$ is a $1$-shallow minor
  of $G$ for $0 \le i \le \ell$, as the edges $e_1, \ldots, e_{i-1}$ that were
  contracted to vertices in $R$ in order to construct~d$G_i$ had one endpoint
  each in $R$ and $V(G)\setminus R$, the endpoint in $V(G)\setminus R$
  being deleted after each contraction.
  Thus, $e_1, \ldots, e_{i-1}$ induce a set of stars in $V(G)$, and $G_i$ is
  obtained from $G$ by contracting these stars. We therefore conclude that
  $G_i$ is a $1$-shallow minor of $G$. In particular, this implies that
  $G$ is planar and has at most $3|R|-6$ edges between vertices of $R$.

  Since there are at most $3|R|-6$ edges between vertices of $R$, we have that
  $\ell \le 3|R|-6$.
  Since the iterative process stopped, we have that
  for every vertex $v\in V(G_\ell)\setminus R$, $N_R(v)$ is a clique in $G_\ell$.

  We conclude with a result of Wood~\cite[Corollary~2]{wood2007maximum}
  showing that in an $n$-vertex planar graph there can be at most $8(n-2)$
  many cliques. At most $3|R|-6$ vertices $u$ with a possibly unique
  neighborhood $N_R(u)$ were contracted and at most $8(|R|-2)$
  vertices $v$ with different $N_R(v)$ that induce a clique in $G_\ell$ remain.
  We add the vertices of $|R|$ to conclude that $|V(G)|\leq 12|R|$.
\end{proof}

By \cref{lem:res-degree} we have $|R|\le 31(1-\rho)\gamma$, which immediately
implies the following corollary.

\begin{cor}\label{cor:size-g}
  $|V(G)| \le 372(1-\rho)\gamma$.
\end{cor}


\subsection{Conclusion of the algorithm}


We now pick an arbitrary $\e>0$ and select all vertices with high degree
$\Gamma$ into our
dominating set, where $\Gamma$ is chosen such that there exist at
most $(\e/8)(1-\rho)\gamma$ vertices of degree at least $\Gamma$.

\begin{tcolorbox}[colback=red!5!white,colframe=red!50!black]
  Let $\e>0$, $\e':= \e/8$,
  $\Gamma\coloneqq 6\frac{372}{\e'}$, and \quad
  $D_3^1\coloneqq \{v\in V(G) ~:~ d(v)>\Gamma\}$.
\end{tcolorbox}

\begin{lemma}\label{lem:size-D31}
  $|D_3^1|\le \e'(1-\rho)\gamma$.
\end{lemma}
\begin{proof}
  We assume the opposite and count the number of edges and vertices of~$G$.
  When we sum the degree of the vertices, we get twice the number of
  edges. Hence $2\cdot |E(G)|\ge 6\frac{372}{\e'} \cdot \e'(1-\rho)\gamma$.
  Therefore, with \cref{cor:size-g},
  $|E(G)|\ge 3\cdot 372(1-\rho)\gamma\geq 3|V(G)|$. This
  contradict the fact the graph is planar.
\end{proof}

After picking $D_3^1$ into the dominating set, marking the neighbors of
$D_3^1$ as dominated and updating the set $R$, we can delete the
vertices of $D_3^1$. We are left with a graph of maximum degree $\Gamma$.

\begin{tcolorbox}[colback=red!5!white,colframe=red!50!black]
  Let $D_3^2$ be the set computed by the LOCAL algorithm of \cref{cor:LP-approx}
  with parameter $\e'$.
\end{tcolorbox}

Let $D_3=D_3^1 \cup D_3^2$. We already noted that the definition of $D_3$ implies that
$D_1\cup D_2\cup D_3$ is a dominating set of $G$. We now conclude the
analysis of the size of this computed set.

\begin{lemma}\label{lem:D3-LP}
  We have that $|D_3| \le (7+\e)(1-\rho)\gamma$
\end{lemma}
\begin{proof}
  By \cref{cor:LP-approx} and \cref{lem:size-D31} we have
  $|D_3^2|\leq 7(1+\e')(1-\rho)\gamma$, and $|D_3^1|\le \e'(1-\rho)\gamma$.
  This with,~$\e'=\e/8$ conclude the proof.
\end{proof}

% Let $D_3=D_3^1 \cup D_3^2$. We already noted that the definition of $D_3$ implies that
% $D_1\cup D_2\cup D_3$ is a dominating set of $G$. We now conclude the
% analysis of the size of this computed set.
%
% First, by
% \cref{lem:size-D12} we have $|D_1\cup D_2|<4\gamma+4\rho\gamma$.  Then,
% by \cref{lem:size-D31} we have $|D_3^1|\le \e'(1-\rho)\gamma$
% and by \cref{cor:LP-approx} we have $|D_3^2|\leq 7(1+\e')(1-\rho)\gamma$.
%
% Therefore $|D_1 \cup D_2 \cup D_3|\leq \gamma(4+4\rho +\e'(1-\rho)+7(1+\e')(1-\rho))
% \leq \gamma(11+8\e'-3\rho -8\e'\rho)$.\linebreak
% As $\rho\in[0,1]$, this is maximized when $\rho=0$. Hence
% $|D_1 \cup D_2 \cup D_3| < \gamma(11+8\e') \le \gamma(11+\e)$.

We can now conclude the proof of our main result.
\begin{theorem}\label{thm:planar}
  There exists a distributed LOCAL algorithm that, for every $\e>0$, and every
  planar graph~$G$, computes in a constant number of rounds a dominating set
  of size at most $(11+\e)\gamma(G)$.
\end{theorem}
\begin{proof}
  First, $D_1,D_2$, and $D_3$ are computed locally, in a bounded number of
  rounds, and $D_1 \cup D_2 \cup D_3$ dominates $G$.
%
  Second, by \cref{lem:size-D12} we have $|D_1\cup D_2|<4\gamma+4\rho\gamma$.
  And, by \cref{lem:D3-LP} we have $|D_3|\leq (7+\e)(1-\rho)\gamma$.

  Therefore $|D_1 \cup D_2 \cup D_3|\leq \gamma(4+4\rho +7 - 7\rho +\e - \e\rho)
  \leq \gamma(11+\e-3\rho -\e\rho)$.
  As $\rho\in[0,1]$, this is maximized when $\rho=0$. Hence
  $|D_1 \cup D_2 \cup D_3| \le \gamma(11+\e)$.

\end{proof}
