% !TEX root = sirocco-main.tex

\section{Introduction}\label{sec:intro}

A dominating set in an undirected and simple graph $G$ is a set
$D\subseteq V(G)$ such that every vertex $v\in V(G)$ either belongs
to $D$ or has a neighbor in $D$. The dominating set problem is a
classical NP-complete problem~\cite{karp1972reducibility} with many
applications in theory and practice, see e.g.~\cite{du2012connected,sasireka2014applications}. In this paper we study
the distributed time complexity of finding
dominating sets in planar graphs in the classical LOCAL model of distributed computing.
In this model, a distributed system is modeled
by an undirected (planar) graph~$G$. Every vertex represents a
computational entity and the vertices communicate through the edges of
$G$. The vertices are equipped with unique identifiers and initially,
every vertex is only aware of its own identity. A computation then
proceeds in synchronous rounds.  In every round, every vertex sends
messages to its neighbors, receives messages from its neighbors and
performs an arbitrary computation.  The complexity of a LOCAL
algorithm is the number of rounds until all vertices return their
answer, in our case, whether they belong to a dominating set
or~not.


The problem of approximating
dominating sets in the LOCAL model has received considerable
attention in the literature. Since in general graphs
it is not possible to compute a
constant factor approximation in a constant number of rounds~\cite{KuhnMW16},
much effort has been invested to improve the ratio between approximation
factor and number of rounds on special graph classes. A very successful
line of structural analysis of graph properties that can lead to improved
algorithms was started by the influential paper of Lenzen et al.~\cite{lenzen2013distributed}, who in particular proved that on planar graphs
a 130-approximation
of a minimum dominating set can be computed in a constant number of
rounds. A careful analysis of Wawrzyniak~\cite{wawrzyniak2014strengthened}
later showed that the algorithm computes in fact a 52-approximation.
In terms of lower bounds, Hilke et al.~\cite{hilke2014brief} showed that there is no
deterministic local algorithm (constant-time distributed graph algorithm) that
finds a~$(7-\epsilon)$-approximation of a minimum dominating set on
planar graphs, for any positive constant~$\epsilon$. Better approximation
ratios are known for some special cases, e.g.\ 32 if the planar graph is
triangle-free \mbox{\cite[Theorem 2.1]{alipour2020distributed}}, 18 if the planar graph has girth
five~\cite{alipour2020local} and 5 if the graph is
outerplanar (and this bound is tight)~\cite[Theorem 1]{bonamy2021tight}.

In this work we tighten the gap between the best-known
lower bound of~$7$ and the best-known upper bound of $52$ on planar graphs
by providing a new approximation algorithm computing an $(11+\e)$\hspace{1pt}-\hspace{1pt}approximation (for every $\e>0$).

Our algorithm proceeds in three phases.
The first phase is a preprocessing phase that was similarly employed in the
algorithm of Lenzen et al.~\cite{lenzen2013distributed}. In a key lemma,
Lenzen et al.\ proved that there are only few vertices whose open neighborhood
cannot be dominated by at most six vertices.
We improve this lemma and show
that there are only slightly more vertices whose open neighborhood cannot be
dominated by \emph{three} other vertices. All these vertices are selected into an
initial partial dominating set and as a consequence the open neighborhoods of all
remaining vertices can be dominated by at most three vertices.

%\bigskip
%\noindent\hrulefill
%\vspace{-6mm}
%\subsubsection*{Acknowledgements} We thank the anonymous
%referee for pointing out that the third phase of our algorithm can be
%improved by the use of LP-based techniques when the maximum
%degree is bounded.


By defining the notion of \emph{pseudo-covers},
Czygrinow et  al.~\cite{czygrinow2018distributed} provided a tool to carry
out a fine grained analysis of the vertices that can potentially dominate
the remaining neighborhoods.
Using ideas of~\cite{kublenz2020distributed} and~\cite{siebertz2019greedy}
we provide an even finer analysis for planar graphs on which we
base the second phase of our distributed algorithm and compute a
second partial dominating set.


For the third phase, call the number of non-dominated neighbors of a vertex~$v$
the \emph{residual degree} of $v$. We prove that
after the second phase we are left with a graph where every vertex
has residual degree at most~$30$. In particular, every vertex
from a minimum dominating set $D$ can dominate at most 30
non-dominated vertices and we conclude that the set $R$
of non-dominated vertices has size bounded by~$31|D|$ (each
vertex dominates its neighbors and itself). Hence, we could at this
point pick all non-dominated vertices to add at most $31|D|$ vertices
and conclude. Instead, we study two different ways to proceed.

Our first option is to apply an LP-approximation based on
aresults of Bansal and Umboh~\cite{bansal2017tight}, who showed
that a very simple selection procedure leads to a constant factor
approximation when the solution to the dominating set linear
program (LP) is given. As shown by Kuhn et al.~\cite{kuhn2006price} we can approximate such a solution in a
constant number of rounds when the maximum degree $\Delta$
of the graph is bounded.
To apply these results we have to overcome two obstacles. First,
note that even though we have established that the maximum residual
degree is bounded by $30$, we may still have unbounded maximum
degree $\Delta$. We overcome this problem by keeping only a few
representative potential dominators around the set $R$. By a simply
density argument there can be only very few high degree vertices left
that we simply select into the dominating set. The second obstacle,
which is easily overcome, is that we do not need to dominate the
whole remaining graph but only the set $R$. This requires a small
adaptation of the LP-formulation of the problem and a proof that the
algorithm of Bansal and Umboh still works for this slightly different problem.
In total, in this third phase of the algorithm we add at most $(7+\e)|D|$ vertices to
the dominating set, leading to an
$(11+\e)$\hspace{1pt}-\hspace{1pt}approximation in total (\cref{thm:planar}).
%On
%the downside, we use quite heavy machinery and the number of
%rounds depends on the LOCAL LP-approximation algorithm of Kuhn et al.,
%which is constant but large.

\smallskip
Our second option is to study a parallel distributed version of the
classical greedy algorithm. We proceed in a greedy manner in $30$ rounds as follows. In the first round, if a non-dominated
vertex has a neighbor of residual degree~$30$, it elects one such neighbor into
the dominating set (or if it has residual degree $30$ itself, it may choose itself).
The neighbors of the chosen elements are
marked as dominated and the residual degrees are updated. Note that
all non-dominated neighbors of a vertex of residual degree $30$
in this round choose a dominator, hence, the residual degrees
of all vertices of residual degree $30$ are decreased to~$0$, hence, after
this round there are no vertices of residual degree $30$ left.
In the second round, if a non-dominated vertex has a neighbor
of residual degree~$29$, it elects
one such vertex into the dominating set, and so on, until after $30$ rounds
in the final round every vertex chooses a dominator. Unlike in the general
case, where
nodes cannot learn the current maximum residual degree in a constant
number of rounds, by establishing
an upper bound on the maximum residual degree and proceeding in exactly
this number of rounds, we ensure that we iteratively exactly choose the
vertices of maximum residual degree. It remains to analyze the performance
of this algorithm.


A simple density argument shows
that there cannot be too many vertices of degree $i\geq 6$ in a planar
graph. At a first glance it seems that the algorithm would perform worst
when in every of the $30$ rounds it would pick as many vertices as possible,
as the constructed dominating set would grow as much as possible. However,
this is not the case, as picking many high degree vertices at the same time makes
the largest progress towards dominating the whole graph. It turns
out that there is a delicate balance between the vertices that we pick
in round $i$ and the remaining non-dominated vertices that leads
to the worst case. 
%We formulate these conditions as a
%linear program and solve the linear program. 
In total, this
leads to a 20\hspace{1pt}-\hspace{1pt}approximation (\cref{thm:planar-greedy}).
While this approach falls short of achieving the best approximation ratio
the algorithm is simple and interesting to study in its own right.

\smallskip
We then analyze our algorithm on more restricted graph classes, and prove that
it computes the following approximations of factors: ($8+\e$) for triangle-free
planar graphs, $(7+\e)$ for bipartite planar graphs, $(8+\e)$ for outerplanar
graphs, and $7$ for planar graphs of girth $5$
(\cref{thm:bip,thm:tri,thm:outer,thm:girth-2} respectively).
% We then analyze our algorithm on bipartite planar graphs,
% triangle-free planar graphs,
% planar graphs of girth $5$ and outerplanar graphs and prove that
% it computes a 12, 14, 7 and 12 approximation, respectively (\cref{thm:bip,thm:tri,thm:girth,thm:outer}).
This
improves the currently best known approximation ratios of 32
and 18 for triangle-free planar graphs and planar graphs of girth $5$,
respectively. %, while our algorithm fails short of achieving
%the optimal approximation ratio of 5 on outerplanar graphs.

%While we could tighten the gap between the best known lower and upper
%bound on planar graphs there is still much room for improvement. We
%believe that the optimum approximation rate is much closer to $7$ than
%to $24$. We also stress that we are not satisfied with our brute-force
%CPLEX proof, but we must unfortunately leave a more elegant proof
%for future work.





%
%, see e.g.~\cite{akhoondian2018distributed,%
%akhoondian2016local,%
%alipour2020local,%
%amiri2016brief,%
%amiri2019distributed,%
%barenboim2018fast,%
%czygrinow2008fast,%
%czygrinow2018distributed,%
%DBLP:conf/stoc/GhaffariKM17,%
%hilke2014brief,%
%kublenz2020distributed,%
%KuhnMW16,%
%lenzen2013distributed,%
%lenzen2008leveraging,%
%lenzen2010minimum,%
%DBLP:conf/stoc/RozhonG20,%
%wawrzyniak2013brief,%
%wawrzyniak2014strengthened}.
