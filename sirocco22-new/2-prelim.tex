% !TEX root = sirocco-main.tex

\section{Preliminaries}

In this paper we study the distributed time complexity of finding
dominating sets in planar graphs in the classical LOCAL model of
distributed computing.
We assume familiarity with this model and refer to the survey~\cite{suomela2013survey} for extensive
background on distributed computing and the LOCAL model.

We use standard notation from graph theory and refer to the
textbook~\cite{diestel} for extensive background.  All graphs in this
paper are undirected and simple. We write $V(G)$ for the vertex set of
a graph $G$ and~$E(G)$ for its edge set. The \emph{girth} of a graph
$G$ is the length of a shortest cycle in $G$. A graph is called
\emph{triangle-free} if it does not contain a triangle, that is, a
cycle of length three as a subgraph. Equivalently, a triangle-free
graph is a graph of girth at least four.

A graph is \emph{bipartite} if its vertex set can be partitioned into
two parts such that all its edges are incident with two vertices from
different parts. More generally, the \emph{chromatic number}~$\chi(G)$
of a graph $G$ is the minimum number $k$ such that the vertices of $G$
can be partitioned into $k$ parts such that all edges are incident
with two vertices from different parts. Hence, the bipartite graphs
are exactly the graphs with chromatic number two. A set $A$ is
\emph{independent} if all two distinct vertices $u,v\in A$ are
non-adjacent.  Every graph $G$ contains an independent set of size at
least $\lceil|V(G)|/\chi(G)\rceil$. Every bipartite graph is triangle-free.

A graph is \emph{planar} if it can be embedded in the plane, that is,
it can be drawn on the plane in such a way that its edges intersect
only at their endpoints. By the famous theorem of Wagner, planar
graphs can be characterized as those graphs that exclude the complete
graph $K_5$ on five vertices and the complete bipartite $K_{3,3}$ with
parts of size three as a minor.  A graph~$H$ is a \emph{minor} of a
graph~$G$, written~$H\minor G$, if there is a set
\mbox{$\{G_v :v\in V(H)\}$} of pairwise vertex disjoint and connected
subgraphs $G_v\subseteq G$ such that if~$\{u,v\}\in E(H)$, then there
is an edge between a vertex of~$G_u$ and a vertex of~$G_v$. We
call~$V(G_v)$ the \emph{branch set} of $v$ and say that it is
\emph{contracted} to the vertex~$v$.  We call $H$ a \emph{$1$-shallow
  minor}, written~$H\minor_1 G$, if $H\minor G$ and there is a minor
model \mbox{$\{G_v :v\in V(H)\}$} witnessing this, such that all
branch sets $G_v$ have radius at most $1$, that is, in each~$G_v$
there exists $w$ adjacent to all other vertices of $G_v$. In other
words, $H\minor_1 G$ if~$H$ is obtained from $G$ by deleting some
vertices and edges and then contracting a set of pairwise disjoint
stars. We refer to~\cite{nevsetvril2012sparsity} for an in-depth study
of the theory of sparsity based on shallow minors.

A graph is \emph{outerplanar} if it has an embedding in the plane such
that all vertices belong to the unbounded face of the embedding.
Equivalently, a graph is outerplanar if it does not contain the
complete graph $K_4$ on four vertices and the complete bipartite graph
$K_{2,3}$ with parts of size $2$ and $3$, respectively, as a minor. If
$J\minor H$ and $H\minor G$, then $J\minor G$, hence a minor of a
planar graph is again planar and a minor of an outerplanar graph is
again outerplanar.


By Euler's formula, planar graphs are sparse: every planar $n$-vertex
graph ($n\geq 3$) has at most $3n-6$ edges (and a graph with at most
two vertices has at most one edge). The ratio $|E(G)|/|V(G)|$ is
called the \emph{edge density} of $G$. In particular, every planar
graph $G$ has edge density strictly smaller than three.

\begin{lemma}\label{lem:densities}
  Let $G$ be a planar graph. Then the edge density of $G$ is strictly
  smaller than $3$ and $\chi(G)\leq 4$. Furthermore,

  \vspace{-2mm}
  \begin{enumerate}
  \item if $G$ is bipartite, then the edge density of $G$ is strictly
    smaller than $2$ and $\chi(G)\leq 2$,\smallskip
  \item if $G$ is triangle-free or outerplanar, then the edge density
    of $G$ is strictly smaller than $2$ and $\chi(G)\leq 3$.
    % and\smallskip
    % \item if $G$ has girth at least $5$, then the edge density of
    %   $G$ is strictly smaller than $\frac{5}{3}$ and
    %   $\chi(G)\leq 3$.
  \end{enumerate}
\end{lemma}

An orientation of a graph $G$ is a directed graph $\vec{G}$ that
for every edge $\{u,v\}\in E(G)$ has one of the arcs $(u,v)$ or
$(v,u)$. The out-degree $d^+(v)$ of a vertex~$v$ in an orientation $\vec{G}$ 
of $G$ is the number $|\{ w : (v,w)\in E(\vec{G}\}|$. The following
lemma is implicit in the work of Hakimi~\cite{sup_orie}, see also 
\mbox{\cite[Proposition 3.3]{nevsetvril2012sparsity}}. 

\begin{lemma}\label{lem:orientations}
Let $d=\lceil\max_{H\subseteq G}\{|E(H)|/|V(H)|\}\rceil$. Then $G$ has an 
orientation with maximum out-degree $d$. 
\end{lemma}
%\begin{proof}
%We start with an arbitrary orientation $\vec{G}$. Assume there is a vertex $v$ 
%with out-degree greater than~$d$. Consider the subgraph $\vec{H}$
%induced by all vertices that are reachable from $v$. The underlying undirected
%graph $H$ has $\sum_{v\in V(H)}d^+(v)$ many edges. As $v$ has 
%out-degree strictly larger than $d$ we conclude that there must be one 
%vertex $w$ in $\vec{H}$ that has out-degree strictly smaller than $d$. By definition
%of $\vec{H}$ there exists a directed path from $v$ to $w$. By reversing the 
%edges on that path we decrease the out-degree of $v$ by one, increase the
%out-degree of $w$ by one and leave the out-degrees of all other vertices of
%the path unchanged. By iterating this procedure we successively improve
%the orientation until no vertex of out-degree greater than $d$ remains. 
%\end{proof}

As every subgraph of a (triangle-free or outerplanar) planar graph 
is again a (triangle-free or outerplanar) planar graph we
have the following corollary. 

\begin{cor}\label{cor:planar-orientations}
Let $G$ be a planar graph. Then 

  \vspace{-2mm}
\begin{enumerate}
\item $G$ has an orientation with maximum out-degree $3$. \smallskip
\item Furthermore, if $G$ is triangle-free or outerplanar, then 
$G$ has an orientation with maximum out-degree~$2$. 
\end{enumerate}
\end{cor}

For a graph $G$ and $v\in V(G)$ we write
$N(v)=\{u~:~\{u,v\}\in E(G)\}$ for the \emph{open neighborhood} of $v$
and $N[v]=N(v)\cup\{v\}$ for the \emph{closed neighborhood}
of~$v$. For a set $A\subseteq V(G)$ let $N[A]=\bigcup_{v\in A}N[v]$.
% We write $N_r[v]$ for the set of vertices at distance at most $r$
% from a vertex $v$.
A~\emph{dominating set} in a graph~$G$ is a set $D\subseteq V(G)$ such
that $N[D]=V(G)$. We write $\gamma(G)$ for the size of a minimum
dominating set of $G$. For $W\subseteq V(G)$ we say that a set
$Z\subseteq V(G)$ \emph{dominates} $W$ if $W\subseteq N[Z]$.

\smallskip
In the following we mark important definitions and assumptions about
our input graph in gray boxes and steps of the algorithm in red boxes.

\begin{tcolorbox}
  We fix a planar graph $G$ and a minimum dominating set~$D$ of $G$
  with $\gamma \coloneqq |D|=\gamma(G)$.
\end{tcolorbox}
