% !TEX root = sirocco-main.tex


\section{Phase 2: Analyzing the local dominators}\label{sec:D2}

The second phase of our algorithm is inspired by results of Czygrinow
et al.~\cite{czygrinow2018distributed} and the greedy domination
algorithm for biclique-free graphs of~\cite{siebertz2019greedy}.
Czygrinow et al.~\cite{czygrinow2018distributed} defined the notion of
\emph{pseudo-covers}, which provide a tool to carry out a fine grained
analysis of vertices that can potentially belong to the sets $A_v$
used to dominate the red neighborhood~$N_R(v)$ of a vertex $v$. This
tool can in fact be applied to much more general graphs than planar
graphs, namely, to all graphs that exclude some complete bipartite
graph~$K_{t,t}$.  A refined analysis for classes of bounded expansion
was provided by Kublenz et al.~\cite{kublenz2020distributed}.  We
provide an even finer analysis for planar graphs on which we base a
second phase of our distributed algorithm.

We first describe what our algorithm computes, and then provide bounds
on the number of selected vertices. Intuitively, we select every pair
of vertices with sufficiently many neighbors in common.

\begin{tcolorbox}[colback=red!5!white,colframe=red!50!black]
\begin{itemize}
\item For $v\in V(G)$ let
  $B_v\coloneqq \{z\in V(G)\setminus \{v\}: |N_R(v)\cap N_R(z)|\geq
  10\}$.\smallskip
\item Let $W$ be the set of vertices $v\in V(G)$ such that
  $B_v \neq \emptyset$.\smallskip
\item Let $D_2\coloneqq \bigcup\limits_{v\in W} (\{v\}\cup B_v)$.
\end{itemize}
\end{tcolorbox}

Once $D_1$ has been computed in the previous phase, 2 more rounds of
communication are enough to compute the sets $B_v$ and $D_2$.
%
Before we update the residual degrees, let us analyze the sets $B_v$
and~$D_2$.  First note that the definition is symmetric: since
$N_R(v)\cap N_R(z)=N_R(z)\cap N_R(v)$ we have for all $v,z\in V(G)$ if
$z\in B_v$, then $v\in B_z$. In particular, if $v\in D_1$ or
$z\in D_1$, then $N_R(v)\cap N_R(z)=\emptyset$, which immediately
implies the following lemma.

% \begin{restatable}{lemma}{recapLemWAndD}\label{lem:WcapD1}
%   We have $W\cap D_1=\emptyset$ and for every $v\in V(G)$ we have
%   $B_v\cap D_1=\emptyset$.
% \end{restatable}

\begin{lemma}\label{lem:WcapD1}
  We have $W\cap D_1=\emptyset$ and for every $v\in V(G)$ we have
  $B_v\cap D_1=\emptyset$.
\end{lemma}

%\begin{proof}
%  Let $v,z\in V(G)$. If $v\in D_1$, then
%  $N_R(v)=\emptyset$. Similarly if $z\in D_1$, then
%  $N_R(v)\cap N(z)=N(v)\cap N_R(z) =\emptyset$.
%\end{proof}

Now we prove that for every $v\in W$, the set $B_v$ cannot be too big,
and has nice properties.
\begin{lemma}\label{lem:dominating-dominators}
  For all vertices $v\in W$ we have

  \vspace{-5pt}
  \begin{itemize}
  \item $B_v \subseteq A_v$ (hence $|B_v|\le 3$), and \smallskip
  \item if $v\not\in \hat{D}$, then $B_v\subseteq D$.
  \end{itemize}
\end{lemma}

\begin{proof}
  Assume $A_v=\{v_1,v_2,v_3\}$ (a set of possibly not distinct
  vertices) and assume there exists
  $z\in V(G)\setminus \{v,v_1,v_2, v_3\}$ with
  $|N_R(v) \cap N_R(z)| \geq 10$.  As $v_1, v_2, v_3$ dominate $N_R(v)$,
  and hence also \mbox{$N_R(v)\cap N_R(z)$}, there must be some~$v_i$,
  $1\leq i\leq 3$, with
  \mbox{$|N_R(v) \cap N_R(z) \cap N[v_i]| \geq \lceil 10/3\rceil \geq
    4$}.  Therefore, $|N_R(v) \cap N_R(z) \cap N(v_i)| \geq 3$,
  which shows that $K_{3,3}$ is a subgraph of~$G$, contradicting the
  assumption that $G$ is planar.

  If furthermore $v\not\in \hat{D}$, by definition of $\hat{D}$, we
  can find $w_1,w_2, w_3$ from $D$ that dominate $N(v)$, and in
  particular $N_R(v)$.  If $z\in V(G)\setminus \{v,w_1,w_2, w_3\}$
  with $|N_R(v) \cap N_R(z)| \geq 10$ we can argue as above to obtain
  a contradiction.
\end{proof}

% In the light of \cref{lem:dominating-dominators}, we select all
% paires of nodes with sufficiently large intersecting neighborhood.
%
% \begin{tcolorbox}
%   For $v\in V(G)$ let $B_v\coloneqq \{z\in V(G)\setminus \{v\}:
%   |N(v)\cap N(z)|\geq 19\}$.
% \end{tcolorbox}

% \begin{corollary}\label{cor:dominating-dominators}
%   For every vertex $v$, $B_v\subseteq A_v$, in particular,
%  $|B_v|\leq 6$ and if $v\not\in \hat{D}$, then $B_v\subseteq D$.
% \end{corollary}
%
% \begin{tcolorbox}
%   We define $W$ as the set of vertices $v\in V(G)$ such
%   that $B_v\neq \emptyset$. We define \[D_2\coloneqq \bigcup_{v\in W}
%   (\{v\}\cup B_v).\]
% \end{tcolorbox}
% \vspace{0mm}

% Our algorithm now proceeds as follows. Obviously, every vertex $v$
% can locally compute the set $B_v$. The algorithm adds the set $D_2$
% to the dominating set, removes~$D_2$ from the graph and marks all
% vertices dominated by $D_2$ as dominated.  \alex{here again 'remove'
% vertices?}

Let us now analyze the size of $D_2$. For this we refine the set $D_2$
and define
\begin{tcolorbox}
  \begin{enumerate}
    \item $D_2^1\coloneqq \bigcup_{v\in W\cap D}
    (\{v\}\cup B_v)$, \smallskip
    \item $D_2^2\coloneqq \bigcup_{v\in W\cap (\hat{D}\setminus D)}
    (\{v\}\cup B_v)$, and \smallskip
    \item $D_2^3\coloneqq \bigcup_{v\in W\setminus (D\cup \hat{D})}
    (\{v\}\cup B_v)$.
  \end{enumerate}
\end{tcolorbox}

\smallskip
Obviously $D_2=D_2^1\cup D_2^2\cup D_2^3$. We now bound the size of the
refined sets $D_2^1,D_2^2$ and $D_2^3$.

\begin{lemma}\label{lem:size-D21}
  $|D_2^1\setminus D|\leq 3\gamma$.
\end{lemma}
\begin{proof}
  We have
  \[|D_2^1\setminus D|= |\bigcup_{v\in W\cap D} (\{v\}\cup
    B_v)\setminus D|\leq |\bigcup_{v\in W\cap D}B_v|\leq \sum_{v\in
      W\cap D}|B_v|.\] By \cref{lem:dominating-dominators} we have
  $|B_v|\leq 3$ for all $v\in W$ and as we sum over $v\in W\cap D$ we
  conclude that the last term has order at most $3\gamma$.
\end{proof}

\begin{lemma}\label{lem:size-D22}
  $D_2^2 \subseteq \hat D$ and therefore
  $|D_2^2\setminus D|< 4\gamma$.
\end{lemma}
\begin{proof}
  Let $v\in \hat{D}\setminus D$ and let $z\in B_v$. By symmetry,
  $v\in B_z$ and according to \cref{lem:dominating-dominators}, if
  $z\not\in \hat{D}$, then~$v\in D$.  Since this is not the case, we
  conclude that $z\in\hat{D}$.  Hence $B_v\subseteq \hat{D}$ and, more
  generally, $D_2^2\subseteq \hat{D}$.  Finally, according to
  \cref{lem:D-hat} we have $|\hat{D}\setminus D|<4\gamma$.
\end{proof}

Finally, the set $D_2^3$, which appears largest at first glance, was
actually already counted, as shown in the next lemma.
\begin{lemma}\label{lem:size-D23}
  $D_2^3\subseteq D_2^1$.
\end{lemma}
\begin{proof}
  If $v\not\in \hat{D}$, then $B_v\subseteq D$ by
  \cref{lem:dominating-dominators}.  Hence $v\in B_z$ for some
  $z\in D$, and $v\in D_2^1$.
\end{proof}

Again, it is intuitively clear that the situation when the sets
$D_2^i$ are large does not lead to the worst case for the overall
algorithm. For example, when $D_2^1$ is large we have added many
vertices of the optimum dominating set $D$. For a formal analysis, we
analyze the number of vertices of $D$ that have been selected so far.

\begin{tcolorbox}
  Let $\rho\in [0,1]$ be such that $|(D_1 \cup D_2)\cap D| =\rho\gamma$.
\end{tcolorbox}
% \alex{$\rho$ instead of $\e$ for the LP approx, and $\delta$ for the residual degree}

\smallskip
\begin{lemma}\label{lem:size-D12}
  We have $|D_1\cup D_2| < 4\gamma+4\rho\gamma$.
\end{lemma}
\begin{proof}
  By \cref{lem:size-D23} we have $D_2^3\subseteq D_2^1$, hence,
  $D_1\cup D_2=D_1\cup D_2^1\cup D_2^2$. By \cref{lem:D-hat} we have
  $D_1 \subseteq \hat D$ and by \cref{lem:size-D22} we also have
  $D_2^2 \subseteq \hat D$, hence $D_1\cup D_2^2\subseteq \hat D$.
  Again by \cref{lem:D-hat}, $|\hat D \setminus D|<4\gamma$ and
  therefore $|(D_1 \cup D_2^2 )\setminus D| < 4 \gamma$.

  We have $W\cap D\subseteq D_2^1\cap D$, hence with
  \cref{lem:dominating-dominators} we conclude that
  \[
    \big\vert D_2^1 \setminus D \big\vert \leq
    \Big\vert\bigcup\limits_{v\in D \cap D_2^1}B_v\Big\vert \leq
    \sum\limits_{v\in D \cap D_2^1} |B_v| \leq 3\rho\gamma,
  \]
  hence $(D_1\cup D_2)\setminus D<4\gamma+3\rho\gamma$. Finally,
  $D_1\cup D_2=(D_1\cup D_2)\setminus D\cup ((D_1\cup D_2)\cap D)$ and
  with the definition of $\rho$ we conclude
  $|D_1\cup D_2|<4\gamma + 4\rho\gamma$.
\end{proof}

The analysis of the next and final step of the algorithm will actually
show that the worst case is obtained when $\rho=0$.

We now update the residual degrees, that is, we update $R$ as
$V(G)\setminus N[D_1\cup D_2]$ and for every vertex the number
$\delta_R(v)=N(v)\cap R$ accordingly.
%
We finally show that after the first two phases of the algorithm we
are in the very nice situation where all residual degrees are
small. %This will allow us to proceed in a greedy manner.

  \begin{lemma}\label{lem:res-degree}
    For all $v\in V(G)$ we have $\delta_R(v)\leq 30$.
  \end{lemma}
  \begin{proof}
    First, every vertex of $D_1\cup D_2$ has residual degree $0$.
    Assume that there is a vertex $v$ of residual degree at least $31$.
    As $v$ is not in $D_1$, its $31$ non-dominated neighbors are
    dominated by a set $A_v$ of at most 3 vertices. Hence there is a
    vertex~$z$ (not in $D_1$ nor $D_2$) with $|N_R(v)\cap N_R[z]|\geq
    \lceil 31/3\rceil = 11$, hence, $|N_R(v)\cap N_R(z)|\geq 10$.
	This contradicts that $v$ is not in~$D_2$.
  \end{proof}
