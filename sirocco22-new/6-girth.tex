% !TEX root = sirocco-main.tex


\section{Restricted classes of planar graphs}
In this section we further restrict the input graphs, requiring e.g.\ planarity
together with a lower bound on the girth. Our algorithm works exactly as before, however,
using different parameters. From the different edge densities and
chromatic numbers of
the restricted graphs we will then derive different constants and
as a result a better approximation factor. Throughout this section
we use the same notation as in the first part of the paper and state
in the adapted lemmas with the same numbers as in the first part
the adapted sizes of the respective sets.

As in the general case in the first phase we begin by computing
the set $D_1$ and analyzing it in terms of the auxiliary set $\hat{D}$.

%The proofs of most lemmas
%can be easily adapted from the general case and we will only go into
%detail when needed. In \cref{sec:triangle-free} we prove that on
%triangle-free planar graphs we can compute a 16-approximation, improving
%the currently best known bound of 32. In \cref{sec:girth} we prove
%that on planar graphs of girth 5 we can compute a 9-approximation,
%improving the best known bound of 18, and finally in \cref{sec:outer}
%we prove that on outerplanar graphs our algorithm computes a 13-approximation. This does not improve the tight known bound
%of 5, however, it demonstrates that our algorithm works robustly
%on this interesting subclass.


% \setcounter{lemma}{4}

% \begin{acor}\label{a-lenzen-improved-planar}\hspace{-1.7mm}\textbf{.}
%
%   \vspace{-6mm}
%   \textit{\begin{enumerate}
%     \item If $G$ is bipartite, then $|\hat{D}\setminus D| < 2\gamma$.\smallskip
%     \item If $G$ is triangle-free, outerplanar, or has girth 5,
%       then $|\hat{D}\setminus D| < 3\gamma$.
%   \end{enumerate}}
% \end{acor}
\begin{adapted}{lenzen-improved-planar}{a-lenzen-improved-planar}

  \vspace{-6mm}
  \textit{\begin{enumerate}
    \item If $G$ is bipartite, then $|\hat{D}\setminus D| < 2\gamma$.\smallskip
    \item If $G$ is triangle-free, outerplanar, or has girth 5,
      then $|\hat{D}\setminus D| < 3\gamma$.
  \end{enumerate}}
\end{adapted}

\begin{proof}
This is immediate from \cref{lem:densities} and \cref{lenzen-improved}.
\end{proof}

The inclusion $D_1\subseteq \hat D$ continues to hold and the bound
on the sizes as stated in the next lemma is again a direct consequence of the corollary.

\smallskip
% \begin{alemma}\label{alem:D-hat}\hspace{-1.7mm}\textbf{.}
%     We have $D_1\subseteq \hat D$, and
%   \textit{\begin{enumerate}
%     \item if $G$ is bipartite,
%       then $|\hat{D}\setminus D| < 2\gamma$ and $|\hat{D}|< 3\gamma$.\smallskip
%     \item if $G$ is triangle-free, outerplanar, or has girth 5,
%       then $|\hat{D}\setminus D| < 3\gamma$ and $|\hat{D}|< 4\gamma$.
%   \end{enumerate}}
% \end{alemma}
\begin{adapted}{lem:D-hat}{alem:D-hat}
    We have $D_1\subseteq \hat D$, and
  \textit{\begin{enumerate}
    \item if $G$ is bipartite,
      then $|\hat{D}\setminus D| < 2\gamma$ and $|\hat{D}|< 3\gamma$.\smallskip
    \item if $G$ is triangle-free, outerplanar, or has girth 5,
      then $|\hat{D}\setminus D| < 3\gamma$ and $|\hat{D}|< 4\gamma$.
  \end{enumerate}}
\end{adapted}


In case of triangle-free planar graphs (in particular in the case of bipartite
planar graphs) we proceed with the second phase exactly as in the second phase of
the general algorithm (\cref{sec:D2}), however, the parameter 10 is replaced by
the parameter $7$. In  case of planar graphs of girth at least five or outerplanar
graphs, we simply set $D_2=\emptyset$.

\begin{tcolorbox}[colback=red!5!white,colframe=red!50!black]
  If $G$ is triangle-free:

  \begin{itemize}
    \item For $v\in V(G)$ let $B_v\coloneqq \{z\in V(G)\setminus
      \{v\}: |N_R(v)\cap N_R(z)|\geq 7\}$.\smallskip
    \item Let $W$ be the set of vertices $v\in V(G)$ such
      that $B_v \neq \emptyset$.\smallskip
    \item Let $D_2\coloneqq \bigcup\limits_{v\in W} (\{v\}\cup B_v)$.
  \end{itemize}

  If $G$ has girth at least $5$ or $G$ is outerplanar, let $D_2=\emptyset$.
\end{tcolorbox}

\cref{lem:WcapD1} is based only on the definition of $B_v$ and $W$ and
does not use particular properties of planar graphs, hence, it also holds
in the restricted case and we recall it for convenience.

% \recapLemWAndD

% \ref*{lem:WcapD1}
\setcounter{lemma}{7-1}
\begin{lemma}\label{alem:WcapD1}
  We have $W\cap D_1=\emptyset$ and for every $v\in V(G)$ we have
  $B_v\cap D_1=\emptyset$.
\end{lemma}

The next lemma uses the triangle-free property.

% \begin{alemma}\label{alem:dominating-dominators}\hspace{-1.7mm}\textbf{.}
%   If $G$ is triangle-free, then for all vertices $v\in W$ we have
%
%   \vspace{-5pt}
%   \begin{itemize}
%   \item $B_v \subseteq A_v$ (hence $|B_v|\le 3$), and \smallskip
%   \item if $v\not\in \hat{D}$, then $B_v\subseteq D$.
%   \end{itemize}
% \end{alemma}
\begin{adapted}{lem:dominating-dominators}{alem:dominating-dominators}
  If $G$ is triangle-free, then for all vertices $v\in W$ we have

  \vspace{-5pt}
  \begin{itemize}
  \item $B_v \subseteq A_v$ (hence $|B_v|\le 3$), and \smallskip
  \item if $v\not\in \hat{D}$, then $B_v\subseteq D$.
  \end{itemize}
\end{adapted}
\begin{proof}
  Assume $A_v=\{v_1,v_2, v_3\}$ and assume there is $z\in V(G)\setminus \{v,v_1,v_2, v_3\}$
  with $|N_R(v) \cap N_R(z)| \geq 7$.
  As the vertices $v_1, v_2, v_3$ dominate $N_R(v)$, and hence $N_R(v)\cap N_R(z)$,
  there must be some~$v_i$, $1\leq i\leq 3$, with
  \mbox{$|N_R(v) \cap N_R(z) \cap N[v_i]| \geq \lceil 7/3\rceil \geq 3$}.
  Then one of the following holds: either
  \mbox{$|N_R(v) \cap N_R(z) \cap N(v_i)| \geq 3$},  or
  \mbox{$|N_R(v) \cap N_R(z) \cap N(v_i)| =2$}.
  The first case shows that $K_{3,3}$ is a subgraph of~$G$
  contradicting the assumption that $G$ is planar.
  The second case implies that $v_i\in N_R(v)$. In this situation, by picking
  $w \in N_R(v) \cap N_R(z) \cap N(v_i)$, we get that $(v,v_i,w)$ is a triangle,
  hence we also reach a contradiction.

  If furthermore $v\not\in \hat{D}$, by definition of $\hat{D}$,
  we can find $w_1,w_2, w_3$ from $D$
  that dominate $N(v)$, and in particular $N_R(v)$.
  If $z\in V(G)\setminus \{v,w_1,w_2, w_3\}$
  with $|N_R(v) \cap N_R(z)| \geq 7$ we can argue as above to obtain
  a contradiction.
\end{proof}


For our analysis we again split $D_2$ into three sets $D_2^1, D_2^2$ and
$D_2^3$. The next lemmas hold also for the restricted cases. We repeat
them for convenience.

% \begin{alemma}\label{alem:size-D21}
%   If $G$ is triangle-free, then $|D_2^1\setminus D|\leq 3\gamma$.
% \end{alemma}
\begin{adapted}{lem:size-D21}{alem:size-D21}
  If $G$ is triangle-free, then $|D_2^1\setminus D|\leq 3\gamma$.
\end{adapted}

% \begin{alemma}\label{alem:size-D22}
%   If $G$ is triangle-free, then $D_2^2 \subseteq \hat D$ and therefore
%   $|D_2^2\setminus D|< 3\gamma$.
% \end{alemma}
\begin{adapted}{lem:size-D22}{alem:size-D22}
  If $G$ is triangle-free, then $D_2^2 \subseteq \hat D$ and therefore
  $|D_2^2\setminus D|< 3\gamma$.
\end{adapted}

% \begin{alemma}\label{alem:size-D23}
%   If $G$ is triangle-free, then $D_2^3\subseteq D_2^1$.
% \end{alemma}
\begin{adapted}{lem:size-D23}{alem:size-D23}
  If $G$ is triangle-free, then $D_2^3\subseteq D_2^1$.
\end{adapted}

Again, for a fine analysis, we
analyze the number of vertices of $D$ that have been selected
so far and let $\rho\in [0,1]$ be such that $|(D_1\cup D_2)\cap D| =\rho\gamma$.

% \begin{alemma}\label{alem:size-D12}
% \mbox{ }
% \vspace{-1mm}
% \begin{enumerate}
% \item If $G$ is bipartite, then $|D_1\cup D_2| < 2\gamma+4\e\gamma$.\smallskip
% \item If $G$ is triangle-free, then $|D_1\cup D_2| < 3\gamma+4\e\gamma$.\smallskip
% \item If $G$ has girth at least $5$ or is outerplanar, then $|D_1\cup D_2| < 3\gamma+\e\gamma$.
% \end{enumerate}
% \end{alemma}
\begin{adapted}{lem:size-D12}{alem:size-D12}
  \begin{enumerate}
    \item If $G$ is bipartite, then $|D_1\cup D_2| < 2\gamma+4\rho\gamma$.\smallskip
    \item If $G$ is triangle-free, then $|D_1\cup D_2| < 3\gamma+4\rho\gamma$.\smallskip
    \item If $G$ has girth at least $5$ or is outerplanar, then $|D_1\cup D_2| < 3\gamma+\rho\gamma$.
  \end{enumerate}
\end{adapted}
\begin{proof}
If $G$ is outerplanar or $G$ has girth at least $5$, then $D_2=\emptyset$.
By \ref{alem:D-hat} we have $D_1\subseteq \hat{D}$ and
$|\hat{D}\setminus D|<3\gamma$, hence $(D_1\cup D_2)\setminus D<3\gamma$.

If $G$ is triangle-free, by \ref{alem:size-D23} we have $D_2^3\subseteq D_2^1$, hence,
  $D_1\cup D_2=D_1\cup D_2^1\cup D_2^2$. By \ref{alem:D-hat} we have
  $D_1 \subseteq \hat D$ and by \ref{alem:size-D22} we also have
  $D_2^2 \subseteq \hat D$, hence $D_1\cup D_2^2\subseteq \hat D$.
  Again by \ref{alem:D-hat}, if $G$ is bipartite, then
  $|\hat D \setminus D|<2\gamma$, therefore $|(D_1 \cup D_2^2 )\setminus D| < 2\gamma$, and if $G$ is triangle-free,
  then $|\hat D \setminus D|<3\gamma$,
  therefore $|(D_1 \cup D_2^2 )\setminus D| < 3\gamma$.
  We have $W\cap D\subseteq D_2^1\cap D$, hence with
  \ref{alem:dominating-dominators} we conclude that
  \[
    \big\vert D_2^1 \setminus D \big\vert \leq
    \Big\vert\bigcup\limits_{v\in D \cap D_2^1}B_v\Big\vert \leq
    \sum\limits_{v\in D \cap D_2^1} |B_v| \leq 3\rho\gamma,
  \]
  hence $(D_1\cup D_2)\setminus D<2\gamma+3\rho\gamma$ if
  $G$ is bipartite and $(D_1\cup D_2)\setminus D<3\gamma+3\rho\gamma$
  if $G$ is triangle-free.

  Finally,
  $D_1\cup D_2=(D_1\cup D_2)\setminus D\cup (D_1\cup D_2)\cap D$ and
  with the definition of $\rho$ we conclude

  \vspace{-2mm}
  \begin{enumerate}
  \item $|D_1\cup D_2|<2\gamma + 4\rho\gamma$ if $G$ is bipartite, \smallskip
  \item $|D_1\cup D_2|<3\gamma + 4\rho\gamma$ if $G$ is triangle-free, \smallskip
  \item $|D_1\cup D_2|<3\gamma + \rho\gamma$ if $G$ has girth at least
  $5$ or is outerplanar.
  \end{enumerate}

\end{proof}

Again, we now update the residual degrees, that is, we update
$R$ as $V(G)\setminus N[D_1\cup D_2]$ and for every vertex the
number $\delta_R(v)=N(v)\cap R$ accordingly and proceed with
the third phase.

% \begin{alemma}\label{alem:res-degree}\hspace{-1.7mm}\textbf{.}
%
% \vspace{-6mm}
% \textit{\begin{enumerate}
% \item If $G$ is triangle-free, then for all $v\in V(G)$ we have $\delta_R(v)\leq 18$.\smallskip
% \item If $G$ has girth at least $5$, then for all $v\in V(G)$ we have $\delta_R(v)\leq 3$.\smallskip
% \item If $G$ is outerplanar, then for all $v\in V(G)$ we have $\delta_R(v)\leq 9$.
% \end{enumerate}}
% \end{alemma}
\begin{adapted}{lem:res-degree}{alem:res-degree}
  \textit{\begin{enumerate}
    \item If $G$ is triangle-free, then for all $v\in V(G)$ we have $\delta_R(v)\leq 18$.\smallskip
    \item If $G$ has girth at least $5$, then for all $v\in V(G)$ we have $\delta_R(v)\leq 3$.\smallskip
    \item If $G$ is outerplanar, then for all $v\in V(G)$ we have $\delta_R(v)\leq 9$.
  \end{enumerate}}
\end{adapted}
\begin{proof}
  Every vertex of $D_1\cup D_2$ has residual degree $0$, hence, we
  need to consider only vertices that are not in $D_1$ or $D_2$.

  First assume that the graph is triangle-free and
  assume that there is a vertex $v$ of residual degree at least $19$.
  As $v$ is not in $D_1$, its $19$ non-dominated
  neighbors are dominated by a
  set $A_v$ of at most~3 vertices. Hence, there is vertex $z$ (not in $D_1$
  nor $D_2$) dominating at least $\lceil 19/3\rceil = 7$ of them.
  Here, $z$ cannot be one of these 7 vertices, otherwise it would be connected
  to $v$ and there would be a triangle in the graph.
  Therefore we
  have $|N_R(v)\cap N_R(z)|\geq 7$, contradicting that $v$ is not in~$D_2$.

  Now assume that $G$ has girth at least $5$ and
  assume that there is a vertex $v$ of residual degree at least~$4$.
  As $v$ is not in $D_1$, its $4$ non-dominated
  neighbors are dominated by a
  set $A_v$ of at most~3 vertices. Hence, there is vertex $z$ (not in $D_1$
  nor $D_2$) dominating at least $\lceil 4/3\rceil = 2$ of them.
  Here, $z$ cannot be one of these 2 vertices, otherwise it would be connected
  to $v$ and there would be a triangle in the graph. However, $z$ can
  also not be any other vertex, as otherwise we find a cycle of length $4$,
  contradicting that~$G$ has girth at least $5$.

  Finally, assume that $G$ is outerplanar and
  assume that there is a vertex $v$ of residual degree at least $10$.
  As $v$ is not in $D_1$, its $10$ non-dominated
  neighbors are dominated by a
  set $A_v$ of at most~3 vertices. Hence, there is vertex $z$ (not in $D_1$
  nor $D_2$) dominating at least $\lceil 10/3\rceil = 4$ of them.
  Therefore $|N(v)\cap N(z)|\geq 3$, and we find a $K_{2,3}$ as a
  subgraph, contradicting that $G$ is outerplanar.
\end{proof}

\subsection{LP-based approximation}

When proceeding with the LP-based approximation as in \cref{sec:LP}
we conclude with \cref{cor:LP-approx} and \cref{lem:size-D31} to obtain the
following statement:
\begin{adapted}{lem:D3-LP}{alem:D3-LP}
  When $G$ is triangle free of bipartite, $|D_3|\le (5+\e)(1-\rho)\gamma$.
\end{adapted}
We can then conclude each individual case.



% \todoA{check the number of the additive gain}
% \alex{use adapted version of \cref{lem:D3-LP} to gain ($1-\e$) additive factor}

\begin{theorem}\label{thm:tri}
  There exists a distributed LOCAL algorithm that, for every $\e>0$ and every
  triangle free planar graph $G$, computes in a constant number of rounds a
  dominating set of size at most $(8+\e)\gamma(G)$.
\end{theorem}
\begin{proof}
  By \ref{alem:size-D12} we have $|D_1\cup D_2|<3\gamma+4\rho\gamma$ and
  by \ref{alem:D3-LP} we have that~$|D_3|\le (5+\e)(1-\rho)\gamma$.
  %
  Hence $|D_1 \cup D_2 \cup D_3| < (8+\e -\rho-\e\rho)\gamma$.
  %
  As $\rho\in[0,1]$, this is maximized when $\rho=0$. Hence
  \mbox{$|D_1 \cup D_2 \cup D_3|< (8+\e) \gamma$}.
\end{proof}

\begin{theorem}\label{thm:bip}
  There exists a distributed LOCAL algorithm that, for every $\e>0$ and
  every bipartite planar graph
  $G$, computes in a constant number of rounds a
  dominating set of size at most $(7+\e)\gamma(G)$.
\end{theorem}
\begin{proof}
  By \ref{alem:size-D12} we have $|D_1\cup D_2|<2\gamma+4\rho\gamma$ and
  by \ref{alem:D3-LP} we have that~$|D_3|\le (5+\e)(1-\rho)\gamma$.
  %
  Hence $|D_1 \cup D_2 \cup D_3| < (7+\e-\rho-\e\rho)\gamma$.
  %
  As $\rho\in[0,1]$, this is maximized when $\rho=0$. Hence
  \mbox{$|D_1 \cup D_2 \cup D_3|< (7+\e) \gamma$}.
\end{proof}

As we will see, the greedy approach can improve the following theorem.

\begin{theorem}\label{thm:girth}
  There exists a distributed LOCAL algorithm that, for every $\e>0$ and
  every planar graph
  $G$ of girth at least~$5$, computes in a constant number of rounds a
  dominating set of size at most $(8+\e)\gamma(G)$.
\end{theorem}
\begin{proof}
  By \ref{alem:size-D12} we have $|D_1\cup D_2|<3\gamma+\rho\gamma$ and
  by \ref{alem:D3-LP} we have that~$|D_3|\le (5+\e)(1-\rho)\gamma$.
  %
  Hence $|D_1 \cup D_2 \cup D_3| < (8+\e-4\rho-\e\rho)\gamma$.
  %
  As $\rho\in[0,1]$, this is maximized when $\rho=0$. Hence
  \mbox{$|D_1 \cup D_2 \cup D_3|< (8+\e) \gamma$}.
\end{proof}

Also the following theorem is not optimal by the results of Bonamy et al.~\cite{bonamy2021tight}.

\begin{theorem}\label{thm:outer}
  There exists a distributed LOCAL algorithm that, for every $\e>0$ and
  every outerplanar graph
  $G$, computes in a constant number of rounds a
  dominating set of size at most $(8+\e)\gamma(G)$.
\end{theorem}
\begin{proof}
  By \ref{alem:size-D12} we have $|D_1\cup D_2|<3\gamma+\rho\gamma$ and
  by \ref{alem:D3-LP} we have that~$|D_3|\le (5+\e)(1-\rho)\gamma$.
  %
  Hence $|D_1 \cup D_2 \cup D_3| < (8+\e-4\rho-\e\rho)\gamma$.
  %
  As $\rho\in[0,1]$, this is maximized when $\rho=0$. Hence
  \mbox{$|D_1 \cup D_2 \cup D_3|< (8+\e) \gamma$}.
\end{proof}

\subsection{Greedy approximation}

When proceeding by computing a dominating set $D_3$ with the
greedy algorithm presented in~\cref{sec:greedy}
in the respective number of rounds we get the following
results.

\setcounter{lemma}{12} % this is to match the non-adapted lemmas
% This is now irrelevant with the adapted environment

% \begin{alemma}\label{alem:total-H}
%   If $G$ is triangle-free or outerplanar,
%   for every $1\le i$, $\sum\limits_{j\le i}|\Delta_j| \le |R_i|$.
% \end{alemma}
\begin{adapted}{lem:total-h}{alem:total-H}
  If $G$ is triangle-free or outerplanar,
  for every $1\le i$, $\sum\limits_{j\le i}|\Delta_j| \le |R_i|$.
\end{adapted}

% \begin{alemma}\label{alem:tri-h1}
%   If $G$ is triangle-free or outerplanar,
%   for every $1\le i$, $|R_i| \le (i+1)(1-\e)\gamma$.
% \end{alemma}
\begin{adapted}{lem:h1}{alem:tri-h1}
  If $G$ is triangle-free or outerplanar,
  for every $1\le i$, $|R_i| \le (i+1)(1-\rho)\gamma$.
\end{adapted}

% \begin{alemma}\label{alem:tri-delta}
%   If $G$ is triangle-free or outerplanar,
%   for every $5\le i$, $|\Delta_i| \le \frac{2|R_i|}{i-4}$.
% \end{alemma}
\begin{adapted}{lem:delta}{alem:tri-delta}
  If $G$ is triangle-free or outerplanar,
  for every $5\le i$, $|\Delta_i| \le \frac{2|R_i|}{i-4}$.
\end{adapted}

% \begin{alemma}\label{alem:tri-h2}
%   If $G$ is triangle-free or outerplanar,
%   for every $1\le i$, $|R_i| \le |R_{i+1}| - \frac{(i-3)|\Delta_{i+1}|}{2}$.
% \end{alemma}
\begin{adapted}{lem:h2}{alem:tri-h2}
  If $G$ is triangle-free or outerplanar,
  for every $1\le i$, $|R_i| \le |R_{i+1}| - \frac{(i-3)|\Delta_{i+1}|}{2}$.
\end{adapted}

The proofs of \ref{alem:total-H} to \ref{alem:tri-h2} are copies of the ones
for \cref{lem:total-h,lem:h1,lem:delta,lem:h2}, with the execption that the edge
density of $3$ for planar graphs if now replaced by $2$ for triangle-free and
outerplanar.
%
Similarly to \cref{lem:size-D3} we formulate (and present in
\cref{sec:linear-prog}) a linear program to maximize $|D_3|$ under these
constraints, yielding the following lemma.

% \begin{alemma}\label{alem:size-D3}\hspace{-1.7mm}\textbf{.}
%
% \vspace{-4mm}
% \textit{\begin{enumerate}
% \item If $G$ is triangle-free, then  $|D_3|\le 10.5(1-\e)\gamma$.\smallskip
% \item If $G$ has girth at least $5$, then $|D_3|\le 4(1-\e)\gamma$.\smallskip
% \item If $G$ is outerplanar, then $|D_3|\le 8.6(1-\e)\gamma$.
% \end{enumerate}}
% \end{alemma}
\begin{adapted}{lem:size-D3}{alem:size-D3}
  \begin{enumerate}
    \item If $G$ is triangle-free, then  $|D_3|\le 10.5(1-\rho)\gamma$.\smallskip
    \item If $G$ has girth at least $5$, then $|D_3|\le 4(1-\rho)\gamma$.\smallskip
    \item If $G$ is outerplanar, then $|D_3|\le 8.6(1-\rho)\gamma$.
  \end{enumerate}
\end{adapted}

The following theorems are derived from \cref{thm:planar-greedy} just as before.

\begin{theorem}\label{thm:tri-2}
  There exists a distributed LOCAL algorithm that, for every triangle free planar
  graph $G$, computes in a constant number of rounds a dominating set
  of size at most $14\gamma(G)$.
\end{theorem}
%\begin{proof}
%By
%\cref{alem:size-D12} we have $|D_1\cup D_2|<3\gamma+4\e\gamma$.  Then,
%by \cref{alem:size-D3} we have $|D_3|\le 10.5(1-\e)\gamma$.
%%
%Therefore $|D_1 \cup D_2 \cup D_3| < 13.5\gamma -6.5\e\gamma$.
%%
%As $\e\in[0,1]$, this is maximized when $\e=0$. Hence
%\mbox{$|D_1 \cup D_2 \cup D_3|< 13.5 \gamma$}.
%\end{proof}

\begin{theorem}\label{thm:bip-2}
  There exists a distributed LOCAL algorithm that, for every bipartite planar graph
  $G$, computes in a constant number of rounds a
  dominating set of size at most $13\gamma(G)$.
\end{theorem}
%\begin{proof}
%By
%\cref{alem:size-D12} we have $|D_1\cup D_2|<2\gamma+4\e\gamma$.  Then,
%by \cref{alem:size-D3} we have $|D_3|\le 10.5(1-\e)\gamma$.
%%
%Therefore $|D_1 \cup D_2 \cup D_3| < 12.5\gamma -6.5\e\gamma$.
%%
%As $\e\in[0,1]$, this is maximized when $\e=0$. Hence
%\mbox{$|D_1 \cup D_2 \cup D_3|< 12.5 \gamma$}.
%\end{proof}

In the case of high girth graphs the greedy algorithm outperforms the
LP-based algorithm.

\begin{theorem}\label{thm:girth-2}
  There exists a distributed LOCAL algorithm that, for every planar graph
  $G$ of girth at least~$5$, computes in a constant number of rounds a
  dominating set of size at most $7\gamma(G)$.
\end{theorem}
%\begin{proof}
%By
%\cref{alem:size-D12} we have $|D_1\cup D_2|<3\gamma+\e\gamma$.  Then,
%by \cref{alem:size-D3} we have $|D_3|\le 4(1-\e)\gamma$.
%%
%Therefore $|D_1 \cup D_2 \cup D_3| < 7\gamma -3\e\gamma$.
%%
%As $\e\in[0,1]$, this is maximized when $\e=0$. Hence
%\mbox{$|D_1 \cup D_2 \cup D_3|< 7 \gamma$}.
%\end{proof}

\begin{theorem}\label{thm:outer-2}
  There exists a distributed LOCAL algorithm that, for every outerplanar graph
  $G$, computes in a constant number of rounds a
  dominating set of size at most $12\gamma(G)$.
\end{theorem}
%\begin{proof}
%By
%\cref{alem:size-D12} we have $|D_1\cup D_2|<3\gamma+\e\gamma$.  Then,
%by \cref{alem:size-D3} we have $|D_3|\le 8.6(1-\e)\gamma$.
%%
%Therefore $|D_1 \cup D_2 \cup D_3| < 11.6\gamma -7.6\e\gamma$.
%%
%As $\e\in[0,1]$, this is maximized when $\e=0$. Hence
%\mbox{$|D_1 \cup D_2 \cup D_3|< 11.6 \gamma$}.
%\end{proof}


% \subsection{Triangle-free planar graphs}\label{sec:triangle-free}
% We first consider classes of planar graphs that have no triangles (equivalently, that have girth at least $4$.) The first phase of the algorithm parallels
% the first phase of the general algorithm (\cref{sec:step1}), however
% the numbers $\nn_0$ and $\nn_1$ are now smaller. We will use
% the following well-known lemma.
% \idea{We should have:
%   \begin{itemize}
%     \item 3-col hence so smaller $D_1$.
%     \item $|A_v|\le 3$.
%     \item $\ge 7$ in the def of $B_v$ (instead of 11).
%     \item red-degree 18 and not 50 for the beginning of step 3.
%   \end{itemize}
% }
% \begin{lemma}\label{lem:tri-euler}
% Every $n$-vertex triangle-free planar graph $G$ $(n\geq 3)$ has at
% most $2n-4$ edges. Therefore $\nn_0(G)<2$ and $\nn_1(G)<3$
% \end{lemma}
% %\begin{proof}
% %  We use Euler's formula which states that on a connected planar graph $v-e+f=2$,
% %  where $f$ is the number of phases of $G$. We then use the fact that every edge
% %  lies between 2 phases and that every phase is surrounded by at least 4 edges.
% %  Hence $2e\ge 4f$, and therefore $e\le 2\cdot v-4$.
% %  Unfortunately, a $1$-shallow minor of a triangle-free graph might contain a
% %  triangle. The only bound that we have for $\nn_1(G)$ is therefore the same
% %  as for planar graphs in general.
% %\end{proof}
%
% Just as in the general case, the first phase of the algorithm is
% to compute the following set $D_1$, which can be done in only
% 2 rounds of communication.
% %The following lemma states that there exist only few such vertices.
% \begin{tcolorbox}[colback=red!5!white,colframe=red!50!black]
%   \vspace{-4mm}
%   \begin{align*}
%     D_1\coloneqq \{v\in V(G) : \text{ for all sets } A\subseteq V(G)\setminus \{v\}
%     & \text{ with $N(v)\subseteq N[A]$} \text{ we have $|A|> 5\}$.}
%   \end{align*}
% \end{tcolorbox}
%
% The bounds given by~\cref{lem:tri-euler} when used in \cref{lem:lenzen-generic} yield the following result.
% \begin{lemma}[Corollary of \cref{lem:lenzen-generic}]\label{lem:tri-lenzen}
%   We have $|D_1\setminus D|< 3\gamma$.
% \end{lemma}
%
% \begin{tcolorbox}
%   For every $v\in V(G)\setminus D_1$, we fix $A_v\subseteq V(G)\setminus \{v\}$
%   such that: \center $N_R(v)\subseteq N[A_v]$ and $|A_v|\leq 5$.
% \end{tcolorbox}
% The fact that here $|A_v|\leq 5$ (instead of $6$ in the general planar case)
% helps to make the second step more efficient.
%
%
% \begin{tcolorbox}
% \vspace{-4mm}
%     \begin{align*}
%   \hat D\coloneqq \{v\in V(G) : \text{ for all sets } A\subseteq D\setminus \{v\}
%   & \text{ with $N(v)\subseteq N[A]$} \text{ we have $|A|> 5\}$.}
%   \end{align*}
% \end{tcolorbox}
%
% \cref{lem:lenzen-generic} also provides better bounds for $\hat D$ than in general.
% \begin{lemma}\label{lem:tri-D-hat}
%   We have that $D_1\subseteq \hat D$, $|\hat{D}\setminus D|< 3\gamma$, and
%   $|\hat{D}|< 4\gamma$.
% \end{lemma}
%
%
% We proceed with the second phase exactly as in the second phase of
% the general algorithm (\cref{sec:D2}), however, the parameter 19 is replaced by 11.
%
%
% \begin{tcolorbox}[colback=red!5!white,colframe=red!50!black]
%   \begin{itemize}
%     \item For $v\in V(G)$ let $B_v\coloneqq \{z\in V(G)\setminus
%       \{v\}: |N_R(v)\cap N_R(z)|\geq 11\}$.\smallskip
%     \item Let $W$ be the set of vertices $v\in V(G)$ such
%       that $B_v \neq \emptyset$.\smallskip
%     \item Let $D_2\coloneqq \bigcup\limits_{v\in W} (\{v\}\cup B_v)$.
%   \end{itemize}
% \end{tcolorbox}
%
% For our analysis we again split $D_2$ into three sets.
%
% \begin{tcolorbox}
%   \begin{enumerate}
%     \item $D_2^1\coloneqq \bigcup_{v\in W\cap D}
%     (\{v\}\cup B_v)$, \smallskip
%     \item $D_2^2\coloneqq \bigcup_{v\in W\cap (\hat{D}\setminus D)}
%     (\{v\}\cup B_v)$, and \smallskip
%     \item $D_2^3\coloneqq \bigcup_{v\in W\setminus (D\cup \hat{D})}
%     (\{v\}\cup B_v)$.
%   \end{enumerate}
% \end{tcolorbox}
%
% \begin{lemma}\label{lem:tri-dominating-dominators}
%   For all vertices $v\in W$, we have:
%   \begin{itemize}
%     \item $B_v \subseteq A_v$ (hence $|B_v|\le 5$), and
%     \item if $v\not\in \hat{D}$, then $B_v\subseteq D$.
%   \end{itemize}
% \end{lemma}
%
% This is the only lemma that uses the triangle-free property, and is therefore the only one that we prove again.
% \begin{proof}
%   Assume $A_v=\{v_1,\ldots, v_5\}$ and assume there is $z\in V(G)\setminus \{v,v_1,\ldots, v_5\}$
%   with $|N_R(v) \cap N_R(z)| \geq 11$.
%   As the vertices $v_1, \ldots, v_5$ dominate $N_R(v)$, and hence $N_R(v)\cap N_R(z)$,
%   there must be some~$v_i$, $1\leq i\leq 5$, with
%   \mbox{$|N_R(v) \cap N_R(z) \cap N[v_i]| \geq \lceil 11/5\rceil \geq 3$}.
%   Then either
%   \mbox{$|N_R(v) \cap N_R(z) \cap N(v_i)| \geq 3$},  or
%   \mbox{$|N_R(v) \cap N_R(z) \cap N(v_i)| =2$}.
%   The first case shows that $K_{3,3}$ is a subgraph of~$G$
%   contradicting the assumption that $G$ is planar.
%   The second case implies that $v_i\in N_R(v)$, hence by picking
%   $w \in N_R(v) \cap N_R(z) \cap N(v_i)$, we get that $(v,v_i,w)$ is a triangle,
%   hence we also reach a contradiction.
%
%   If furthermore $v\not\in \hat{D}$, by definition of $\hat{D}$,
%   we can find $w_1,\ldots, w_5$ from $D$
%   that dominate $N(v)$, and in particular $N_R(v)$.
%   If $z\in V(G)\setminus \{v,w_1,\ldots, w_5\}$
%   with $|N_R(v) \cap N_R(z)| \geq 11$ we can argue as above to obtain
%   a contradiction.
% \end{proof}
%
% The following lemmas are straightforward adaptations of
% \cref{lem:size-D21,lem:size-D22,lem:size-D23}.
% \begin{lemma}\label{lem:tri-size-D21}
%   $|D_2^1\setminus D|\leq 5\gamma$.
% \end{lemma}
% \begin{lemma}\label{lem:tri-size-D22}
%   $D_2^2 \subseteq \hat D$ and therefore $|D_2^2\setminus D|< 3\gamma$.
% \end{lemma}
% \begin{lemma}\label{lem:tri-size-D23}
%   $D_2^3\subseteq D_2^1$.
% \end{lemma}
%
% W then apply the same reasoning by counting how many elements of $D$ have been
% picked so far.
% \begin{tcolorbox}
%   Let $\e\in [0,1]$ be such that $|(D_1 \cup D_2)\cap D| =\e\gamma$.
% \end{tcolorbox}
%
% The following is a straightforward adaptation of \cref{lem:size-D12}.
% \begin{lemma}\label{lem:tri-size-D12}
%   We have that $|D_1\cup D_2| < 3\gamma+6\e\gamma$
% \end{lemma}
%
% We now update the residual degrees, that is, we update
% $R$ as $V(G)\setminus N[D_1\cup D_2]$ and for every vertex the
% number $\delta_R(v)=N(v)\cap R$ accordingly.
% For the final phase we observe that
% every vertex has residual degree at most 50.
%
% \begin{lemma}\label{lem:tri-res-degree}
% For all $v\in V(G)$ we have $\delta_R(v)\leq 50$.
% \end{lemma}
% \begin{proof}
%   First, every vertex of $D_1\cup D_2$ has residual degree $0$.
%   Assume that there is a vertex $v$ of residual degree at least $51$.
%   As $v$ is not in $D_1$ its $51$ non-dominated
%   neighbors are dominated by a
%   set $A_v$ of at most~5 vertices. Hence, there is be vertex $z$ (not in $D_1$
%   nor $D_2$) dominating at least $\lceil 51/5\rceil = 11$ of them.
%   Here, $z$ cannot be one of these 11 vertices, otherwise it would be connected
%   to $v$ and there will be a triangle in the graph.
%   Therefore we
%   have $|N_R(v)\cap N_R(z)|\geq 11$, contradicting that $v$ is not in~$D_2$.
% \end{proof}
%
% We proceed to compute a dominating set of the remaining vertices
% as before, but this time in 50 rounds.
%
% \begin{tcolorbox}[colback=red!5!white,colframe=red!50!black]
%   For $50\geq i\geq 0$,  for every $v\in R$ in parallel:\\[2mm]
%   if there is some $u$ with $\dd_R(u)=i$ and ($\{u,v\}\in E(G)$ or $u=v$), then\\
%   \mbox{ } $\dom_i(v)\coloneqq \{u\}$ (pick one such $u$ arbitrarily),\\
%   \mbox{ } $\dom_i(v)\coloneqq \emptyset$ otherwise.
%   \begin{itemize}
%     \item $R_i \coloneqq R$ \hfill \textit{\small What currently remains to be dominated}
%     \item $\Delta_i \coloneqq \bigcup\limits_{v\in R} \dom_i(v)$ \hfill \textit{\small What we pick in this step}
%     \item $R \coloneqq R \setminus N[\Delta_{i}]$ \hfill \textit{\small Update red vertices}
%   \end{itemize}
%   Finally, $D_3:=  \bigcup\limits_{1\le i\le 50} \Delta_i$.
% \end{tcolorbox}
%
% The following analysis follows exactly the same lines as the analysis in
% \cref{sec:greedy}, which we carried out with \cref{lem:correctness} to \cref{lem:size-D3}.
%
% \begin{lemma}
% When the algorithm has finished the iteration with parameter $i\geq 1$, then all vertices have residual degree at most $i-1$. In particular,
% the algorithm computes a dominating set of $G$.
% \end{lemma}
%
% \begin{lemma}\label{lem:tri-total-H}
%   For every $i\le 50$, $\sum\limits_{j\le i}|\Delta_j| \le |R_i|$.
% \end{lemma}
%
% \begin{lemma}\label{lem:tri-h1}
%   For every $i\le 50$, $|R_i| \le (i+1)(1-\e)\gamma$.
% \end{lemma}
%
% \begin{lemma}\label{lem:tri-delta}
%   For every $5\le i\le 50$, $|\Delta_i| \le \frac{2|R_i|}{i-4}$.
% \end{lemma}
%
% \begin{lemma}\label{lem:tri-h2}
%   For every $i\le 50$, $|R_i| \le |R_{i+1}| - \frac{(i-3)|\Delta_{i+1}|}{2}$.
% \end{lemma}
%
% % Intuitively, these lemmas follows simple reasonings. \cref{lem:h1} holds because
% % the elements of $D$ that have not been selected yet form a dominating set for
% % every $R_i$. Additionally, these dominators have bounded residual degree.
% % \cref{lem:delta} holds thank to an argument similar to the one of
% % \cref{lem:lenzen}: the edge density of the graph induced by $R_i$ and $\Delta_i$
% % is at most 3, and every vertices of $\Delta_i$ has high degree.
% % And finally, \cref{lem:h2} is prooved with a similar argument:
% % once $\Delta_i$ is fixed, its neighborhood cannot
% % be too small (this would create a dense subgraph). Therefore we can give a lower
% % bound to the number of elements that are newly dominated.
%
% \begin{lemma}\label{lem:tri-size-D3}
%   $|D_3|\le 12.8(1-\e)\gamma$.
% \end{lemma}
%
% Again, the proof for the last lemma is done by formulating the
% constraints as a linear program, which can be found in the appendix.
%
% \smallskip
% We are ready to summarize. The definition of $D_3$ implies that
% $D_1\cup D_2\cup D_3$ is a dominating set of $G$. We conclude the
% analysis of the size of the computed set.
% First, \cref{lem:tri-size-D12} states that $|D_1\cup D_2|<3\gamma+6\e\gamma$.
% Then, by \cref{lem:tri-size-D3}, we have that $|D_3|\le 12.8(1-\e)\gamma$.
% %
% Therefore $|D_1 \cup D_2 \cup D_3|
% % \le 3\gamma +8\e\gamma + 20.3(1-\e)\gamma
% \le 15.8\gamma -6.8\e\gamma$.
% As $\e\in[0,1]$, this is maximized when $\e=0$. Hence
% \mbox{$|D_1 \cup D_2 \cup D_3|< 15.8 \gamma$}.
%
% \begin{theorem}\label{thm:tri}
% There exists a distributed LOCAL algorithm that, for every triangle free planar
% graph $G$, computes in a constant number of rounds a dominating set
% of size at most $16\gamma(G)$.
% \end{theorem}
%
%
%
% %%%%%%%%%%%%%%%%%%%%%%%%%%%%%%%%%%%%%%%%%%%%%%%%%%%%%%%%%%%%%%%%%%%%%%%%%%%%%%%%
% %%%%%%%%%%%%%%%%%%%%%%%%%%%%%%%%%%%%%%%%%%%%%%%%%%%%%%%%%%%%%%%%%%%%%%%%%%%%%%%%
% %%%%%%%%%%%%%%%%%%%%%%%%%%%%%%%%%%%%%%%%%%%%%%%%%%%%%%%%%%%%%%%%%%%%%%%%%%%%%%%%
%
% \subsection{Planar graphs of girth 5}\label{sec:girth}
% We now consider the case of planar graphs of girth at least $5$.
% The first phase of the algorithm again parallels
% the first phase of the general algorithm (\cref{sec:step1}). We will use
% the following lemma.
%
% \idea{We should have:
%   \begin{itemize}
%     \item 3-col hence so smaller $D_1$.
%     \item $|A_v|\le 3$.
%     \item $B_v$ still empty.
%     \item red-degree 3 instead of 5.
%     \item total 7 approx.
%   \end{itemize}
% }
%
% \begin{lemma}\label{lem:girth-euler}
% Every $n$-vertex planar graph $G$ ($n\geq 3$) of girth at least $5$ has
% at most $\frac{5}{3}n-\frac{10}{3}$ edges. Therefore $\nn_0(G)<\frac{5}{3}$ and $\nn_1(G)<3$.
% \end{lemma}
% \begin{proof}
%   We use Euler's formula, which states that a connected planar graph
%   satisfies $n-m+f=2$, where~$n$ is the number of vertices, $m$
%   is the number of edges and $f$ is the number of faces of $G$. We
%   then use the fact that every edge lies on $2$ faces and that the
%   boundary of every face consists of at least $5$ edges.
%   Hence $2m\ge 5f$.
%   We then have $5n-5m+5f=10$, hence $5n-5m+2e\ge 10$ and therefore
%   $m\le \frac{5}{3}\cdot n-\frac{10}{3}$. This remains true if the
%   graph is not connected as we may apply the same reasoning to every
%   connected component.
% \end{proof}
%
% Just as in the general case, the first phase of the algorithm is
% to compute the following set $D_1$.
% %The following lemma states that there exist only few such vertices.
% \begin{tcolorbox}[colback=red!5!white,colframe=red!50!black]
%   \vspace{-4mm}
%   \begin{align*}
%     D_1\coloneqq \{v\in V(G) : \text{ for all sets } A\subseteq V(G)\setminus \{v\}
%     & \text{ with $N(v)\subseteq N[A]$} \text{ we have $|A|> 5\}$.}
%   \end{align*}
% \end{tcolorbox}
%
% \begin{lemma}[Adaptation of \cref{lem:lenzen}]\label{lem:girth-lenzen}
%   We have $|D_1\setminus D|< (2+1/4)\gamma$.
% \end{lemma}
%
%
% \alex{add "same" proof. Careful, for the 1-minor, the edge density can still be 3.}
%
% Note that the residual degree is already bounded.
% \begin{lemma}\label{lem:girth-degree}
%   Every vertex of $G$ of degree at least $6$ belongs to $D_1$.
% \end{lemma}
% \begin{proof}
%   Let $v$ be a vertices of degree at least $6$ and $v_1,\ldots,v_5$
%   vertices (other than $v$) such that $N(v)\subseteq \bigcup_{i\le 5}N[v_i]$.
%   Hence there is a $v_i$ with $|N(v)\cap N[v_i]|\ge 2$.
%   Observe that if $v_i$ is one of these two vertices, we have a triangle, and
%   that otherwise we have a circle of length 4. In both case this contradicts the
%   fact that the graph has girth at least 5.
% \end{proof}
%
% \subsubsection{Step 2}
% Observe now that the second sept of the algorithm is trivial. It is impossible
% to find different vertices $v,z$ such that $|N(v)\cap N(z)|\ge 2$.
%
% \begin{tcolorbox}[colback=red!5!white,colframe=red!50!black]
%     $D_2\coloneqq \emptyset$.
% \end{tcolorbox}
%
% We then define $\e$ like we did in \cref{sec:D2}.
% \begin{tcolorbox}
%   Let $\e\in [0,1]$ be such that $|D_1\cap D| =\e\gamma$.
% \end{tcolorbox}
%
% \subsubsection{Step 3}
% And finally, we pick all vertices that where not dominated yet.
%
% \begin{tcolorbox}[colback=red!5!white,colframe=red!50!black]
%   Let $D_3\coloneqq \{v\in V(G) : v\not\in N[D_1]\}$.
% \end{tcolorbox}
%
% \begin{lemma}\label{lem:degree-girth}
%   We have $|D_3|\le 6(1-\e)\gamma$.
% \end{lemma}
%
% \begin{proof}
%   First note that $D\setminus D_1$ dominates all vertices of $G\setminus N[D_1]$.
%   Then every vertex of $D\setminus D_1$ has residual degree at most $5$.
%   Finally, the definition of $\e$ implies that $|D\setminus D_1|= (1-\e)\gamma$.
%   All together, we get that $|D_3|\le 6(1-\e)\gamma$
% \end{proof}
%
% We conclude with the fact that $D_1\cup D_3$ is a dominating set of $G$, and
% that $|D_1\cup D_3|< (2+1/4)\gamma + \e\gamma + 6(1-\e)\gamma$.
% %
% Hence $|D_1\cup D_3|< (8+1/4)\gamma - 5\e\gamma \le  (8+1/4)\gamma$.
%
% \begin{theorem}\label{thm:girth}
%   There exists a distributed LOCAL algorithm that, for every planar graph
%   $G$ of girth at least $5$, computes in a constant number of rounds a
%   dominating set of size at most $(8+1/4)\gamma(G)$.
% \end{theorem}
%
% \alex{I don't think that CPLEX helps here, but we can look.}
% \alex{Cplex also leads to 6 for $D_3$, we might still put it in the appendix}
%
%
% %%%%%%%%%%%%%%%%%%%%%%%%%%%%%%%%%%%%%%%%%%%%%%%%%%%%%%%%%%%%%%%%%%%%%%%%%%%%%%%%
% %%%%%%%%%%%%%%%%%%%%%%%%%%%%%%%%%%%%%%%%%%%%%%%%%%%%%%%%%%%%%%%%%%%%%%%%%%%%%%%%
% %%%%%%%%%%%%%%%%%%%%%%%%%%%%%%%%%%%%%%%%%%%%%%%%%%%%%%%%%%%%%%%%%%%%%%%%%%%%%%%%
%
% \subsection{Outerplanar graphs}\label{sec:outer}
%
% Outerplanar graphs are exactly the graphs excluding $K_4$ and $K_{2,3}$.
% Therefore a minor of an outerplanar graph is again outerplanar. Additionally,
% outerplanar graphs are 2-degenerate and hence for every outerplanar graphs $G$,
% we have $|E_G|\le 2V_G$. Although an optimal bound of $5$ is already known for
% outerplanar graphs (see \cite[Theorem 1]{bonamy2021tight}), we investigate
% how our algorithm performs there.
%
% \idea{We should have:
%   \begin{itemize}
%     \item 3-col hence so same size for $D_1$, but better property.
%     \item $|A_v|\le 3$.
%     \item $B_v$ still empty.
%     \item red-degree 12 instead of 17.
%     \item Same equation unless new improvement.
%   \end{itemize}
% }
%
% \begin{tcolorbox}[colback=red!5!white,colframe=red!50!black]
%   \vspace{-4mm}
%   \begin{align*}
%     D_1\coloneqq \{v\in V(G) : \text{ for all sets } A\subseteq V(G)\setminus \{v\}
%     & \text{ with $N(v)\subseteq N[A]$} \\ & \hspace{9mm} \text{we have $|A|> 4\}$.}
%   \end{align*}
% \end{tcolorbox}
%
% The first phase of the algorithm is to compute this set $D_1$ which can be done
% in only 2 rounds of communication.
% The following lemma states that there exist only few such vertices.
%
% \begin{lemma}[Adaptation of \cref{lem:lenzen}]\label{lem:outer-lenzen}
%   We have $|D_1\setminus D|< 2\gamma$.
% \end{lemma}
% \alex{add "same" proof. the edge density is 2.}
%
%
%
% \subsubsection{Step 2}
% Observe now that the second sept of the algorithm is trivial. It is impossible
% to find different vertices $v,z$ such that $|N(v)\cap N(z)|\ge 3$.
%
% \begin{tcolorbox}[colback=red!5!white,colframe=red!50!black]
%     $D_2\coloneqq \emptyset$.
% \end{tcolorbox}
%
% We then define $\e$ like we did in \cref{sec:D2}.
% \begin{tcolorbox}
%   Let $\e\in [0,1]$ be such that $|D_1\cap D| =\e\gamma$.
% \end{tcolorbox}
%
% \subsubsection{Step 3}
% As in previous case, the residual degree is bounded.
% \begin{lemma}\label{lem:outer-degree}
%   Every vertex of $G$ of degree at least $17$ belongs to $D_1$.
% \end{lemma}
% \begin{proof}
%   Let $v$ be a vertices of degree at least $17$ and $v_1,\ldots ,v_4$
%   vertices (other than $v$) such that $N(v)\subseteq \bigcup_{i\le 4}N[v_i]$.
%   Hence there is a $v_i$ with $|N(v)\cap N[v_i]|\geq \lceil 17/4\rceil \geq 4$.
%   Therefore $|N(v)\cap N(v_i)|\geq 3$, and we have a $K_{2,3}$.
% \end{proof}
%
% \begin{tcolorbox}[colback=red!5!white,colframe=red!50!black]
%   For $16\geq i\geq 0$,  for every $v\in R$ in parallel:\\
%   If there is some $u$ with $\dd_R(u)=i$ and $(u,v)\in E(G)$, then\\
%   $\dom_i(v)\coloneqq \{u\}$ (pick one arbitrarily),\\
%   $\dom_i(v)\coloneqq \emptyset$ otherwise.
%   \begin{itemize}
%     \item $R_i \coloneqq R$ \hfill \textit{\small What currently remains to be dominated}
%     \item $\Delta_i \coloneqq \bigcup\limits_{v\in R} \dom_i(v)$ \hfill \textit{\small What we pick this step}
%     \item $R \coloneqq R \setminus N[\Delta_{i}]$ \hfill \textit{\small Update red vertices}
%   \end{itemize}
%   And finally, $D_3:=  \bigcup\limits_{1\le i\le 50} \Delta_i $
% \end{tcolorbox}
%
% \alex{add why it is a dom set, and why the residual degree shrinks (remove whats below)}
%
% As the last step consist of picking $R_9$ i.e.\ all non-dominated vertices, it
% is clear that $D_1\cup D_2\cup D_3$ is a dominating set, and is computed in a
% bounded number of communication rounds. We now turn to bounding the size of $D_3$.
%
%
% \alex{The idea is that now we carry one to the end. We have the same equations,
% with one new one:}
% With this algorithm, we make sure to never take more than what there is left to dominate.
% \begin{lemma}\label{lem:outer-total-H}
%   For every $i\le 16$, $\sum\limits_{j\le i}|\Delta_j| \le |R_i|$.
% \end{lemma}
%
%
%
% \begin{lemma}\label{lem:outer-h1}
%   For every $i\le 16$, $|R_i| \le (i+1)(1-\e)\gamma$.
% \end{lemma}
%
% \begin{lemma}\label{lem:outer-delta}
%   For every $5\le i\le 16$, $|\Delta_i| \le \frac{2|R_i|}{i-4}$.
% \end{lemma}
%
% \begin{lemma}\label{lem:outer-h2}
%   For every $i\le 16$, $|R_i| \le |R_{i+1}| - \frac{(i-3)|\Delta_{i+1}|}{2}$.
% \end{lemma}
%
%
%
%
%
% \alex{changed from 17 to 16. see with Ozan and update result}
% \begin{lemma}\label{lem:outer-size-D3}
%   $|D_3|\le 10.4(1-\e)\gamma$.
% \end{lemma}
% \alex{ref to cplex in appendix}
%
%
%
% \subsubsection{Cleaning up}
% We conclude with the fact that $D_1\cup D_3$ is a dominating set of $G$, and
% that $|D_1\cup D_3|< 2\gamma + \e\gamma + 10.4(1-\e)\gamma$.
% %
% Hence $|D_1\cup D_3|< 12.4\gamma - 8.4\e\gamma \le  12.4\gamma$.
%
% \begin{theorem}\label{thm:outer}
%   There exists a distributed LOCAL algorithm that, for every outerplanar graph
%   $G$, computes in a constant number of rounds a
%   dominating set of size at most $13\gamma(G)$.
% \end{theorem}
