% !TEX root = main.tex

\section{Preliminaries}

In this paper we study the distributed time complexity of finding
dominating sets in undirected and simple graphs in the classical
LOCAL model of distributed computing.
We assume familiarity with this model and refer to the survey~\cite{suomela2013survey} for an extensive
background on distributed computing and on the LOCAL model.

We use standard notation from graph theory and refer to the
textbook~\cite{diestel} for extensive background.  All graphs in this
paper are undirected and simple. We write $V(G)$ for the vertex set of
a graph $G$ and~$E(G)$ for its edge set. The \emph{girth} of a graph
$G$ is the length of a shortest cycle in~$G$. A graph is 
\emph{triangle-free} if it does not contain a triangle (that is, a
cycle of length three) as a subgraph. Equivalently, a triangle-free
graph is a graph with  girth at least four.

A graph is \emph{bipartite} if its vertex set can be partitioned into
two parts such that all its edges are incident with two vertices from
different parts. We write $K_{s,t}$ for the complete bipartite
graph with parts of size $s$ and $t$, respectively. A set $A$ is
\emph{independent} if no two vertices $u,v\in A$ are
adjacent. We write $\alpha(G)$ for the size of the largest
independent set in a graph~$G$. The \emph{Hall ratio} $\rho(G)$ of
$G$ is defined as $\max\bigl\{|V(H)|/\alpha(H)\mid H\subseteq G\bigr\}$.
Hence, every subgraph $H$ of $G$ contains an independent set of
size at least $|V(H)|/\rho(G)$.

%More generally, the \emph{chromatic number}~$\chi(G)$
%of a graph $G$ is the minimum number $k$ such that the vertices of $G$
%can be partitioned into $k$ parts such that every edge connects
%two vertices from different parts. Hence, the bipartite graphs
%are exactly the graphs with chromatic number two.
%Every graph $G$ contains an independent set of size at
%least $\lceil|V(G)|/\chi(G)\rceil$.

A graph~$H$ is a \emph{minor} of a
graph~$G$, written~$H\minor G$, if there is a set
\mbox{$\{G_v :v\in V(H)\}$} of  vertex disjoint and connected
subgraphs $G_v\subseteq G$ such that if~$\{u,v\}\in E(H)$, then there
is an edge between a vertex of~$G_u$ and a vertex of~$G_v$. We 
say that 
$V(G_v)$ is the \emph{branch set} of $v$ and say that it is
\emph{contracted} to the vertex~$v$. 
The set~\mbox{$\{G_v :v\in V(H)\}$} is a \emph{minor model} of $H$.
The  \emph{depth} of a minor model is the maximum radius of one of its branch sets.
 We call $H$ a \emph{$1$-shallow
  minor}, written~$H\minor_1 G$, if $H\minor G$ and there is a minor
model \mbox{$\{G_v :v\in V(H)\}$} with depth at most $1$ witnessing this.
In other
words, $H\minor_1 G$ if $H$ is obtained from $G$ by deleting some
vertices and edges and then contracting a set of pairwise disjoint
stars. We refer to~\cite{nevsetvril2012sparsity} for an in-depth study
of the theory of sparsity based on shallow minors.

A graph is \emph{planar} if it can be embedded in the plane, that is,
it can be drawn on the plane in such a way that its edges intersect
only at their endpoints. By the famous theorem of Wagner, planar
graphs can be characterized as those graphs that exclude the complete
graph $K_5$ on five vertices and the complete bipartite $K_{3,3}$ with
parts of size three as a minor. In particular, a minor of a planar graph
is again planar. 

A graph is \emph{outerplanar} if it has an embedding in the plane such
that all vertices belong to the unbounded face of the embedding.
Equivalently, a graph is outerplanar if it does not contain the
complete graph $K_4$ on four vertices and the complete bipartite graph
$K_{2,3}$ with parts of size~$2$ and~$3$, respectively, as a minor. 
Again, a minor of an outerplanar graph is again outerplanar. 

By Euler's formula, planar graphs are sparse: every planar $n$-vertex
graph ($n\geq 3$) has at most $3n-6$ edges (and a graph with at most
two vertices has at most one edge).
The ratio $|E(G)|/|V(G)|$ is
called the \emph{edge density} of $G$.
In particular, every planar
graph $G$ has edge density strictly smaller than three.
We define
\begin{align*}
\nabla_0(G)&=\max\Bigl\{\ \frac{|E(H)|}{|V(H)|}\ \Bigr|\  H\subseteq G\Bigr\},\\
\nabla_1(G)&=\max\Bigl\{\ \frac{|E(H)|}{|V(H)|}\ \Bigr| \  H\minor_1 G\Bigr\},\\
\nabla_1^B(G)&=\max\Bigl\{\ \frac{|E(H)|}{|V(H)|}\ \Bigr| \  H\minor_1 G, H\text{ bipartite}\Bigr\}.
\end{align*}


  It is immediate 
that $\nabla_0(G) \le \nabla_1^B(G)\le\nabla_1(G)$. Note that every graph $G$ (and each of its subgraphs)
contains a vertex with degree at most $2\nabla_0(G)$. By iteratively
removing a minimum degree vertex and its neighbors, we can  find a 
large independent set, as stated in the next lemma. The bounds for graphs
on surfaces are well known.

% Every
% graph $G$ has a vertex of degree at most
% $2\nabla_0(G)$ and is $2\nabla_0(G)+1$-colorable.
%
% \begin{lemma}\label{lem:densities}
%   Let $G$ be a planar graph. Then the edge density of $G$ is strictly
%   smaller than $3$ and $\chi(G)\leq 4$. Furthermore,
%
%   \vspace{-2mm}
%   \begin{enumerate}
%   \item if $G$ is bipartite, then the edge density of $G$ is strictly
%     smaller than $2$ and $\chi(G)\leq 2$,\\[-6mm]
%   \item if $G$ is triangle-free or outerplanar, then the edge density
%     of $G$ is strictly smaller than $2$ and $\chi(G)\leq 3$.
%     % and\smallskip
%     % \item if $G$ has girth at least $5$, then the edge density of
%     %   $G$ is strictly smaller than $\frac{5}{3}$ and
%     %   $\chi(G)\leq 3$.
%   \end{enumerate}
% \end{lemma}
\begin{lemma}\label{lem:bounds}
For all graphs $G$ we have $\rho(G)\leq 2\nabla_0(G)+1$. Furthermore,
  \begin{enumerate}
    \item if $G$ is planar, then $\nabla_0(G)<3$, $\nabla_1^B(G)<2$ and $\rho(G)\le 4$;
    \item if $G$ is outerplanar, or planar and triangle free, then $\nabla_0(G)<2$ and $\rho(G)\le 3$;
    \item if $G$ is planar and bipartite, then $\nabla_0(G)<2$ and $\rho(G)\le 2$.
  \end{enumerate}
\end{lemma}

For a graph $G$ and $v\in V(G)$ we write
$N(v)=\{u\colon\{u,v\}\in E(G)\}$ for the \emph{open neighborhood} of $v$
and $N[v]=N(v)\cup\{v\}$ for the \emph{closed neighborhood}
of~$v$. For a set $A\subseteq V(G)$ let $N[A]=\bigcup_{v\in A}N[v]$.
% We write $N_r[v]$ for the set of vertices at distance at most $r$
% from a vertex $v$.
A~\emph{dominating set} in a graph~$G$ is a set $D\subseteq V(G)$ such
that $N[D]=V(G)$. We write~$\gamma(G)$ for the size of a minimum
dominating set of $G$. For a set $R\subseteq V(G)$ we say that a set
$Z\subseteq V(G)$ \emph{dominates} or \emph{covers}
$R$ if $R\subseteq N[Z]$. For $v\in V(G)$ we let 
$\Nr(v)=N(v)\cap R$ and $\dr(v)=|\Nr(v)|$. 


%\smallskip
% In the following we mark important definitions and assumptions about
% our input graph in gray boxes and steps of the algorithm in red boxes.



An \emph{orientation} of a graph $G$ is a directed graph $\vec{G}$ that
 has exactly one of the arcs $(u,v)$ and
$(v,u)$
for each edge $\{u,v\}\in E(G)$. The \emph{out-degree} $d^+(v)$ of a vertex $v$ in  $\vec{G}$
is the number of arcs leaving~$v$. The following
lemma is implicit in the work of Hakimi~\cite{sup_orie}, see \mbox{also~\cite[Proposition 3.3]{nevsetvril2012sparsity}.}

\begin{lemma}\label{lem:orientations}
  Every graph $G$ has an
  orientation with maximum out-degree $\nabla_0$.
\end{lemma}
%\begin{proof}
%  We start with an arbitrary orientation $\vec{G}$. Assume there is a vertex $v$
%  with out-degree greater than~$d$. Consider the subgraph $\vec{H}$
%  induced by all vertices that are reachable from $v$. The underlying undirected
%  graph $H$ has $\sum_{v\in V(H)}d^+(v)$ many edges. As $v$ has
%  out-degree strictly larger than $d$ we conclude that there must be one
%  vertex $w$ in $\vec{H}$ that has out-degree strictly smaller than $d$. By definition
%  of $\vec{H}$ there exists a directed path from $v$ to $w$. By reversing the
%  edges on that path we decrease the out-degree of $v$ by one, increase the
%  out-degree of $w$ by one and leave the out-degrees of all other vertices of
%  the path unchanged. By iterating this procedure we successively improve
%  the orientation until no vertex of out-degree greater than $d$ remains.
%\end{proof}

%As every subgraph of a (triangle-free or outerplanar) planar graph
%is again a (triangle-free or outerplanar) planar graph we
%have the following corollary.
We immediately deduce the next corollary.

\begin{corollary}\label{cor:planar-orientations}
  Let $G$ be a planar graph. Then
  \begin{enumerate}
    \item $G$ has an orientation with maximum out-degree $3$.%\smallskip
    \item If $G$ is triangle-free or outerplanar, then
    $G$ has an orientation with maximum out-degree~$2$.
  \end{enumerate}
\end{corollary}

\begin{tcolorbox}
  \begin{tabular}{p{.45\textwidth}p{.5\textwidth}}
  	  In the following we fix&and (possibly defined in terms of $\nabla_1$)\\
  	  \hlx{vv}
  	\begin{itemize}
  		\item the input graph $G$,
  		\item a minimum dominating set~$D$,
  		\item $\gamma \coloneqq |D|$,
  		\item the parameter $\nabla_1\ge \nabla_1(G)$,
  	\end{itemize}
  	&
  	\begin{itemize}
  		\item the parameter $\nabla_0\in[\nabla_0(G),\nabla_1]$,
  		\item the parameter $\nn\in (\nabla_1^B(G),\nabla_1+1]$,
  		\item the parameters $s\leq t$ with $K_{s,t}\not\subseteq G$,
  		%\item the parameter $g\ge g(G)$,
  	\end{itemize}\\
  \hlx{vv}
  \end{tabular}\\
  where $D$ and $\gamma$ are used only to analyze the performance of the algorithm, and where all the  parameters are integers.
%  \begin{itemize}
%  \item a graph $G$,
%  \item a minimum dominating set~$D$ of $G$,
%  \item $\gamma \coloneqq |D|$,
%  \item $\nabla_0\ge \nabla_0(G)$,
%  \item $\nabla_1\ge \nabla_1(G)$,
%  \item $\nn > \nabla_1^B(G) $,
%  \item integers $s,t$ such that $K_{s,t}\not\subseteq G$, and
%  \item the genus $g$ of $G$.
%  \end{itemize}
\end{tcolorbox}

Above, $[a,b]$ denotes the closed real interval containing all 
$x$ with $a\leq x\leq b$ and $(a,b]$ denotes the half-open 
interval containing all $x$ with $a<x\leq b$. We can choose
$s=t=2\lfloor\nabla_0(G)\rfloor+1$.

%The algorithm can be called with a set of parameters bounding
%$\nabla_0$, $\nabla_1, s,t$ and $g$ to give the best approximation
%ratio. It does not require all of these parameters as input though,
%it is also sufficient to give as input only a bound on $\nabla_1$, as $\nabla_1$
%functionally bounds all other parameters as follows.
