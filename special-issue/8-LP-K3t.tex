% !TEX root = main.tex

\subsection{Phase 3: LP-based approximation}\label{sec:LP-planar}
%
%\sebi{Update the general case such that we can work
%with arbitrary $\epsilon$.}
%
%Recall that we chose $\rho$ such that $|(D_1 \cup D_2)\cap D| =\rho\gamma$.
%When $\rho\gamma$ vertices of $D$ were already chosen into
%the partial dominating set $D_1\cup D_2$ we have $|D_R|\leq (1-\rho)\gamma$.
%With \cref{cor:planar-orientations} we conclude the following corollary.
%
%\sebi{Put this to the general case.}
%
%\begin{corollary}\label{cor:LP-approx}
%  Let $G$ be a graph that has an orientation with maximum out-degree
%  $d$, let $R\subseteq V(G)$, let~$D_R$ be a
%  minimum dominating set of $R$, and let $\e>0$. Then we can
%  compute a set $D'$ of size at most~$(2d+1)(1+\e)|D_R|$ that dominates
%  $R$ in $\Oof(\log \Delta/(\log (1+\e))$ rounds in the LOCAL
%  model.
%
%  In particular, for our algorithm when
%
%  \vspace{-2mm}
%  \begin{enumerate}
%    \item $G$ is planar, then $|D'|\leq 7(1+\e)|D_R|
%      \leq 7(1+\e)(1-\rho)\gamma$, and when
%    \item $G$ is planar and triangle-free or outerplanar, then
%      $|D'|\leq 5(1+\e)|D_R|\leq 5(1+\e)(1-\rho)\gamma$.
%  \end{enumerate}
%\end{corollary}

We now proceed with the LP-based approximation as in the general case
presented in \cref{sec:LP}. Recall that for any desired $\epsilon>0$
we defined $\Gamma$ as an high degree and defined
$D_3^1$ as the set of all vertices with degree greater than $\Gamma$.
We added $D_3^1$ to the dominating set and were able to call the
LP-based approximation algorithm of \cref{cor:LP-approx-general}. We finally
obtained a set $D_3$ dominating the remaining vertices with
$|D_3|\leq (2\nabla_0+1)(1+\epsilon)\gamma_R$ according to \cref{lem:D3-LP}.


We now conclude our main theorem, \cref{thm:K3t-free-total}, stating
that the algorithm on $K_{3,t}$-free graphs computes a dominating
set of size at most $(6\nabla_1+3)\gamma$.

\begin{proof}[Proof of \cref{thm:K3t-free-total}]
First, $D_1,D_2$, and $D_3$ are computed locally, in a bounded number of
  rounds, and additionally the set $D_1 \cup D_2 \cup D_3$ dominates $G$. Then,
  \begin{align*}
  	|D_1\cup D_2|&<\rho(G)\gamma+2\nabla\eta\gamma&\text{(by \cref{lem:size-D12})}\\
  	&\leq (2\nabla_1+1)\gamma+2\nabla_1\eta\gamma&\text{(as $\rho(G)\leq 2\nabla_1+1$ and $\nabla\leq \nabla_1$)}
	\intertext{and}
  	|D_3|&\leq (2\nabla_0+1)(1+\epsilon)(1-\eta)\gamma&\text{(by \cref{lem:D3-LP})}\\
	&\leq (2\nabla_1+1)(1+\epsilon)(1-\eta)\gamma&\text{(as $\nabla_0\leq \nabla_1$)}
  	\intertext{ By choosing $\epsilon=1$,}
  	|D_1 \cup D_2 \cup D_3|  &\leq (2\nabla_1+1+2\nabla_1\eta+(4\nabla_1+2)(1-\eta))\gamma\\
  	&\leq  (6\nabla_1+3)\gamma&\text{(maximized when $\eta=0$)}
  \end{align*}
\end{proof}

\subsection{Planar graphs}

Finally, we complete the analysis of our algorithm for planar graphs,
showing that it computes an $(11+\epsilon)$-approximation. In the
following we fix a planar graph $G$.

\begin{proof}[Proof of \cref{thm:planar}]
We revisit the proof of \cref{thm:K3t-free-total} and plug in the
numbers for planar graphs. For planar graphs we have
$K_{3,3}\not\subseteq G$, $\nabla_0=3$, $\nabla=2$ and
$\rho=4$, as stated in \cref{lem:bounds}.
Therefore, by \cref{lem:size-D12} we have $|D_1\cup D_2|<4\gamma+4\eta\gamma$   and by \cref{lem:D3-LP} we have \linebreak $|D_3|\leq (7+\e)(1-\eta)\gamma$. Hence, $|D_1 \cup D_2 \cup D_3|\leq \gamma(4+4\eta +7 - 7\eta +\e - \e\eta)
  \leq \gamma(11+\e-3\eta -\e\eta)$.
  As $\eta\in[0,1]$, this is maximized when $\eta=0$. Hence
  $|D_1 \cup D_2 \cup D_3| \le \gamma(11+\e)$.
\end{proof}
