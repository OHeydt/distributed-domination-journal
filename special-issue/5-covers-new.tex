% !TEX root = main.tex

\section{Phase 2: Reducing residual degrees -- pseudo-covers and domination sequences}\label{sec:phase2}

After the first phase of the algorithm we have established the situation
that for every vertex \mbox{$v\in V(G)\setminus D_1$} the residual neighborhood
$\Nr(v)$ is dominated by set $A_v\subseteq V(G)\setminus\{v\}$
of size at most $2\nabla$.  For all vertices of
$V(G)\setminus \hat{D}$ we have chosen $A_v\subseteq D\setminus\{v\}$.
Observe that the set $\bigcup_{v\in V(G)}A_v$ has very good
domination properties. First, already the sets $A_v$ for $v\in D$
dominate almost all vertices that remain to be dominated, except possibly
the vertices of $D$ itself: We have $R\subseteq \bigcup_{v\in D} N_R[v]
= D\cup \bigcup_{v\in D} N_R(v)=D\cup \bigcup_{v\in D} A_v$, hence
$R\setminus D\subseteq \bigcup_{v\in D}A_v$.
Second, $\bigcup_{v\in V(G)}A_v$ is small, as
$|\bigcup_{v\in V(G)}A_v|\leq
\sum_{v\in V(G)} |A_v|
=\sum_{v\in \hat{D}}|A_v|+ \sum_{v\in V(G)\setminus\hat{D}}|A_v|$.
So~$|\bigcup_{v\in V(G)}A_v|
\leq (\rho(G)+1)\gamma\cdot 2\nabla+\gamma
\in \Oof(\gamma)$.

\pagebreak
In the second phase of
the algorithm we aim to find a good approximation of the sets $A_v$.
We follow the approach of Czygrinow et al.~\cite{czygrinow2018distributed}
and define \emph{pseudo-covers}, which describe candidate
vertices for the sets $A_v$. We will then consider a selection process that
can be carried out in parallel for all vertices, which is based on the
definition of \emph{domination sequences}, and allows to select
a bounded number of candidate vertices. The domination properties
of the selected vertices are worse than that of the sets $A_v$, however,
at the end of the second phase we will be in the situation that the
residual degree of each vertex is bounded by an absolute constant
depending only on the graph class under consideration.\vspace{-2mm}

\subsection{Pseudo-covers}

Following the presentation of~\cite{czygrinow2018distributed}, we name and fix
the following constants for the rest of this article. The reason to choose
the constants as given will become clear in the course of the proof.

\begin{tcolorbox}
\hfill
\begin{tabular}{l l}
$\kappa$       & $\coloneqq~ \max\{2\nabla_0,2\nn\}$,\\
$\lambda$  & $\coloneqq~ 1/\kappa$,
\vspace{-1mm}
\end{tabular}
\hfill
\begin{tabular}{l l}
$\mu$    & $\coloneqq~ 2\kappa/\lambda=2\kappa^2$,\\
$\nu$       & $\coloneqq~ k\mu = 2\kappa^3$.
\end{tabular}
\hfill~
\end{tcolorbox}


\begin{definition}
A vertex $z\in V(G)$ is \emph{$\lambda$-strong} for a vertex set $W\subseteq V(G)$ if $|N[z]\cap W|\geq \lambda|W|$.
\end{definition}

The following is the key definition by Czygrinow et al.~\cite{czygrinow2018distributed}.

\begin{definition}
  A \emph{pseudo-cover} (with parameters $\kappa, \lambda, \mu, \nu$)
  of a set $W\subseteq V(G)$ is a
  sequence $(v_1,\ldots, v_m)$ of vertices
  such that for every $i\le m$ we have:\\[-6mm]
  \begin{itemize}
    \item $m\leq \kappa$,\\[-5mm]
    \item $v_i$ is $\lambda$-strong for $W\setminus\bigcup_{j<i}N[v_j]$,\\[-5mm]
    \item $|N[v_i]\cap (W\setminus\bigcup_{j<i} N[v_j])|\geq \mu$, and\\[-5mm]
    \item $|W\setminus \bigcup_{j\le m}N[v_j]|\leq \nu$.
  \end{itemize}
\end{definition}
Intuitively, all but at most $\nu$ elements of the set $W$ are covered by the $(v_i)_{i\le m}$.
Additionally, each element $v_i$ of the pseudo-cover dominates both an
$\lambda$-fraction of the part of $W$ that is not yet dominated by the $v_j$
for $j<i$, and at least $\mu$ elements.
Note that with our choice of constants, if there are more than $\nu$ vertices not
covered yet, any vertex that covers a $\lambda$-fraction of what remains also
covers at least $\mu$ elements.


The next lemma shows how to derive the existence of pseudo-covers from
the existence of small dominating sets.

\begin{lemma}\label{lem:cover-to-pseudo-cover}
  Let $W\subseteq V(G)$ be of size at least
  $\nu$ and let~$Z$ be a dominating set of $W$ with~$\kappa$ elements.
  There exists an ordering of the vertices of $Z$ as $z_1,\ldots, z_\kappa$
  and $m\leq \kappa$ such that $(z_1,\ldots, z_m)$ is a pseudo-cover of $W$.
\end{lemma}
\begin{proof}
  We build the order greedily by induction. We order the elements by neighborhood size, while removing the neighborhoods of the previously ordered vertices. More precisely, assume that $(z_1,\ldots,z_i)$ have been defined for \mbox{some~$i\ge 0$}. We then define $z_{i+1}$ as the element that maximizes $|N[z] \cap (W \setminus \bigcup_{j\le i}N[z_j])|$.

  Once we have ordered all vertices of $Z$, we define $m$ as the maximal integer not larger than $\kappa$ such that for every $i \le m$ we have:
  \begin{itemize}
    \item $z_i$ is $\lambda$-strong for $W\setminus\bigcup_{j<i}N[z_j]$, and
    \item $|N[z_i] \cap (W \setminus \bigcup_{j\le i}N[z_j])| \ge \mu$.
  \end{itemize}

  This ensures that $(z_1,\ldots, z_m)$ satisfies the first 3 properties of a
  pseudo-cover of $W$. It only remains to check the last one.
  To do so, we define $W' \coloneq W \setminus\bigcup_{i\le m}N[z_i]$. We want to prove
  that $|W'| \le \nu$. Note that because $Z$ covers~$W$, if $m=\kappa$ we
  have $W'=\emptyset$ and we are done. We can therefore assume
  that $m<\kappa$ and $W'\neq \emptyset$. Since $Z$ is a dominating set of $W$,
  we also know that $(z_{m+1},\ldots z_\kappa)$ is a dominating set of $W'$,
  therefore there is an element in $(z_{m+1},\ldots z_\kappa)$ that
  dominates at least a $1 / \kappa$ fraction of $W'$. Thanks to the
  previously defined order, we know that~$z_{m+1}$ is such an element.
  Since $\lambda = 1/\kappa$, it follows that~$z_{m+1}$ is $\lambda$-strong
  for $W'$.
  This, together with the definition of $m$, we have that $|N[z_i] \cap (W \setminus \bigcup_{j\le i}N[z_j])| < \mu$ meaning that $|N[z_{m+1}] \cap W'| < \mu$. This implies that $|W'|/\kappa < \mu$. And since $\mu = \nu/\kappa$, we have $|W'|<\nu$.
  %
  Hence, $(z_1,\ldots, z_m)$ is a pseudo-cover of $W$.
\end{proof}

%We will apply the pseudo-covers to neighborhoods
%of vertices.

While there can exist unboundedly many dominating sets for a set $W\subseteq V(G)$,
a nice observation of Czygrinow et al.\ was that the number of
pseudo-covers is bounded whenever the input graph excludes some
biclique $K_{s,t}$ as a subgraph. We do not state the result in this
generality, as it leads to enormous constants. Instead, we focus on the
case where small dominating sets $A_v$ exist, implying that $\nabla_0(G)$,
and therefore $\kappa$, are bounded.

% \alex{Old version:}
%
%   \begin{lemma}\label{lem:num-high-degree-old}
%     Let $W\subseteq V(G)$ of size at least $8\nabla_1 / \alpha^2$.
%     Then there are less than $4\nabla_1/\alpha$ vertices that are
%     $\alpha$-strong for $W$.
%   \end{lemma}
%   \begin{proof}
%     Assume that there is such a set $W$ with at least $c\coloneq 4\nabla_1 /\alpha$ vertices that are $\alpha$-strong for~$W$.
%     We build a $1$-shallow minor $H$ of the graph $G$ with $|W|$ branch sets.
%     Each branch set is either a single element of $W$, or a pair $\{w,a\}$, where $w$ is in $W$ and $a$ is an $\alpha$-strong vertex for $W$, connected to $w$, and that is not in $W$. This is obtained by iteratively contracting one edge of an $\alpha$-strong vertex with a vertex of $W$. This is possible because $\alpha |W|>c$, so during the process and for any \mbox{$\alpha$-strong} vertex we can find a connected vertex in $W$ that is not part of any contraction.
%
%     Once this is done, we have that $|V_H|=|W|$. For the edges, each of the \mbox{$\alpha$}-strong vertices can account for $\alpha |W|$ many edges. We need to subtract~$c^2$ from the total as we do not count twice an edge between two strong vertices. Therefore $|E_H| \ge c\alpha |W| - c^2$. Note also that because $|W| \ge 8\nabla_1 / \alpha^2$, we have that $ 2\nabla_1\ge (4\nabla_1)^2 / (\alpha^2|W|)$. All of this together leads to:
%
%     \[\frac{|E_H|}{|V_H|} \ge \frac{c\alpha |W| - c^2}{|W|} \ge 4 \nabla_1 - \frac{(4\nabla_1)^2}{\alpha^2|W|} \ge 4\nabla_1 - 2\nabla_1 > \nabla_1.\]
%
%     This contradicts the definition of $\nabla_1$.
%   \end{proof}


% \alex{new lemma, $4\nabla_1$ got replaced by $2\nabla_0$}

\begin{lemma}\label{lem:num-high-degree}
  Let $W\subseteq V(G)$ of size at least $\mu$.
  Then there are less than $\kappa^2$ vertices that are
  $\lambda$-strong for $W$.
\end{lemma}
\begin{proof}
  Assume that there is such a set $W$ with $|W|\ge \mu$ and
  $c$ many vertices that are $\lambda$-strong for~$W$.
  Let $H$ be the subgraph of $G$ induced by $W$ and the $\lambda$-strong vertices.
  We first have that $|V_H| \le |W|+c$. Second we have that
  $|E_H|\ge \lambda|W|c-\nabla_0c$, because there are $c$ vertices that have
  degree at least $\lambda|W|$ and there are at most $\nabla_0c$ many vertices
  between them.

\pagebreak
  We then have that $|E_H|\le \nabla_0|V_H|$ hence
  $\lambda|W|c-\nabla_0c\le \nabla_0(|W|+c)$ from which we
  derive~$c(\lambda |W|-2\nabla_0)\le \nabla_0|W|$.
  Now, using that $|W|>\mu\ge 2k^2$ we have $\lambda|W|>2k>4\nabla_0$.
  Hence $\lambda|W|-2\nabla_0 \ge \lambda|W|/2$.
  We can finally deduce that $c(\lambda|W|/2)\le \nabla_0|W|$ and therefore we
  have that $c\le 2\nabla_0/\lambda=\kappa^2$.
\end{proof}


This leads quickly to a bound on the number of pseudo-covers.

% \alex{old version:}
%   \begin{lemma}\label{lem:num-pseudo-covers-old}
%     For every $W\subseteq V(G)$ of size at least $\ell$, the number of
%     pseudo-covers is less than~$(4\nabla_1/\alpha)^k$.
%   \end{lemma}
%
%   The proof of the lemma is exactly as the proof of Lemma~7 in the
%   presentation of Czygrinow et al.~\cite{czygrinow2018distributed},
%   we reprove it for the sake of completeness.
%
%
%   \begin{proof}
%     Let $W$ a set of size at least $\ell$. For every $i\le k$, we define $C_i$ as the
%     set of partial pseudo-covers of~$W$ of size at most $i$, that is,
%     all sets of at most $i$ vertices that can be extended to a pseudo-cover
%     of~$W$.
%     So $C_k$ is the set of pseudo-covers of $W$ while $C_1$ only contains
%     $\alpha$-strong vertices for $W$.
%
%     \cref{lem:num-high-degree} implies that $|C_1|< 4\nabla_1/\alpha$.
%     \cref{lem:num-high-degree} also implies that for every $i<k$, we have
%     $|C_{i+1}| < |C_{i}|\cdot (4\nabla_1/\alpha)$. We therefore conclude that
%     $|C_k| < (4\nabla_1/\alpha)^k$.
%   \end{proof}

\begin{lemma}\label{lem:num-pseudo-covers}
  For every $W\subseteq V(G)$ of size at least $\mu$, the number of
  pseudo-covers is less than~$\kappa^{2\kappa}$.
\end{lemma}

The proof of the lemma is exactly as the proof of Lemma~7 in the
presentation of Czygrinow et al.~\cite{czygrinow2018distributed},
we reprove it for the sake of completeness.


\begin{proof}
  Let $W$ a set of size at least $\mu$. For every $i\le \kappa$, we define $C_i$ as the
  set of partial pseudo-covers of~$W$ of size at most $i$, that is,
  all sets of at most $i$ vertices that can be extended to a pseudo-cover
  of~$W$.
  So $C_\kappa$ is the set of pseudo-covers of $W$ while $C_1$ only contains
  $\lambda$-strong vertices for $W$.

  \cref{lem:num-high-degree} implies that $|C_1|< \kappa^2$.
  \cref{lem:num-high-degree} also implies that for every $i<\kappa$, we have
  $|C_{i+1}| < |C_{i}|\cdot \kappa^2$. We therefore conclude that
  $|C_k| < (\kappa^2)^\kappa$.
\end{proof}

\begin{tcolorbox}
  We write $\Tt(v)$ for the set of all pseudo-covers
  of $N(v)$ and $\Pp(v)$ for the set of all vertices that appear in a
  pseudo-cover of $N(v)$.
\end{tcolorbox}

The proof of \cref{lem:num-pseudo-covers} also bounds the number $\Pp(v)$.
\begin{corollary}\label{cor:nb-dominators}
For every $v\in V(G)$ with $|N_R(v)|> \mu$, we have
  % $|\Tt(v)|< k^{2k}$ and
  $|\Pp(v)|\le \kappa^{2\kappa}$.
\end{corollary}

%We recall all fixed parameters for easy to find reference.
%\textcolor{red}{Motivate the definition of these parameters in
%the construction of the pseudo covers.}
%
%\bigskip
%\begin{tcolorbox}
%\begin{tabular}{l l}
%- $G$ & :~ fixed graph. \\
%- $\gamma$ & :~ $\gamma(G)$.\\
%- $\nabla_1$ & :~ $\nabla_1(G)$. \\
%- $t$ & :~ smallest integer such that $G$
%excludes $K_{t,t}$ as a subgraph\\
%- $D_1$ & :~ defined and computed in \cref{lem:neighborhood-dom1}.\\
%- $D$ & :~ fixed dominating set of $G$ of size $\gamma$ (not computed).\\
%- $\hat{D}$ & :~ defined in \cref{lem:neighborhood-dom2} (not computed).\\
%- $D'$ & :~ $D \cup \hat{D}$ (not computed).
%\end{tabular}
%\end{tcolorbox}

% For the rest of this article we fix
% $\alpha\coloneqq 1/k$, $\ell=8\nabla_1/\alpha^2+1=4k^3+1$ and $q=4k^4$.
% These constants will be needed in the following definitions of pseudo-covers
% and domination sequences.
% We also fix the following constants to follow the presentation
% of~\cite{czygrinow2018distributed}.
