% !TEX root = main.tex


\section{$K_{3,t}$-free graphs}

We now turn our attention to graphs 
that exclude $K_{3,t}$ for some $t$ (and with bounded $\nabla_1$). 
The most prominent graphs
with these properties are graphs that embed into a surface of bounded
genus and in particular planar graphs. 


\subsection{Phase 1: Preprocessing}

The general preprocessing phase described in \cref{sec:step1} remains 
unchanged. Recall that we defined $D_1$ as $\{v\in V(G) : \text{ for all } A\subseteq V(G)\setminus \{v\} \text{ with $N(v)\subseteq N[A]$ we have $|A|> (2\nabla-1)\}$}.$ Recall as well that for every $v\in V(G)\setminus D_1$, we fixed
  $A_v\subseteq V(G)\setminus \{v\}$ such that
  $\Nr(v)\subseteq N[A_v]$ and $|A_v|\leq 2\nabla-1$. Furthermore, 
  for $V(G)\setminus \hat{D}$ we
assumed $A_v\subseteq D\setminus\{v\}$.
 
\subsection{Phase 2: local dominators in the $K_{3,t}$-free case}\label{sec:D2}

In this second phase, things get simpler than in \cref{sec:phase2}. Since
we now assume that we exclude $K_{3,t}$ the domination sequences of \cref{def:dom-sequence} only have
length two. We can therefore simplify the analysis of the domination sequences. We simply select every pair
of vertices with sufficiently many neighbors in common.

\begin{tcolorbox}[colback=red!5!white,colframe=red!50!black]
\begin{itemize}
\item For $v\in V(G)$ let
  $B_v\coloneqq \{z\in V(G)\setminus \{v\}: |\Nr(v)\cap \Nr(z)|\geq
  (2\nabla-1)t+1\}$.\smallskip
\item Let $W$ be the set of vertices $v\in V(G)$ such that
  $B_v \neq \emptyset$.\smallskip
\item Let $D_2\coloneqq \bigcup\limits_{v\in W} (\{v\}\cup B_v)$.
\end{itemize}
\end{tcolorbox}

Once $D_1$ has been computed in the previous phase, 2 more rounds of
communication are enough to compute the sets $B_v$ and $D_2$.
%
Before we update the residual degrees, let us analyze the sets $B_v$
and~$D_2$.  First note that the definition is symmetric: since
$\Nr(v)\cap \Nr(z)=\Nr(z)\cap \Nr(v)$ we have for all $v,z\in V(G)$ if
$z\in B_v$, then $v\in B_z$. In particular, if $v\in D_1$ or
$z\in D_1$, then $\Nr(v)\cap \Nr(z)=\emptyset$, which immediately
implies the following lemma.

\begin{lemma}\label{lem:WcapD1}
  We have $W\cap D_1=\emptyset$ and for every $v\in V(G)$ we have
  $B_v\cap D_1=\emptyset$.
\end{lemma}
%\begin{proof}
%  Let $v,z\in V(G)$. If $v\in D_1$, then
%  $\Nr(v)=\emptyset$. Similarly if $z\in D_1$, then
%  $\Nr(v)\cap N(z)=N(v)\cap \Nr(z) =\emptyset$.
%\end{proof}

Now we prove that for every $v\in W$, the set $B_v$ cannot be too big,
and has nice properties. 

\begin{lemma}\label{lem:dominating-dominators}
  For all vertices $v\in W$ we have

  \vspace{-5pt}
  \begin{itemize}
  \item $B_v \subseteq A_v$, (hence $|B_v|\leq (2\nabla-1)$) and \smallskip
  \item if $v\not\in \hat{D}$, then $B_v\subseteq D$.
  \end{itemize}
\end{lemma}

\begin{proof}
  Assume $A_v=\{v_1,\ldots, v_\ell\}$ (a set of possibly not distinct
  vertices) and assume there exists
  $z\in V(G)\setminus \{v,v_1,\ldots v_\ell\}$ with
  $|\Nr(v) \cap \Nr(z)| \geq (2\nabla-1)t+1$.  As $v_1, \ldots, v_\ell$ dominate~$\Nr(v)$,
  and hence also \mbox{$\Nr(v)\cap \Nr(z)$}, and $\ell\leq (2\nabla-1)$, there must be some~$v_i$,
  $1\leq i\leq \ell$, with
  \mbox{$|\Nr(v) \cap \Nr(z) \cap N[v_i]| \geq \lceil ((2\nabla-1)t+1)/(2\nabla-1)\rceil \geq
    t+1$}.  Therefore, $|\Nr(v) \cap \Nr(z) \cap N(v_i)| \geq t$,
  which shows that $K_{3,t}$ is a subgraph of~$G$, contradicting the
  assumption.

  If furthermore $v\not\in \hat{D}$, by definition of $\hat{D}$, we
  can find $w_1,\ldots, w_\ell$ from $D$ that dominate $N(v)$, and in
  particular $\Nr(v)$.  If $z\in V(G)\setminus \{v,w_1,\ldots, w_\ell\}$
  with $|\Nr(v) \cap \Nr(z)| \geq (2\nabla-1)t+1$ we can argue as above to obtain
  a contradiction.
\end{proof}

\pagebreak

% In the light of \cref{lem:dominating-dominators}, we select all
% paires of nodes with sufficiently large intersecting neighborhood.
%
% \begin{tcolorbox}
%   For $v\in V(G)$ let $B_v\coloneqq \{z\in V(G)\setminus \{v\}:
%   |N(v)\cap N(z)|\geq 19\}$.
% \end{tcolorbox}

% \begin{corollary}\label{cor:dominating-dominators}
%   For every vertex $v$, $B_v\subseteq A_v$, in particular,
%  $|B_v|\leq 6$ and if $v\not\in \hat{D}$, then $B_v\subseteq D$.
% \end{corollary}
%
% \begin{tcolorbox}
%   We define $W$ as the set of vertices $v\in V(G)$ such
%   that $B_v\neq \emptyset$. We define \[D_2\coloneqq \bigcup_{v\in W}
%   (\{v\}\cup B_v).\]
% \end{tcolorbox}
% \vspace{0mm}

% Our algorithm now proceeds as follows. Obviously, every vertex $v$
% can locally compute the set $B_v$. The algorithm adds the set $D_2$
% to the dominating set, removes~$D_2$ from the graph and marks all
% vertices dominated by $D_2$ as dominated.  \alex{here again 'remove'
% vertices?}

Let us now analyze the size of $D_2$. For this we refine the set $D_2$
and define
\begin{tcolorbox}
  \begin{enumerate}
    \item $D_2^1\coloneqq \bigcup_{v\in W\cap D}
    (\{v\}\cup B_v)$, \smallskip
    \item $D_2^2\coloneqq \bigcup_{v\in W\cap (\hat{D}\setminus D)}
    (\{v\}\cup B_v)$, and \smallskip
    \item $D_2^3\coloneqq \bigcup_{v\in W\setminus (D\cup \hat{D})}
    (\{v\}\cup B_v)$.
  \end{enumerate}
\end{tcolorbox}

\smallskip
Obviously $D_2=D_2^1\cup D_2^2\cup D_2^3$. We now bound the size of the
refined sets $D_2^1,D_2^2$ and $D_2^3$.

\begin{lemma}\label{lem:size-D21}
  $|D_2^1\setminus D|\leq (2\nabla-1)\gamma$.
\end{lemma}
\begin{proof}
  We have
  \[|D_2^1\setminus D|= |\bigcup_{v\in W\cap D} (\{v\}\cup
    B_v)\setminus D|\leq |\bigcup_{v\in W\cap D}B_v|\leq \sum_{v\in
      W\cap D}|B_v|.\] By \cref{lem:dominating-dominators} we have
  $|B_v|\leq (2\nabla-1)$ for all $v\in W$ and as we sum over $v\in W\cap D$ we
  conclude that the last term has order at most $(2\nabla-1)\gamma$.
\end{proof}

\begin{lemma}\label{lem:size-D22}
  $D_2^2 \subseteq \hat D$ and therefore
  $|D_2^2\setminus D|< \rho(G)\gamma$.
\end{lemma}
\begin{proof}
  Let $v\in \hat{D}\setminus D$ and let $z\in B_v$. By symmetry,
  $v\in B_z$ and according to \cref{lem:dominating-dominators}, if
  $z\not\in \hat{D}$, then $v\in D$.  Since this is not the case, we
  conclude that $z\in\hat{D}$.  Hence $B_v\subseteq \hat{D}$ and, more
  generally, $D_2^2\subseteq \hat{D}$.  Finally, according to
  \cref{lenzen-improved} we have $|\hat{D}\setminus D|<\rho(G)\gamma$.
\end{proof}

Finally, the set $D_2^3$, which appears largest at first glance, was
actually already counted, as shown in the next lemma.
\begin{lemma}\label{lem:size-D23}
  $D_2^3\subseteq D_2^1$.
\end{lemma}
\begin{proof}
  If $v\not\in \hat{D}$, then $B_v\subseteq D$ by
  \cref{lem:dominating-dominators}.  Hence $v\in B_z$ for some
  $z\in D$, and $v\in D_2^1$.
\end{proof}

Recall that we defined $\eta\in [0,1]$ to be such that $|(D_1 \cup D_2)\cap D| =\eta\gamma$.

% \begin{lemma}\label{lem:size-D2}
%   We have that $|D_2| < 3\gamma + 7\e\gamma$.
% \end{lemma}
% \begin{proof}
%   First, by \cref{lem:size-D22}, we have $|D_2^2|< 3\gamma+\e\gamma$. Then, with
%   \cref{lem:size-D21}, we have $|D_2^1|< 6\e\gamma$. Finally, with
%   \cref{lem:size-D23} we conclude that $|D_2|<3\gamma + 7\e\gamma$
% \end{proof}
%
% \begin{lemma}\label{lem:size-D1}
%   We have that $|D_1| < 3\gamma + \e\gamma$.
% \end{lemma}

% \alex{replacement for Lemmas 7 and 8 below}

\smallskip
\begin{lemma}\label{lem:size-D12}
  We have $|D_1\cup D_2| < \rho(G)\gamma+2\nabla\eta\gamma$.
\end{lemma}
\begin{proof}
  By \cref{lem:size-D23} we have $D_2^3\subseteq D_2^1$, hence,
  $D_1\cup D_2=D_1\cup D_2^1\cup D_2^2$. By \cref{lem:size-D1} we have
  $D_1 \subseteq \hat D$ and by \cref{lem:size-D22} we also have
  $D_2^2 \subseteq \hat D$, hence $D_1\cup D_2^2\subseteq \hat D$.
  Again by \cref{lenzen-improved}, $|\hat D \setminus D|<\rho(G)\gamma$ and
  therefore $|(D_1 \cup D_2^2 )\setminus D| < \rho(G) \gamma$.

  We have $W\cap D\subseteq D_2^1\cap D$, hence with
  \cref{lem:dominating-dominators} we conclude that
  \[
    \big\vert D_2^1 \setminus D \big\vert \leq
    \Big\vert\bigcup\limits_{v\in D \cap D_2^1}B_v\Big\vert \leq
    \sum\limits_{v\in D \cap D_2^1} |B_v| \leq (2\nabla-1)\rho\gamma,
  \]
  hence $(D_1\cup D_2)\setminus D<\rho(G)\gamma+(2\nabla-1)\eta\gamma$. Finally,
  $D_1\cup D_2=(D_1\cup D_2)\setminus D\cup ((D_1\cup D_2)\cap D)$ and
  with the definition of $\eta$ we conclude
  $|D_1\cup D_2|<\rho(G)\gamma + 2\nabla\eta\gamma$.
\end{proof}
%The analysis of the next and final step of the algorithm will actually
%show that the worst case is obtained when $\eta=0$.

We now update the residual degrees, that is, we update $R$ as
$V(G)\setminus N[D_1\cup D_2]$ and for every vertex the number
$\dr(v)=|\Nr(v)|$ accordingly.


Just as before, we show that after the first two phases of the algorithm we
are in the very nice situation where all residual degrees are
small. 

\begin{lemma}\label{lem:res-degree}
  For all $v\in V(G)$ we have $\dr(v)\leq (2\nabla-1)^2t+(2\nabla-1)$.
\end{lemma}
\begin{proof}
  First, every vertex of $D_1\cup D_2$ has residual degree $0$.
  Assume that there is a vertex $v$ of residual degree at least $(2\nabla-1)^2t+(2\nabla-1)+1$.
  As $v$ is not in $D_1$, its residual neighbors are
  dominated by a set $A_v$ of at most $(2\nabla-1)$ vertices. Hence there is a
  vertex~$z$ (not in $D_1$ nor $D_2$) with $|\Nr(v)\cap \Nr[z]|\geq (2\nabla-1)t+2 = ((2\nabla-1)t+1)+1$, hence, $|\Nr(v)\cap \Nr(z)|\geq (2\nabla-1)t+1$.
  This contradicts that $v$ is not in~$D_2$.
\end{proof}
