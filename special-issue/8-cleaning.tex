% !TEX root = main.tex

\section{Alternative Phase 3: Greedy domination}\label{sec:greedy-be}

We now consider an alternative approach for the third phase, which does
not improve the approximation factor, however, is conceptually much
simpler and interesting in its own. Recall that we bounded
$\Dr$ as $\kappa^{s-1}(t+s-1+(s-1)\nu)$, which is a bound on the residual
degree $\dr(v)$ of all vertices.

\newcommand{\dR}{\Dr}
\newcommand{\ddR}{\dr}
\newcommand{\pick}{P}

We simulate the classical greedy algorithm, which in each
round selects a vertex of maximum residual degree. Here, we let all
non-dominated vertices that have a neighbor of maximum residual degree
choose such a neighbor as its dominator (or if they have maximum
residual degree themselves, they may choose themselves). In general
this is not possible for a LOCAL algorithm, however, as we established
a bound on the maximum degree we can proceed as follows. We
let~$i={\dR}$. Every red vertex that has at least one neighbor of residual
degree ${\dR}$ arbitrarily picks one of them and elects it to the
dominating set. Then every vertex recomputes its residual degree and
$i$ is set to ${\dR}-1$. We continue until $i$ reaches $0$ when all
vertices are dominated. More formally, we define several sets as
follows.

\smallskip
\begin{tcolorbox}[colback=red!5!white,colframe=red!50!black]
  For ${\dR}\geq i\geq 0$,  for every $v\in R$ in parallel:\\[2mm]
  If there is some $u$ with ${\ddR}(u)=i$ and ($\{u,v\}\in E(G)$ or $u=v$), then\\
  \mbox{ } $\dom_i(v)\leftarrow \{u\}$ (pick one such $u$ arbitrarily),\\
  \mbox{ } $\dom_i(v)\leftarrow \emptyset$ otherwise.
  \begin{itemize}
    \item $R_i \leftarrow R$ \hfill \textit{\small What currently remains to be dominated}
    \item $\pick_i \leftarrow \bigcup\limits_{v\in R} \dom_i(v)$ \hfill \textit{\small What we pick in this step}
    \item $R \leftarrow R \setminus N[\pick_{i}]$ \hfill \textit{\small Update red vertices}
  \end{itemize}
  Last, $D_3\leftarrow  \bigcup\limits_{0\le i\le d} \pick_i$.
\end{tcolorbox}

\smallskip
Let us first prove that the algorithm in fact computes a dominating set.
\begin{lemma}\label{lem:correctness-general}
  When the algorithm has finished the iteration with parameter
  $i\geq 1$, then all vertices have residual degree at most $i-1$.
\end{lemma}

In particular, after finishing the iteration with parameter $1$, there
is no vertex with residual degree $1$ left and in the final round all
non-dominated vertices choose themselves into the dominating
set. Hence, the algorithm computes a dominating set of $G$.

\begin{proof}
  By induction, before the iteration with parameter $i$, all vertices
  have residual at most $i$. Assume $v$ has residual degree $i$ before
  the iteration with parameter $i$.  In that iteration, all
  non-dominated neighbors of $v$ choose a dominator (possibly $v$, then
  the statement is trivial),
  hence, are removed from $R$. It follows that the residual degree of $v$ after
  the iteration is $0$. Hence, after this iteration and before the
  iteration with parameter $i-1$, we are left with vertices of
  residual degree at most $i-1$.
\end{proof}

For the rest of this section analyze the size of $D_3$ and
we prove the following lemma.

\begin{lemma}\label{lem:greedy-approx}
 We have
 \[
 |D_3|\leq \left(\nabla_0\ln\Bigl(\frac{2{\dR}-4\nabla_0+1}{2\nabla_0+1}\Bigr)+3\nabla_0+1\right)\gamma_R.
 \]
\end{lemma}

Towards establishing the lemma we analyze the sizes of the sets
$\pick_i$ and $R_i$. The next
lemma follows from the fact that every vertex chooses at most one
dominator.

\begin{lemma}\label{lem:total-h}
  For every $0\leq i\le {\dR}$, $\sum\limits_{j\le i}|\pick_j| \le |R_i|$.
\end{lemma}
\begin{proof}
  The vertices of $R_i$ are those that remain to be dominated in the
  last $i$ rounds of the algorithm. As every vertex that remains to be
  dominated chooses at most one dominator in one of the rounds
  $j\leq i$, the statement follows.
\end{proof}

As the vertices of $D_R$ that still dominate non-dominated vertices also
have bounded residual degree, we can conclude that not too many
vertices remain to be dominated.
\begin{lemma}\label{lem:h1}
  For every $0\leq i\le {\dR}$, $|R_i| \le (i+1)\gamma_R$.
\end{lemma}
\begin{proof}
  First note that for every $i$,
  $D_R\setminus \bigcup_{j>i}\pick_j$ is a dominating
  set for $R_i$; additionally each vertex in this set has residual
  degree at most $i$. As every vertex dominates its residual neighbors and
  itself, we conclude $|R_i|\le (i+1)\gamma_R$.
\end{proof}

Finally, we show that we cannot pick too many vertices of high
residual degree. This follows from a simply density argument.
% bounded edge density.

\begin{lemma}\label{lem:h2}
  For every $2\nabla_0< i\leq {\dR}$, $|\pick_i|\leq \frac{\nabla_0}{i-2\nabla_0}(|R_i|-|R_{i-1}|)$.
\end{lemma}

\begin{proof}
Let $2\nabla_0< i\le {\dR}$ be an integer. We bound the size of $\pick_i$
  by a counting argument, using that $G$ (as well as each of its
  subgraphs) have edge density at most $\nabla_0$.

  Let $J := G[\pick_i]$ be the subgraph of $G$ induced by the
  vertices of $\pick_i$, which all have residual degree~$i$. Let
  $K := G[\pick_i \cup (N[\pick_i]\cap R_i)]$ be the subgraph of $G$
  induced by the vertices of $\pick_i$ together with the red
  neighbors that these vertices dominate.

  We have $|E(J)| \leq  \nabla_0|V(J)| = \nabla_0|\pick_i|$. As every
  vertex of $J$ has residual degree exactly $i$, we get
  $|E(K)| \geq i\pick_i - |E(J)| \geq (i-\nabla_0)|\pick_i|$ (we have to
  subtract $|E(J)|$ to not count twice the edges of $K$ that are
  between two vertices of $J$).  We also have
  $|V(K)| \le |V(J)| + |N[\pick_i]\cap R_i|$ and $|E(K)| \leq \nabla_0|V_K|$, hence
  $(i-2\nabla_0)|\pick_i| \leq \nabla_0|N[\pick_i]\cap R_i)|$.
  Now, as $R_{i-1} = R_i \setminus N[\pick_{i}]$, that is,
  $N[\pick_i]\cap R_i=R_i\setminus R_{i-1}$, we get
  $|\pick_i|\leq \frac{\nabla_0}{i-2\nabla_0}(|R_i|-|R_{i-1}|)$.
\end{proof}

% \pagebreak
Let $r_i\coloneqq |R_i|/\gamma_R$ and $d_i\coloneqq |\pick_i|/\gamma_R$.
Our goal is to maximize $S\coloneqq\sum_{0\leq i\leq {\dR}}d_i$ (which we have
to multiply by $\gamma_R$ in the end) under the constraints $d_i\geq 0$
and
\begin{align}
	r_i &\ge \sum_{j\le i} d_j&\text{($0\le i\le {\dR}$)}\label{eq:1}\\
	r_i &\le	i+1&\text{($0\le i\le {\dR}$)}\label{eq:2}\\
%	d_i&\leq \frac{\nabla_0}{i-2\nabla_0}r_{i}&\text{($2\nabla_0 +1\leq i\leq {\dR}$)}\label{eq:3}\\
	d_{i}&\leq \frac{\nabla_0 }{i-2\nabla_0 }\,(r_{i}-r_{i-1})&\text{($2\nabla_0 < i\le {\dR}$)}\label{eq:4}
\end{align}

%\sebi{rename $D$, this is reserved for the dominating set.}
%In the following for readability we write ${\dR}$ for ${\dR}$.
We may assume that ${\dR}\geq 3\nabla_0$, as otherwise,
\cref{lem:greedy-approx} follows immediately from \cref{lem:total-h}.

\smallskip
Let $a$ be the minimum integer such that $d_a>0$.

\begin{lemma}
	We can assume $r_i=0$ for all $i<a$.
\end{lemma}
\begin{proof}
	Putting $r_i=0$ for all $i<a$ obviously preserves \cref{eq:1,eq:2}. It also preserves \cref{eq:4} as the only case
	to check is $i=a-1$ (if $a\ge 2\nabla_0 $), for which the right hand side was possibly increased.
\end{proof}

\begin{lemma}
	If $a\le 3\nabla_0 -1$, then decreasing $d_a$ to $0$ and $r_a$ to $r_a-d_a$ and increasing $d_{a+1}$ to $d_a+d_{a+1}$ preserves  all the constraints and the value of $S$.
\end{lemma}
\begin{proof}
	The sum in \cref{eq:1} does not change if $i>a$ and \cref{eq:1} is obviously satisfied after modifications for $i\le a$.
	\cref{eq:2} is trivially satisfied after modification, as no $r_i$ increases.
	The only changes for \cref{eq:4} correspond to the case $i=a-1$ (for which the left hand side decreases,  while the right hand side increases) or to the case $i=a$ (for which the left hand side increases by $d_a$,  while the right hand side increases by $\nabla_0 /(a+1-2\nabla_0 )\,d_a\ge d_a$).
\end{proof}

From the above lemmas, as $r_a\geq d_a$, it follows that we may assume $a\ge 3\nabla_0 $ and $r_i=0$ for all $i<a$.

%Note that for $i=a-1$, \cref{eq:4} follows from \cref{eq:1}. Hence, if we modify the values, we only have to check \cref{eq:4} for $i\geq a$.

Note that \cref{eq:4} implies
\begin{equation}
	r_{2\nabla_0 }\le r_{2\nabla_0 +1}\le \dots\le r_{\dR}.
\end{equation}

Remark that increasing $r_{\dR}$ obviously preserves \cref{eq:1,eq:4}. Hence, we can assume that $r_{\dR}={\dR}+1$.
Let $b$ be minimum with $r_{i}=i+1$ for all $i\geq b$. Note that $b\ge a$.


\begin{lemma}
	Let $\alpha=\min(b-r_{b-1},\sum_{j<b-1}d_j)$.
If $b\ge 3\nabla_0 +1$, then increasing  $d_{b-1}$ and $r_{b-1}$ by $\alpha$ and decreasing
$\sum_{j<b-1}$ by $\alpha$ preserves the constraints and the value of $S$.
\end{lemma}
\begin{proof}
	\cref{eq:1,eq:2} are obviously preserved. For \cref{eq:4} we have to check the case where $i=b-1$ (for which the right hand side decreases by $\nabla_0 /(b-1-2\nabla_0 )\,\alpha\le\alpha$ and the left hand side decreases by $\alpha$) and the case $i=b-2$  (for which the right hand side increases and the left hand side decreases).
\end{proof}
Applying this lemma, either we can reduce $b$ to $3\nabla_0 $ (hence $b=a$), or we force $d_i=0$ for all $i<b-1$. Thus, $a=b-1$ or $a=b$.

\begin{lemma}
	We can assume that for every $b< i\le {\dR}$ we have  $d_i=\nabla_0 /(i-2\nabla_0 )$.
\end{lemma}
\begin{proof}
Indeed, as $b\ge a\ge 3\nabla_0 $, for $b< i\le {\dR}$,  \cref{eq:1,eq:2,eq:4} reduce to
$d_{i}\le \frac{\nabla_0 }{i-2\nabla_0 }$. Hence, we can assume $d_i=\nabla_0 /(i-2\nabla_0 )$ if $i> b$.
\end{proof}

\begin{lemma}
	We can assume $b=a$.
\end{lemma}
\begin{proof}
	Assume $a=b-1$ and let $\alpha=b-r_a$.
	We have $d_a\le\frac{\nabla_0 }{a-2\nabla_0 } (b-\alpha)$ and $d_b\le \frac{\nabla_0 }{b-2\nabla_0 } (1+\alpha)$.
	If we increase $r_a$ to $b$, we can increase $d_a$ by $\frac{\nabla_0 \alpha}{a-2\nabla_0 }$ and decrease
	$d_b$ by  $\frac{\nabla_0 \alpha}{b-2\nabla_0 }$, while preserving the constraints and increasing $S$.
\end{proof}

\begin{lemma}
	We can assume $a=3\nabla_0 $.
\end{lemma}
\begin{proof}
Assume $a\ge 3\nabla_0 +1$. By putting $r_{a-1}=a$ we can increase $d_{a-1}$ by $\frac{\nabla_0 }{a-1-2\nabla_0 }a$ and decrease~$d_a$ by  $\frac{\nabla_0 }{a-2\nabla_0 }a$. Note that the condition $a\ge 3\nabla_0 +1$ implies that
$\frac{\nabla_0 }{a-1-2\nabla_0 }\le 1$, which is needed to preserve \cref{eq:1}.
\end{proof}

Now we have $a=b=3\nabla_0 $ and we can put $d_a=a+1$.
Hence,  the optimum is

$d_i=0$ if $i<3\nabla_0 $, $d_{3\nabla_0 }=3\nabla_0 +1$ and $d_{3\nabla_0 +i}=\nabla_0 /(\nabla_0 +i)$.
Altogether, we get
\[
S=3\nabla_0 +1+\nabla_0 \sum_{i=\nabla_0 +1}^{{\dR}-2\nabla_0 }\frac{1}{i}=\nabla_0 (H_{{\dR}-2\nabla_0 }-H_{\nabla_0} )+3\nabla_0 +1,
\]
where $H_i=1+1/2+\dots+1/i$ is the $i$th harmonic number.

It is known \cite{DeTemple1991} that
\[
\frac{1}{24(n+1)^2}<H_n-\ln \Bigl(n+\frac12\Bigl)-\gamma<\frac{1}{24n^2},
\]
where $\gamma$ is the Euler--Mascheroni constant.
We deduce that for $n>m$ we have
\[
-\frac{1}{24m^2}<\frac{1}{24}\Bigl(\frac{1}{(n+1)^2}-\frac{1}{m^2}\Bigr)<(H_n-H_m)-\ln \Bigl(\frac{2n+1}{2m+1}\Bigr)<\frac{1}{24}\Bigl(\frac{1}{n^2}-\frac{1}{(m+1)^2}\Bigr)\le 0
\]

 we deduce
 \[
 -\frac{1}{24\nabla_0 ^2}<S-\biggl(
 \nabla_0 \ln\Bigl(\frac{2{\dR}-4\nabla_0 +1}{2\nabla_0 +1}\Bigr)+3\nabla_0 +1\biggr)<0.
 \]



 Hence, with a (negative) error less than $0.042$ we have
 \[
 S\approx \nabla_0 \ln\Bigl(\frac{2{\dR}-4\nabla_0 +1}{2\nabla_0 +1}\Bigr)+3\nabla_0 +1.
 \]

 This finishes the proof of \cref{lem:greedy-approx}. \hfill$\qed$

% \bigskip
%
%Assume $\nabla_0>1$.
%Choose $s=\nabla_0+1$, $t=\nabla_0(\nabla_0+1)$.
%Then, $2\nabla_0+(\nabla_0+1)^2\le (2\nabla_0)^2\le \kappa^2$.
%Thus,
%
%\[
%{\dR}=\kappa^{s-1}(t+s-1+2(s-1)\kappa^3)=
%\kappa^{\nabla_0+3}(2\nabla_0+1)\le 2\kappa^{\nabla_0+4}
%\]
%Hence,
%\[
%\ln((2{\dR}-4\nabla_0 +1)/(2\nabla_0 +1))<\ln {\dR}\le (\nabla_0+4)\ln\kappa+1.
%\]
%
%From which follows
%\[
%|D_3|<\bigl((\nabla_0+4)\ln(2\nabla_1)+1\bigr)\gamma_R.
%\]
%
%Note that if we require $\nabla_0>\nabla_0(G)$, then we have
%\[
%|D_3|<\bigl(\nabla_0(\nabla_0+3)\ln(2\nabla_1)+4\nabla_0+1\bigr)\gamma_R.
%\]

%With the general bound for ${\dR}$, we get (for $\nabla_1\ge 1$)
%\[
%{\dR}<(2\nabla_1)^{2\nabla_1}(4\nabla_1+1+128\nabla_1^5)<(2\nabla_1)^{2\nabla_1}(e^5\nabla_1^5).
%\]
%Hence,
%\[
%\ln((2{\dR}-4\nabla_0 +1)/(2\nabla_0 +1))<\ln {\dR}<(2\nabla_1)\ln(2\nabla_1)+5\nabla_1+5.
%\]
%Thus, as $\nabla_0 \le\nabla_1$ we have
%\[
%S< 2\nabla_1^2\ln (2\nabla_1)+5\nabla_1^2+8\nabla_1+1
%\]
%In particular, for $K_{t,t}$-free graphs, $S=O((t\ln t)^2)$.
