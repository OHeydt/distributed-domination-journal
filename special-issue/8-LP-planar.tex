% !TEX root = main.tex

\subsection{Phase 3: LP-based approximation}\label{sec:LP-planar}

\sebi{Update the general case such that we can work
with arbitrary $\epsilon$.} 

Recall that we chose $\rho$ such that $|(D_1 \cup D_2)\cap D| =\rho\gamma$.
When $\rho\gamma$ vertices of $D$ were already chosen into
the partial dominating set $D_1\cup D_2$ we have $|D_R|\leq (1-\rho)\gamma$.
With \cref{cor:planar-orientations} we conclude the following corollary.

\sebi{Put this to the general case.}

\begin{corollary}\label{cor:LP-approx}
  Let $G$ be a graph that has an orientation with maximum out-degree
  $d$, let $R\subseteq V(G)$, let~$D_R$ be a
  minimum dominating set of $R$, and let $\e>0$. Then we can
  compute a set $D'$ of size at most~$(2d+1)(1+\e)|D_R|$ that dominates
  $R$ in $\Oof(\log \Delta/(\log (1+\e))$ rounds in the LOCAL
  model.

  In particular, for our algorithm when

  \vspace{-2mm}
  \begin{enumerate}
    \item $G$ is planar, then $|D'|\leq 7(1+\e)|D_R|
      \leq 7(1+\e)(1-\rho)\gamma$, and when
    \item $G$ is planar and triangle-free or outerplanar, then
      $|D'|\leq 5(1+\e)|D_R|\leq 5(1+\e)(1-\rho)\gamma$.
  \end{enumerate}
\end{corollary}

\subsubsection{Conclusion of LP approach for $K_{3,t}$-free graphs}

\sebi{Stopped here.}

We now proceed with the LP-based approximation as in the general case
presented in \cref{sec:LP}. Recall that for any desired $\epsilon>0$ 
we defined $\Gamma$ as an high degree and defined 
$D_3^1$ as the set of all vertices with degree greater than $\Gamma$. 
By the choice of $\Gamma$ we concluded with \cref{lem:size-D31}
that $|D_3^1|\leq \epsilon\gamma$. 

We now pick an arbitrary $\e>0$ and select all vertices with degree at least
$\Gamma$ into our
dominating set, where $\Gamma$ is chosen such that there exist at
most $(\e/8)(1-\rho)\gamma$ vertices of degree at least $\Gamma$.

\begin{tcolorbox}[colback=red!5!white,colframe=red!50!black]
  Let $\e>0$, $\e':= \e/8$,
  $\Gamma\coloneqq 6\frac{372}{\e'}$, and \quad
  $D_3^1\coloneqq \{v\in R ~:~ d(v)>\Gamma\}$.
\end{tcolorbox}

\begin{lemma}\label{lem:size-D31}
  $|D_3^1|\le \e'(1-\rho)\gamma$.
\end{lemma}
\begin{proof}
  We assume the opposite and count the number of edges and vertices of $G$.
  When we sum the degree of some of the vertices, we get at most twice the
  number of edges.
  Hence we have~$2\cdot |E(G)|\ge 6\frac{372}{\e'} \cdot \e'(1-\rho)\gamma$.
  Hence, with \cref{cor:size-g},
  $|E(G)|\ge 3\cdot 372(1-\rho)\gamma\geq 3|V(G)|$. This
  contradict the fact the graph is planar.
\end{proof}

After picking $D_3^1$ into the dominating set, marking the neighbors of
$D_3^1$ as dominated and updating the set $R$, we can delete the
vertices of $D_3^1$. We are left with a graph of maximum degree~$\Gamma$.

\begin{tcolorbox}[colback=red!5!white,colframe=red!50!black]
  Let $D_3^2$ be the set computed by the LOCAL algorithm of \cref{cor:LP-approx}
  with parameter $\e'$.
\end{tcolorbox}

Let $D_3=D_3^1 \cup D_3^2$. We already noted that the definition of $D_3$ implies that
$D_1\cup D_2\cup D_3$ is a dominating set of $G$. We now conclude the
analysis of the size of this computed set.

\begin{lemma}\label{lem:D3-LP}
  We have that $|D_3| \le (7+\e)(1-\rho)\gamma$
\end{lemma}
\begin{proof}
  By \cref{cor:LP-approx} and \cref{lem:size-D31} we have
  $|D_3^2|\leq 7(1+\e')(1-\rho)\gamma$, and $|D_3^1|\le \e'(1-\rho)\gamma$.
  This, together with~$\e'=\e/8$, concludes the proof.
\end{proof}

\subsection{Corollaries for graphs on surfaces}

\sebi{Update}

We can now complete the analysis of our algorithm for planar graphs.
\begin{theorem}\label{thm:planar-ccl}
  There exists a distributed LOCAL algorithm that, for every $\e>0$, and every
  planar graph~$G$, computes in a constant number of rounds a dominating set
  of size at most $(11+\e)\gamma(G)$.
\end{theorem}
\begin{proof}
  First, $D_1,D_2$, and $D_3$ are computed locally, in a bounded number of
  rounds, and additionally the set $D_1 \cup D_2 \cup D_3$ dominates $G$.
%
  Second, by \cref{lem:size-D12} we have $|D_1\cup D_2|<4\gamma+4\rho\gamma$;
  and by \cref{lem:D3-LP} we have $|D_3|\leq (7+\e)(1-\rho)\gamma$.

  Therefore $|D_1 \cup D_2 \cup D_3|\leq \gamma(4+4\rho +7 - 7\rho +\e - \e\rho)
  \leq \gamma(11+\e-3\rho -\e\rho)$.
  As $\rho\in[0,1]$, this is maximized when $\rho=0$. Hence
  $|D_1 \cup D_2 \cup D_3| \le \gamma(11+\e)$.
\end{proof}
