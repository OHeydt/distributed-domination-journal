
\documentclass[a4paper,UKenglish,thm-restate,numberwithinsect]{lipics-v2021}
%This is a template for producing LIPIcs articles.
%See lipics-v2021-authors-guidelines.pdf for further information.
%for A4 paper format use option "a4paper", for US-letter use option "letterpaper"
%for british hyphenation rules use option "UKenglish", for american hyphenation rules use option "USenglish"
%for section-numbered lemmas etc., use "numberwithinsect"
%for enabling cleveref support, use "cleveref"
%for enabling autoref support, use "autoref"
%for anonymousing the authors (e.g. for double-blind review), add "anonymous"
%for enabling thm-restate support, use "thm-restate"
%for enabling a two-column layout for the author/affilation part (only applicable for > 6 authors), use "authorcolumns"
%for producing a PDF according the PDF/A standard, add "pdfa"

%\graphicspath{{./graphics/}}%helpful if your graphic files are in another directory


\bibliographystyle{plainurl}% the mandatory bibstyle

\title{Brief announcement: Local domination on planar graphs revisited} %TODO Please add

\titlerunning{} %TODO optional, please use if title is longer than one line

%\author{Simeon Kublenz}{University of Bremen, Germany}{kublenz@uni-bremen.de}{}{}%TODO mandatory, please use full name; only 1 author per \author macro; first two parameters are mandatory, other parameters can be empty. Please provide at least the name of the affiliation and the country. The full address is optional

\author{Sebastian Siebertz}{University of Bremen, Germany}{siebertz@uni-bremen.de}{https://orcid.org/0000-0002-6347-1198}{}

\author{Alexandre Vigny}{University of Bremen, Germany}{vigny@uni-bremen.de}{https://orcid.org/0000-0002-4298-8876}{}

\author{Ozan Heydt}{University of Bremen, Germany}{heydt@uni-bremen.de}{}{}

\authorrunning{S. Siebertz, A. Vigny and O. Heydt} %TODO mandatory. First: Use abbreviated first/middle names. Second (only in severe cases): Use first author plus 'et al.'

\Copyright{?} %TODO mandatory, please use full first names. LIPIcs license is "CC-BY";  http://creativecommons.org/licenses/by/3.0/

\ccsdesc[500]{\textcolor{red}{Theory of computation~Self-organization}} %TODO mandatory: Please choose ACM 2012 classifications from https://dl.acm.org/ccs/ccs_flat.cfm

\keywords{Dominating set, LOCAL algorithm, planar graph} %TODO mandatory; please add comma-separated list of keywords

\category{} %optional, e.g. invited paper

\relatedversion{} %optional, e.g. full version hosted on arXiv, HAL, or other respository/website
%\relatedversiondetails[linktext={opt. text shown instead of the URL}, cite=DBLP:books/mk/GrayR93]{Classification (e.g. Full Version, Extended Version, Previous Version}{URL to related version} %linktext and cite are optional

%\supplement{}%optional, e.g. related research data, source code, ... hosted on a repository like zenodo, figshare, GitHub, ...
%\supplementdetails[linktext={opt. text shown instead of the URL}, cite=DBLP:books/mk/GrayR93, subcategory={Description, Subcategory}, swhid={Software Heritage Identifier}]{General Classification (e.g. Software, Dataset, Model, ...)}{URL to related version} %linktext, cite, and subcategory are optional

%\funding{(Optional) general funding statement \dots}%optional, to capture a funding statement, which applies to all authors. Please enter author specific funding statements as fifth argument of the \author macro.

\acknowledgements{}%optional

%\nolinenumbers %uncomment to disable line numbering

\hideLIPIcs  %uncomment to remove references to LIPIcs series (logo, DOI, ...), e.g. when preparing a pre-final version to be uploaded to arXiv or another public repository

%Editor-only macros:: begin (do not touch as author)%%%%%%%%%%%%%%%%%%%%%%%%%%%%%%%%%%
\EventEditors{John Q. Open and Joan R. Access}
\EventNoEds{2}
\EventLongTitle{42nd Conference on Very Important Topics (CVIT 2016)}
\EventShortTitle{CVIT 2016}
\EventAcronym{CVIT}
\EventYear{2016}
\EventDate{December 24--27, 2016}
\EventLocation{Little Whinging, United Kingdom}
\EventLogo{}
\SeriesVolume{42}
\ArticleNo{23}
%%%%%%%%%%%%%%%%%%%%%%%%%%%%%%%%%%%%%%%%%%%%%%%%%%%%%%%

% End LIPICS stuff

% -- Added Packages --
\usepackage{mathtools}
\usepackage{tikz}
\definecolor{cadmiumgreen}{rgb}{0.0, 0.42, 0.24}
\definecolor{dark-blue}{rgb}{0.05,0.25,1}
\usepackage{tcolorbox}
\usetikzlibrary{patterns,arrows,decorations.pathreplacing}
\usepackage{url}
\usepackage{hyperref}
\hypersetup{
    colorlinks = true,
    linkcolor = blue,
    citecolor=blue
    }
\usepackage[noabbrev,capitalise,nameinlink]{cleveref}
\newcommand\claimref[2]{\hyperref[{#1}]{#2}}

\newcommand{\nc}[1]{\newcommand{#1}}
\newcommand{\Oof}{\mathcal{O}}
\newcommand{\Cc}{\mathscr{C}}
\newcommand{\Aa}{\mathcal{A}}
\newcommand{\Bb}{\mathcal{B}}
\newcommand{\Tt}{\mathcal{T}}
\newcommand{\Nn}{\mathcal{N}}
\newcommand{\Ss}{\mathcal{S}}
\newcommand{\Pp}{\mathcal{P}}
\newcommand{\Qq}{\mathcal{Q}}
\newcommand{\Rr}{\mathcal{R}}
\newcommand{\N}{\mathbb{N}}
\newcommand{\minor}{\preceq}
\newcommand{\red}[1]{\textcolor{red}{#1}}
\nc{\e}{\epsilon}
\nc{\dd}{\delta}

% -- End of added Packages --

% % % % % % % % % % % % % % % % % % % % % % % % % % % % % % % % % % % % % % % %
% % % % % % % % % % % % % % For commenting the text % % % % % % % % % % % % % %
\newcommand{\commentmargin}[1]{\marginpar{\tiny\textit{#1}}}
\newcommand{\commenttext}[1]{ \begin{center} {\fbox{\begin{minipage}[h]{0.9 \linewidth}   {\sf #1} \end{minipage} }} \end{center}}

% Add your name and pick a color !
\newcommand{\alex}[1]{{\color{blue}\commenttext{Alex: #1}}}
\newcommand{\alexpb}[1]{{\color{purple}\commenttext{ATTENTION (Alex): #1}}}
\newcommand{\alexmargin}[1]{\commentmargin{\color{blue}Alex: #1}}



\newcommand{\sebi}[1]{{\color{red}\commenttext{Sebi: #1}}}
\newcommand{\sebimargin}[1]{\commentmargin{\color{red}Sebi: #1}}

\newcommand{\nicole}[1]{{\color{green!60!black}\commenttext{Nicole: #1}}}
\newcommand{\nicolemargin}[1]{\commentmargin{\color{green!60!black}Nicole: #1}}

%\newcommand{\red}[1]{{\color{red}#1}}
%%% uncomment what follows to remove the comments %%%

% \renewcommand{\commentmargin}[1]{}
% \renewcommand{\commenttext}[1]{}
% \renewcommand{\red}[1]{}

% % % % % % % % % % % % % % End commenting the text % % % % % % % % % % % % % %
% % % % % % % % % % % % % % % % % % % % % % % % % % % % % % % % % % % % % % % %



\begin{document}

\maketitle

%TODO mandatory: add short abstract of the document
\begin{abstract}
A dominating set in a graph $G$ is a subset $D\subseteq V(G)$ of vertices
such that all vertices of $G$ are in or adjacent to a vertex of $D$.
While it is known that in general no constant factor approximations
can be computed in a constant number of rounds in the LOCAL model,
this is possible in several restricted graph classes. In particular,
an elegant an simple algorithm of Lenzen et al.\ computes a
52-approximation in a constant number of rounds in planar graphs.
We combine the preprocessing phase of Lenzen et al.'s algorithm
with recent work of Czygrinow et al.\ and Kublenz et al.\ to obtain a
new approach to compute small dominating sets in planar graphs
in a constant number of rounds. While the algorithm falls short
of achieving the best known approximation it gives interesting
new insights into LOCAL domination on planar graphs.
\end{abstract}

\section{Introduction}
A dominating set in an undirected and simple graph $G$ is a set
$D\subseteq V(G)$ such that every vertex $v\in V(G)$ either belongs
to $D$ or has a neighbor in $D$. The dominating set problem is a
classical NP-complete problem~\cite{karp1972reducibility} with many
applications in theory and practice. In this paper we study
the distributed time complexity of finding
dominating sets in the classic LOCAL model of distributed computing.
The problem of approximating
dominating sets in the LOCAL model has received considerable
attention in the literature~\cite{akhoondian2018distributed,
akhoondian2016local,
alipour2020local,
amiri2016brief,
amiri2019distributed,
barenboim2018fast,
czygrinow2008fast,
czygrinow2018distributed,
DBLP:conf/stoc/GhaffariKM17,
hilke2014brief,
kublenz2020distributed,
KuhnMW16,
lenzen2013distributed,
lenzen2008leveraging,
lenzen2010minimum,
DBLP:conf/stoc/RozhonG20,
wawrzyniak2013brief,
wawrzyniak2014strengthened}. Since in general graphs
it is not possible to compute a
constant factor approximation in a constant number of rounds~\cite{KuhnMW16},
much effort has been invested to improve the ratio between approximation
factor and number of rounds on special graph classes. In particular, a
line of structural analysis of graph properties that can lead to improved
algorithms was started by the influential paper of Lenzen et al.~\cite{lenzen2013distributed}, who in particular proved that on planar graphs
a 130-approximation
of a minimum dominating set can be computed in a constant number of
rounds. A careful analysis of Wawrzyniak~\cite{wawrzyniak2014strengthened}
later showed that the algorithm computes in fact a 52-approximation.
In terms of lower bounds, Hilke et al.~\cite{hilke2014brief} showed that there is no
deterministic local algorithm (constant-time distributed graph algorithm) that
finds a~$(7-\epsilon)$-approximation of a minimum dominating set on
planar graphs, for any positive constant~$\epsilon$.

The algorithm of Lenzen et al.~\cite{lenzen2013distributed} works in two
phases. The first phase is a preprocessing step that leaves us with
only vertices whose neighborhoods can be dominated by a few other
vertices. In the second phase it is then sufficient that every remaining
non-dominated vertex chooses a neighbor of maximum residual
degree\footnote{The residual degree of a vertex is the number of
its non-dominated neighbors.}.

Recently, Czygrinow et  al.~\cite{czygrinow2018distributed} proposed
a new algorithm that works for much more general graphs, namely
for graphs that exclude some topological minor. This gain in
applicability to more general graph classes however, comes at the
cost of a much more complicated algorithm. By defining a notion
of \emph{pseudo-covers}, Czygrinow et
al.~\cite{czygrinow2018distributed} provided tool to carry out a
fine grained analysis of vertices that can potentially dominate
the neighborhoods of vertices in graphs that exclude some topological
minor. A sophisticated iterative procedure then leads to the
computation of a dominating set that is only constantly much
larger than an optimum dominating set. Kublenz et al.~\cite{kublenz2020distributed} generalized the algorithm
to classes with bounded expansion and optimized constants,
however, the constants arising in the construction are still
astronomically high.

In this paper we combine the preprocessing approach of
Lenzen et al.~\cite{lenzen2013distributed} with the
construction of Czygrinow et al.~\cite{czygrinow2018distributed}
on planar graphs. As planar graphs are much simpler than
graphs that exclude a topological minor we are able to carry
out the construction of Czygrinow et al.\ without the
complicated construction of pseudo-covers by applying
only very basic combinatorial tools. After the selection of
a small number of initial dominating vertices we are quickly
left with a graph of bounded maximum degree. A parallel
greedy procedure then leads us to the final dominating set.

While we do not achieve the best known approximation factor
of 52, we believe that we gain interesting new insights into
the LOCAL domination properties of planar graphs. In particular,
it is perceivable that a combination with the tools of \cite{wawrzyniak2014strengthened} can lead to an
even improved approximation factor.

%%%%%%%%%%%%%%%%%%%%%%%%%%%%%%%%%%%%%%%%%%%%%%%%%%%%%%%%%%%%%%%%%%%%%%%%%%%%%%%
%%%%%%%%%%%%%%%%%%%%%%%%%%%%%%%%%%%%%%%%%%%%%%%%%%%%%%%%%%%%%%%%%%%%%%%%%%%%%%%
\section{Preliminaries}

We work with the classic LOCAL model of distributed computing
restricted to planar graphs.
In this model, a distributed system is modeled by an undirected planar
graph~$G$,
in which every vertex represents a computational entity of the network and every edge represents a bidirectional communication channel. The vertices are equipped with unique identifiers.
In a distributed algorithm, initially, the nodes have no knowledge about
the network graph. They must then communicate and coordinate
their actions by passing messages to one another in order to achieve
a common goal, in our case, to compute a dominating set of the
network graph. The complexity of a LOCAL algorithm is the number of communication rounds it needs until it returns its answer.

By the famous theorem of Wagner, planar graphs can be characterized
as those graphs that exclude the complete graph $K_5$ and the
complete bipartite $K_{3,3}$ with partitions of size $3$ as a minor.
A graph~$H$ is a minor of a graph~$G$, written~$H\minor G$, if
there is a set \mbox{$\{G_v :v\in V(H)\}$} of pairwise vertex disjoint and
connected subgraphs
$G_v\subseteq G$ such that if~$\{u,v\}\in E(H)$, then there is an edge
between a vertex of~$G_u$ and a vertex of~$G_v$. We call~$V(G_v)$ the
\emph{branch set} of $v$ and say that it is
\emph{contracted} to the vertex~$v$. By Euler's formula, planar graphs
are sparse: every planar $n$-vertex graph ($n\geq 3$) has
at most $3n-6$ edges. Minors of planar graphs are again planar graphs,
so that this formula applies also to all minors of a planar graph. In
the following we assume that each component of our input graph $G$
has at least $3$ vertices. This is justified by the observation that a
minimum dominating set of a disconnected graph is a union of
minimum dominating sets for each component and that we can
compute optimal dominating sets in components of bounded size
in a constant number of rounds.

For a graph $G$ and $v\in V(G)$ we write $N(v)=\{w~:~\{v,w\}\in E(G)\}$
for the \emph{open neighborhood} of $v$ and $N[v]=N(v)\cup\{v\}$ for
the \emph{closed neighborhood} of~$v$. For a set $A\subseteq V(G)$ let
$N[A]=\bigcup_{v\in A}N[v]$.
%We write $N_r[v]$ for the set
%of vertices at distance at most $r$ from a vertex $v$.
A dominating set in a graph~$G$ is a set
$D\subseteq V(G)$ such that $N[D]=V(G)$. We write $\gamma(G)$ for
the size of a minimum dominating set of $G$. For $W\subseteq V(G)$
we say that a set $Z\subseteq V(G)$ \emph{dominates}
 $W$ if $W\subseteq N[Z]$.

\begin{tcolorbox}
  In the following we fix a planar graph $G$ and a minimum dominating
  set $D$ with $\gamma \coloneqq |D|=\gamma(G)$. We assume
  that every component
  of $G$ has at least $3$ vertices.
\end{tcolorbox}

\alex{We need to agree on some color code for the textboxes
  \begin{itemize}
    \item Sets computed by the algorithm (the definition of the sets $D_1$, $D_2$, ...) (RED)
    \item For what is assumed in what follows (not sure that this is needed)
    \item For important definitions (not computed by the algorithm, like $\hat D$) (GREY)
  \end{itemize}
}
%%%%%%%%%%%%%%%%%%%%%%%%%%%%%%%%%%%%%%%%%%%%%%%%%%%%%%%%%%%%%%%%%%%%%%%%%%%%%%%
%%%%%%%%%%%%%%%%%%%%%%%%%%%%%%%%%%%%%%%%%%%%%%%%%%%%%%%%%%%%%%%%%%%%%%%%%%%%%%%
\section{Preprocessing}

\alex{I'm rewriting the paper without saying that we 'remove' vertices and carefully mentioning $D_1$ in he second step, and $D_1\cup D_2$ in the third.}

The algorithm of Lenzen et al.~\cite{lenzen2013distributed} works in two
phases. The first phase is a preprocessing step that leaves us with
only vertices whose neighborhoods can be dominated by a few other
vertices. It is based on the following key combinatorial lemma (which
can be generalized to more general graphs).
We select all vertices whose neighborhood cannot be dominated by $6$ other
vertices.
\begin{tcolorbox}[colback=red!5!white,colframe=red!50!black]
  Let $D_1\coloneqq \{v\in V(G) : $ for all set $A\subseteq V(G)\setminus \{v\}$
  with $N(v)\subseteq N[A]$ we have $|A|> 6\}$.
\end{tcolorbox}

The first phase of the algorithm is to compute this set $D_1$ which can be done
in only 2 rounds of communication.
The following lemma states that there exist only few such vertices.

\begin{lemma}[Lemma 6.3 of Lenzen et al.~\cite{lenzen2013distributed}]\label{lem:lenzen}
  We have $|D_1\setminus D|< 3\gamma$.
\end{lemma}
\alex{add proof to be more self contained? (copy past from sirocco paper)}

The analysis of the next steps of the algorithms require definitions of sets of
vertices that are not computed by the algorithm.
%
First, the neighborhood of the vertices that where not selected can be
dominated by few other vertices.
\begin{tcolorbox}
  For every $v\in V(G)\setminus D_1$, we fix $A_v\subseteq V(G)\setminus \{v\}$
  such that: \center $N(v)\subseteq N[A_v]$ and $|A_v|\leq 6$.
\end{tcolorbox}
There are potentially many such sets $A_v$; we fix one such set arbitrarily.

Second, it is possible to define a set resembling $D_1$ with a stronger
constraint: we can require the sets $A$ to be subsets of the dominating set $D$.
\begin{tcolorbox}
  Let $\hat D\coloneqq \{v\in V(G) : $ for all set $A\subseteq D\setminus \{v\}$
  with $N(v)\subseteq N[A]$ we have $|A|> 6\}$.
\end{tcolorbox}

By following the proof of~\cref{lem:lenzen}, we get similar result for $\hat D$.

% We can also prove by following the lines of the proof of the above
% lemma that almost all vertices $v$ have even better properties,
% namely that we can choose the set~$A_v$ as a
% subset of~$D$ for almost all vertices.
% Note that since $D$ is unknown this set cannot be computed by our algorithm.

\begin{lemma}\label{lem:D-hat}
  We have $|\hat{D}\setminus D|< 3\gamma$, and
  $|\hat{D}|< 4\gamma$.
\end{lemma}

We stress once again that $A_v$ and $\hat D$ are not computed by the algorithm.

%%%%%%%%%%%%%%%%%%%%%%%%%%%%%%%%%%%%%%%%%%%%%%%%%%%%%%%%%%%%%%%%%%%%%%%%%%%%%%%
%%%%%%%%%%%%%%%%%%%%%%%%%%%%%%%%%%%%%%%%%%%%%%%%%%%%%%%%%%%%%%%%%%%%%%%%%%%%%%%
\section{Analyzing the local dominators}

By defining a notion of \emph{pseudo-covers}, Czygrinow et  al.~\cite{czygrinow2018distributed} provided tool to carry out a
fine grained analysis of
vertices that can potentially belong to a set $A$ used to dominate
the neighborhood $N(v)$ of a vertex $v$. This tool can in fact
be applied to much more general graphs than planar graphs, namely,
to all graphs that exclude some complete bipartite graph~$K_{t,t}$.
A refined analysis for classes of bounded expansion was provided
by Kublenz et al.~\cite{kublenz2020distributed}.
We provide an even finer analysis for planar graphs on which we
base a second phase of our distributed algorithm.

We first describe what the algorithm computes, and then provide bounds on the
number of selected vertices. Intuitively, we select every pair of vertices with
sufficiently many neighbors in common.

\begin{tcolorbox}[colback=red!5!white,colframe=red!50!black]
  For $v\in V(G)$ let $B_v\coloneqq \{z\in V(G)\setminus
  \{v\}: |N(v)\cap N(z)|\geq 19\}$.\\
  Then let $W$ be the set of vertices $v\in V(G) \setminus D_1$ such
  that $B_v  \setminus D_1 \neq \emptyset$.\\
  Finally let $D_2\coloneqq \bigcup\limits_{v\in W} (\{v\}\cup B_v)$.
\end{tcolorbox}

Once $D_1$ has been computed in the previous step, 2 more rounds of
communication are enough to compute the sets $B_v$ and $D_2$. We now turn to the
analysis of the sets $B_v$ and $D_2$.
First, $B_v$ cannot be too big, and has nice properties.
\begin{lemma}\label{lem:dominating-dominators}
  For all vertices $v$ of $V(G) \setminus D_1$, we have:
  \begin{itemize}
    \item $B_v \subseteq A_v$ (hence $|B_v|\le 6$), and
    \item if $v\not\in \hat{D}$, then $B_v\subseteq D$.
  \end{itemize}
\end{lemma}

\begin{proof}
  Assume $A_v=\{v_1,\ldots, v_6\}$ and assume there is $z\in V(G)\setminus \{v,v_1,\ldots, v_6\}$
  with $|N(z) \cap N(v)| \geq 19$.
  As $v_1, \ldots, v_6$ dominate $N(v)$ (and in particular $N(v)\cap N(z)$)
  there must be some~$v_i$, $1\leq i\leq 6$, with
  \mbox{$N(z) \cap N(v) \cap N[v_i] \geq \lceil 19/6\rceil \geq 4$}. \\
  Therefore,
  \mbox{$N(z) \cap N(v) \cap N(v_i) \geq 3$}, which shows that $K_{3,3}$ is a
  subgraph of $G$, contradicting the assumption that $G$ is planar.
  \alexmargin{fix linebreak once all is decided.}

  If furthermore $v\not\in \hat{D}$, we can fix $w_1,\ldots, w_6$ from $D$
  that dominate $N(v)$. If $z\in V(G)\setminus \{v,w_1,\ldots, w_6\}$
  with $|N(z) \cap N(v)| \geq 19$ we can argue as above to obtain
  a contradiction.
\end{proof}

% In the light of \cref{lem:dominating-dominators}, we select all paires of nodes
% with sufficiently large intersecting neighborhood.
%
% \begin{tcolorbox}
%   For $v\in V(G)$ let $B_v\coloneqq \{z\in V(G)\setminus \{v\}:
%   |N(v)\cap N(z)|\geq 19\}$.
% \end{tcolorbox}

% \begin{corollary}\label{cor:dominating-dominators}
%   For every vertex $v$, $B_v\subseteq A_v$, in particular,
%  $|B_v|\leq 6$ and if $v\not\in \hat{D}$, then $B_v\subseteq D$.
% \end{corollary}
%
% \begin{tcolorbox}
%   We define $W$ as the set of vertices $v\in V(G)$ such
%   that $B_v\neq \emptyset$. We define \[D_2\coloneqq \bigcup_{v\in W}
%   (\{v\}\cup B_v).\]
% \end{tcolorbox}
% \vspace{0mm}

% Our algorithm now proceeds as follows. Obviously, every
% vertex $v$ can locally compute the set $B_v$. The algorithm
% adds the set $D_2$ to the dominating set, removes~$D_2$ from the graph and marks
% all vertices dominated by $D_2$ as dominated.
% \alex{here again 'remove' vertices?}

Let us now analyze the size of $D_2$. For this we refine the set $D_2$
and define
\begin{tcolorbox}
  \begin{enumerate}
    \item $D_2^1\coloneqq \bigcup_{v\in D}
    (\{v\}\cup B_v)$, \\[-3mm]
    \item $D_2^2\coloneqq \bigcup_{v\in \hat{D}\setminus D}
    (\{v\}\cup B_v)$, and \\[-3mm]
    \item $D_2^3\coloneqq \bigcup_{v\in W\setminus (D\cup \hat{D})}
    (\{v\}\cup B_v)$.
  \end{enumerate}
\end{tcolorbox}

Obviously $D_2=D_2^1\cup D_2^2\cup D_2^3$. We then bound the size of
newly defined sets.

\begin{lemma}\label{lem:size-D21}
  $|D_2^1\setminus D|\leq 6\gamma$.
\end{lemma}
\begin{proof}
  We have $|\bigcup_{v\in D}B_v|\leq \sum_{v\in D}|B_v|\leq 6\gamma$.
\end{proof}

\begin{lemma}\label{lem:size-D22}
  $|D_2^2\setminus D|< 3\gamma$.
\end{lemma}
\begin{proof}
  Let $v\in \hat{D}\setminus D$ and let $z\in B_v$. By symmetry, $v\in B_z$ and
  according to \cref{lem:dominating-dominators}, if $z\not\in \hat{D}$ then $v\in D$.
  Since this is not the case, we conclude that $z\in\hat{D}$.
  Hence $B_v\subseteq \hat{D}$ and, more generally, $D_2^2\subseteq \hat{D}$.
  Finally, according to \cref{lem:D-hat}, we have $|\hat{D}\setminus D|<3\gamma$.
\end{proof}

Finally, the set that appears largest at first glance was actually
already counted, as shown in the next lemma.
\begin{lemma}\label{lem:size-D23}
  $D_2^3\subseteq D_2^1$.
\end{lemma}
\begin{proof}
  If $v\not\in \hat{D}$, then $B_v\subseteq D$ by \cref{lem:dominating-dominators}.
  Hence $v\in B_z$ for some $z\in D$, and $v\in D_2^1$.
\end{proof}

Additionally, observe that the situation where $|D_2^1\setminus D| = 6\gamma$
is not the worst as it would imply that for each vertex $v$ of $D$, $|B_v|=6$,
implying that $D\subseteq D_2^1$, and therefore the algorithm
could stop here.
For a finer analysis, we need to define what is the amount of vertices of $D$
that have been selected in this step.

\begin{tcolorbox}
  Let $\e\in [0,1]$ be such that $|D_2^1\cap D| =\e\gamma$.
\end{tcolorbox}

\begin{lemma}\label{lem:size-D2}
  We have that $|D_2\setminus D| \le 3\gamma + 6\e\gamma$.
\end{lemma}
The proof of~\cref{lem:size-D2} is a direct consequence of
\cref{lem:size-D21,lem:size-D22,lem:size-D23} and the definition of $\e$.

%
% Several observation can be made before jumping to the third and final phase of
% the algorithm.
% First, every vertex $v$ that is not in $D_1\cup D_2$ has degree at most $114$.
% Otherwise, (as $v$ is not in $D_1$) its $115$ neighbors would be dominated by a
% set $A_v$ of at most 6 vertices. Hence there would be vertex $z$ dominating at
% least $\lceil 115/6\rceil = 20$ of them. Therefore we would have
% $|N(v)\cap N(z)|\geq 19$, contradicting that $v$ is not in $D_2$
%
% \alex{remove what's below?}
%
% \begin{tcolorbox}
%   In the following we assume that for all $v\in V(G)$ there does not
%   exist $w\in V(G)\setminus\{v\}$ with $|N(v)\cap N(z)|\geq 19$.
%   In particular, every vertex has at most $6\cdot 19=114$ non-dominated
%   neighbors and $G$ has at most $115\gamma$ non-dominated vertices.
% \end{tcolorbox}


%%%%%%%%%%%%%%%%%%%%%%%%%%%%%%%%%%%%%%%%%%%%%%%%%%%%%%%%%%%%%%%%%%%%%%%%%%%%%%%
%%%%%%%%%%%%%%%%%%%%%%%%%%%%%%%%%%%%%%%%%%%%%%%%%%%%%%%%%%%%%%%%%%%%%%%%%%%%%%%
\section{Greedy domination in planar graphs of maximum
residual degree}

\alex{All of that Section has to be rewritten with the linear programming things.}

For a set of vertices $Z$, we define the \emph{$Z$-residual degree} of a vertex
$v$, noted $\dd^Z(v)$ as the number of neighbors of~$v$ that are not in $N[Z]$.

\begin{lemma}\label{lem:res-degree}
  Every vertex of $G$ has $(D_1\cup D_2)$-residual degree at most $114$.
\end{lemma}
\begin{proof}
  First, every vertex of $D_1\cup D_2$ has residual degree $0$.
  Assume that there is a vertex $v$ of residual degree at least $115$.
  As $v$ is not in $D_1$ its $115$ neighbors are dominated by a
  set $A_v$ of at most 6 vertices. Hence there is be vertex $z$ (not in $D_1$
  nor $D_2$) dominating at least $\lceil 115/6\rceil = 20$ of them. Therefore we
  have $|N(v)\cap N(z)|\geq 19$, contradicting that $v$ is not in~$D_2$.
\end{proof}

In the light of \cref{lem:res-degree}, we could simply pick every element not in $N[D_1\cup D_2]$.
We would get a constant factor approximation,
but not a very good one. We instead choose a more subtle strategy.

We iteratively pick all elements of residual degree $i$, where $i$ starts at
$114$. Then every vertex recomputes its residual degree and $i$ is set to $113$.
We continue until $i$ reaches $10$. After picking all vertices of residual
degree $10$, we simply pick all remaining non-dominated vertices.
The choice of changing strategy when $i=9$ will be explained in the analysis
of the size of $D_3$.
More formally, we define several sets as follows.
\begin{tcolorbox}[colback=red!5!white,colframe=red!50!black]
  Let $Z_{115}\coloneqq D_1\cup D_2$. \hfill \textit{\small initial partial dominating set}\\
  For $114\geq i\geq 9$ let:
  \begin{itemize}
    \item $H_i \coloneqq G\setminus N[Z_{i+1}]$ \hfill \textit{\small non-dominated vertices}
    \item $\Delta_i \coloneqq \{v\in G ~:~ \dd^{Z_{i+1}}(v)=i\}$ \hfill \textit{\small vertices of maximal residual degree}
    \item $Z_{i} \coloneqq Z_{i+1} \cup \Delta_i$ \hfill \textit{\small update partial dominating set}
  \end{itemize}

  And finally, $D_3:= \bigcup\limits_{10\le i\le 114} \Delta_i  \cup H_{9}$
\end{tcolorbox}

As the last step consist of picking $H_9$ i.e.\ all non-dominated vertices, it
is clear that $D_1\cup D_2\cup D_3$ is a dominating set, and is computed in a
bounded number of communication rounds. We now turn to bounding the size of $D_3$.

\begin{lemma}\label{lem:h1}
  For every $9\le i\le 114$, $|H_i| \le (i+1)(1-\e)\gamma$.
\end{lemma}

\begin{lemma}\label{lem:delta}
  For every $10\le i\le 114$, $|\Delta_i| \le \frac{3|H_i|}{i-6}$.
\end{lemma}

\begin{lemma}\label{lem:h2}
  For every $9\le i\le 114$, $|H_i| \le |H_{i+1}| - \frac{(i-5)|\Delta_{i+1}|}{3}$.
\end{lemma}

\alex{expand the following argumentation / justification
  \begin{itemize}
    \item intuitively, if $\Delta_i$ is quite large, then $H_{i-1}$ is almost empty.
    \item we then maximize the size of $D_3$ within those constraints.
    \item We do not prove formal bounds
    \item we solve the problem in CPLEX
    \item the result correspond to put equality instead of inequality (to be checked).
    \item The choice of turning point at $9$ is by testing all possibility.
  \end{itemize}
}
\begin{lemma}\label{lem:size-D3}
  $|D_3|\le 20.3(1-\e)\gamma$.
\end{lemma}
\alex{to be checked}


\begin{proof}[Proof of \cref{lem:h1}]
  First note that for every $i$, $D\setminus Z_i$ is a dominating set for $H_i$.
  Then every vertex of $D\setminus Z_i$ has residual degree at most $i$.
  And finally, $D\setminus Z_i$ is a subset of $D\setminus D_2^1$. Hence by the definition of $\e$, we get that $|D\setminus Z_i|\le (1-\e)\gamma$.

  All together, we get that $|H_i|\le (i+1)(1-\e)\gamma$
\end{proof}

\begin{proof}[Proof of \cref{lem:delta}]
  Let $i$ be an integer in $[10,114]$. We bound the size of $\Delta_i$ by a
  counting argument, using that $G$ (as well as each of its subgraph) is planar,
  and can therefore not have to many edges.

  Let $J := G[\Delta_i]$, the subgraph of $G$ induced by the vertices of
  residual degree $i$. Let $K := G[\Delta_i \cup (N[\Delta_i]\cap H_i)]$,
  the subgraph of $G$ induced by the vertices of residual degree $i$, together
  with their non-dominated neighbors.

  As $J$ is planar, $E_J < 3|\Delta_i|$. As every vertex of $J$ has residual degree
  exactly $i$, we get $E_K > i\Delta_i - |E_J| \ge (i-3)\Delta_i$. We subtracted $|E_J|$ to not count twice
  the edges of $K$ that are between two vertices of $J$.
  We also have that $|K| \le |J| + |H_i|$. We finally apply Euler's formula again to $K$ and get that
  $|E_K| < 3|V_K|$ hence $(i-3)|\Delta_i| < 3|\Delta_i| + 3|H_i|$. And therefore $|\Delta_i|< \frac{3|H_i|}{i-6}$.
\end{proof}

\begin{proof}[Proof of \cref{lem:h2}]
  Similarly to the proof of \cref{lem:delta} (by replacing $i$ by $i+1$),
  we define $J := G[\Delta_{i+1}]$ and $K:= G[\Delta_{i+1} \cup (N[\Delta_{i+1}]\cap H_{i+1})]$.
  We then replace the bound $|K| \le |J| + |H_{i+1}|$ by $|K| \le |J| + |N[\Delta_{i+1}]\cap H_{i+1}|$. We then get:
  \[|E_K| < 3 |V_K| \]
  \[(i+1)|\Delta_{i+1}| - 3|\Delta_{i+1}| < 3(|\Delta_{i+1}| + |N[\Delta_{i+1}]\cap H_{i+1}|)\]
  \[ |N[\Delta_{i+1}\cap H_{i+1}]| > \frac{(i+1-6)|\Delta_{i+1}|}{3}\]

  Finally, see that $H_i = G\setminus N[Z_{i+1}] = H_{i+1} \setminus N[\Delta_{i+1}]$.

  Therefore $|H_i| \le |H_{i+1}| - |N[\Delta_{i+1}\cap H_{i+1}]|$.

  We conclude with $|H_i| < |H_{i+1}| - \frac{(i+1-6)|\Delta_{i+1}}{3}$.


\end{proof}

%
% As $\Delta_0$ contains all non-dominated vertices of $G_0$, which are
% the remaining vertices that have not been dominated, the set
% $D_3\coloneqq \bigcup_{i=0}^{114}\Delta_i$ is a dominating set of
% $G$ of order $\sum_{i=0}^{114}m_i$.
% We have a non-increasing sequence $r_{114}\geq r_{113}\geq
% \ldots\geq r_0$, where $r_{i-1}=r_i-d_i$.
%
% \begin{lemma}
%   $r_i\leq \sum_{j=0}^{i}\gamma_i\cdot i$.
% \end{lemma}
%
%
% We first observe that the algorithm performs best when it selects
% a maximum possible number of high degree vertices as fast as possible.
%
% \begin{lemma}
%   For all $115\leq i\leq 6$ if $n_i\geq i\gamma$, then
%   $\Delta_i>\frac{3}{i-3}\gamma$.
% \end{lemma}
%
% \begin{lemma}
%   For all $114\leq i\leq 6$ we have $\leq |\Delta_i| \leq \frac{3}{i-3}n_i$.
% \end{lemma}
%
% Let $X$ be the set of non-dominated vertices of $G$ and let $Y\coloneqq
% N[X]\setminus X$. For an integer $m$ let $X_{>m}$ be the set of
% vertices of $X$ that have residual degree greater than $m$ and
% similarly, let $Y_{>m}$ be the set of vertices of $Y$ that have residual degree greater than $m$. Let $G'$ be the graph induced by $X$ and $Y$.
% \textcolor{red}{Clean this}
%
% \begin{lemma}\label{lem:num-high-degree}
%   Let $G'$ be $d$-degenerate and let $m\geq 2d$. Then $|X_{>m}|< ... $
%   and $|Y_>m|< ...$ and $|X_{>m}\cup Y_{>m}|<2d/(m-d)|X|$.
% \end{lemma}
% \begin{proof}
%   The number of edges in $G[X]$ would be greater than $m|X_{>m}|-d|X_{>m}|$.
%   This implies $|X_{>m}|<2d/(m-d)|X|$. The number of edges between
%   $X$ and $Y$ would be greater than $m|Y_{>m}|$, which implies
%   $|Y_{>m}|<2d/m|X|$. The maximum is attained when all edges are hidden
%   in $X$, so the final inequality follows.
% \end{proof}
%
% \textcolor{red}{Idea:
%   We now proceed in $115$ rounds $i=115,114,\ldots, 6$. Let $G_{116}\coloneqq
%   G$ and let $G_i$ for $6\leq i\leq 115$ be the graph that is obtained
%   by handling the vertices of residual degree $i$. We get from $G_{i+1}$ to
%   $G_i$ by picking all vertices of residual degree $i$. Assume that in
%   this step we pick $k_i$ vertices of residual degree $i$. Denote by $n_i$
%   the number of vertices of $G_i$. Then $n_i\leq n_{i+1}-k_i(i-3)/6$. By the
%   lemma we have $0\leq k_i\leq 2d/(i-3)n_{i+1}$. The algorithm performs
%   best when $k_i$ for large $i$ is maximal.
%   There is a break-even point for the algorithm where picking $k_i$
%   vertices of degree $i$ has still more effect (by dominating $k_i(i-3)/6$
%   vertices) than leaving us with vertices of degree $i$ and hence
%   having a graph with at most $i\gamma$ vertices.
% }
%
%
%
% For planar graphs we want to minimize the function
% $m+6\cdot 115/(m-3)$. This minimum is attained at $m=29$.
% Let $D_3$ be the set of all vertices of residual degree at least $29$.
% Then by \cref{lem:num-high-degree} we have $|D_3|\leq 27\gamma$.






%%%%%%%%%%%%%%%%%%%%%%%%%%%%%%%%%%%%%%%%%%%%%%%%%%%%%%%%%%%%%%%%%%%%%%%%%%%%%%%
%%%%%%%%%%%%%%%%%%%%%%%%%%%%%%%%%%%%%%%%%%%%%%%%%%%%%%%%%%%%%%%%%%%%%%%%%%%%%%%
\section{Cleaning up and concluding}

$D_1\cup D_2\cup D_3$ is a dominating set of $G$.
First, by \cref{lem:lenzen}, $|D_1\setminus D|\le 3\gamma$.
Then by \cref{lem:size-D2}, we have that $|D_2\setminus D|\le 3\gamma +6\e\gamma$.
Finally, by \cref{lem:size-D3}, we have that $|D_3|\le 20.3(1-\e)\gamma$.
\alexmargin{check constant}

Therefore $|(D_1 \cup D_2 \cup D_3 ) \setminus D|
\le 3\gamma + 3\gamma +6\e\gamma + 20.3(1-\e)\gamma
\le 26.3\gamma -14.3\e\gamma$

As $\e\in[0,1]$, this value is maximized when $\e=0$. Therefore
$|(D_1 \cup D_2 \cup D_3 )|< 28 \gamma$.

\begin{theorem}
There exists a distributed LOCAL algorithm that, for every planar graph
$G$, computes in a constant number of rounds a dominating set
of size at most $28\gamma(G)$.
\end{theorem}

%%
%% Bibliography
%%

%% Please use bibtex,

\bibliography{ref}

\appendix

\end{document}
