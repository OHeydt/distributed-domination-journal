
\documentclass[a4paper,UKenglish,cleveref, autoref, thm-restate]{lipics-v2021}
%This is a template for producing LIPIcs articles.
%See lipics-v2021-authors-guidelines.pdf for further information.
%for A4 paper format use option "a4paper", for US-letter use option "letterpaper"
%for british hyphenation rules use option "UKenglish", for american hyphenation rules use option "USenglish"
%for section-numbered lemmas etc., use "numberwithinsect"
%for enabling cleveref support, use "cleveref"
%for enabling autoref support, use "autoref"
%for anonymousing the authors (e.g. for double-blind review), add "anonymous"
%for enabling thm-restate support, use "thm-restate"
%for enabling a two-column layout for the author/affilation part (only applicable for > 6 authors), use "authorcolumns"
%for producing a PDF according the PDF/A standard, add "pdfa"

%\graphicspath{{./graphics/}}%helpful if your graphic files are in another directory

% -- Added Packages --
\usepackage{mathtools}
\usepackage{tikz}
\definecolor{cadmiumgreen}{rgb}{0.0, 0.42, 0.24}
\definecolor{dark-blue}{rgb}{0.05,0.25,1}
\usepackage{tcolorbox}
\usetikzlibrary{patterns,arrows,decorations.pathreplacing}

\newcommand{\Oof}{\mathcal{O}}
\newcommand{\Cc}{\mathscr{C}}
\newcommand{\Aa}{\mathcal{A}}
\newcommand{\Bb}{\mathcal{B}}
\newcommand{\Tt}{\mathcal{T}}
\newcommand{\Nn}{\mathcal{N}}
\newcommand{\Ss}{\mathcal{S}}
\newcommand{\Pp}{\mathcal{P}}
\newcommand{\Qq}{\mathcal{Q}}
\newcommand{\Rr}{\mathcal{R}}
\newcommand{\N}{\mathbb{N}}
\newcommand{\minor}{\preceq}
\newcommand{\red}[1]{\textcolor{red}{#1}}

% -- End of added Packages --

\bibliographystyle{plainurl}% the mandatory bibstyle

\title{Brief announcement: Local domination on planar graphs revisited} %TODO Please add

\titlerunning{} %TODO optional, please use if title is longer than one line

%\author{Simeon Kublenz}{University of Bremen, Germany}{kublenz@uni-bremen.de}{}{}%TODO mandatory, please use full name; only 1 author per \author macro; first two parameters are mandatory, other parameters can be empty. Please provide at least the name of the affiliation and the country. The full address is optional

\author{Sebastian Siebertz}{University of Bremen, Germany}{siebertz@uni-bremen.de}{https://orcid.org/0000-0002-6347-1198}{}

\author{Alexandre Vigny}{University of Bremen, Germany}{vigny@uni-bremen.de}{https://orcid.org/0000-0002-4298-8876}{}

\author{Ozan Heydt}{University of Bremen, Germany}{heydt@uni-bremen.de}{}{}

\authorrunning{S. Siebertz, A. Vigny and O. Heydt} %TODO mandatory. First: Use abbreviated first/middle names. Second (only in severe cases): Use first author plus 'et al.'

\Copyright{?} %TODO mandatory, please use full first names. LIPIcs license is "CC-BY";  http://creativecommons.org/licenses/by/3.0/

\ccsdesc[500]{\textcolor{red}{Theory of computation~Self-organization}} %TODO mandatory: Please choose ACM 2012 classifications from https://dl.acm.org/ccs/ccs_flat.cfm

\keywords{Dominating set, LOCAL algorithm, planar graph} %TODO mandatory; please add comma-separated list of keywords

\category{} %optional, e.g. invited paper

\relatedversion{} %optional, e.g. full version hosted on arXiv, HAL, or other respository/website
%\relatedversiondetails[linktext={opt. text shown instead of the URL}, cite=DBLP:books/mk/GrayR93]{Classification (e.g. Full Version, Extended Version, Previous Version}{URL to related version} %linktext and cite are optional

%\supplement{}%optional, e.g. related research data, source code, ... hosted on a repository like zenodo, figshare, GitHub, ...
%\supplementdetails[linktext={opt. text shown instead of the URL}, cite=DBLP:books/mk/GrayR93, subcategory={Description, Subcategory}, swhid={Software Heritage Identifier}]{General Classification (e.g. Software, Dataset, Model, ...)}{URL to related version} %linktext, cite, and subcategory are optional

%\funding{(Optional) general funding statement \dots}%optional, to capture a funding statement, which applies to all authors. Please enter author specific funding statements as fifth argument of the \author macro.

\acknowledgements{}%optional

%\nolinenumbers %uncomment to disable line numbering

\hideLIPIcs  %uncomment to remove references to LIPIcs series (logo, DOI, ...), e.g. when preparing a pre-final version to be uploaded to arXiv or another public repository

%Editor-only macros:: begin (do not touch as author)%%%%%%%%%%%%%%%%%%%%%%%%%%%%%%%%%%
\EventEditors{John Q. Open and Joan R. Access}
\EventNoEds{2}
\EventLongTitle{42nd Conference on Very Important Topics (CVIT 2016)}
\EventShortTitle{CVIT 2016}
\EventAcronym{CVIT}
\EventYear{2016}
\EventDate{December 24--27, 2016}
\EventLocation{Little Whinging, United Kingdom}
\EventLogo{}
\SeriesVolume{42}
\ArticleNo{23}
%%%%%%%%%%%%%%%%%%%%%%%%%%%%%%%%%%%%%%%%%%%%%%%%%%%%%%%

\begin{document}

\maketitle

%TODO mandatory: add short abstract of the document
\begin{abstract}
A dominating set in a graph $G$ is a subset $D\subseteq V(G)$ of vertices
such that all vertices of $G$ are in or adjacent to a vertex of $D$. Lenzen et al.\
in 2010 presented a distributed algorithm in the LOCAL model that in a
constant number of computes
a $130$-approximation of a minimum dominating set on planar graphs.
Three years later Wawrzyniak proved that in fact the algorithm computes
a $52$-approximation. Using ideas based on a new result of Czygrinow et al.\
and Kublenz et al.\ we present a slightly modified algorithm that computes
a ... approximation on planar graphs.
\end{abstract}

\section{Introduction}

A dominating set in an undirected and simple graph $G$ is a set
$D\subseteq V(G)$ such that every vertex $v\in V(G)$ either belongs
to $D$ or has a neighbor in $D$. The dominating set problem is a
classical NP-complete problem~\cite{karp1972reducibility} with many
applications in theory and practice. In this paper we study
the distributed time complexity of finding
dominating sets in the classic LOCAL model of distributed computing.
%In this model, a distributed system is modeled by an undirected
%graph~$G$,
%in which every vertex represents a computational entity of the network and every edge represents a bidirectional communication channel. The vertices are equipped with unique identifiers.
%In a distributed algorithm, initially, the nodes have no knowledge about
%the network graph. They must then communicate and coordinate
%their actions by passing messages to one another in order to achieve
%a common goal, in our case, to compute a dominating set of the
%network graph. The complexity of a distributed algorithm in the
%LOCAL model is simply the number of communication
%rounds it needs until it returns its answer.
%
The problem of approximating
dominating sets in the LOCAL model has received considerable
attention in the literature~\cite{akhoondian2018distributed,
akhoondian2016local,
alipour2020local,
amiri2016brief,
amiri2019distributed,
barenboim2018fast,
czygrinow2008fast,
czygrinow2018distributed,
DBLP:conf/stoc/GhaffariKM17,
hilke2014brief,
kublenz2020distributed,
KuhnMW16,
lenzen2013distributed,
lenzen2008leveraging,
lenzen2010minimum,
DBLP:conf/stoc/RozhonG20,
wawrzyniak2013brief,
wawrzyniak2014strengthened}. Since in general graphs
it is not possible to compute a
constant factor approximation in a constant number of rounds~\cite{KuhnMW16},
much effort has been invested to improve the ratio between approximation
factor and number of rounds on special graph classes. In particular, a
line of structural analysis of graph properties that can lead to improved
algorithms was started by the influential paper of Lenzen et al.~\cite{lenzen2013distributed}, who in particular proved that on planar graphs
a 130-approximation
of a minimum dominating set can be computed in a constant number of
rounds. A careful analysis of Wawrzyniak~\cite{wawrzyniak2014strengthened}
later showed that the algorithm computes in fact a 52-approximation. A
result of Alipour and Jafari~\cite{alipour2020local} shows that in $C_4$-free
planar graphs one can compute even an 18-approximation. In terms of lower bounds, Hilke et al.~\cite{hilke2014brief} showed that there is no
deterministic local algorithm (constant-time distributed graph algorithm) that
finds a~$(7-\epsilon)$-approximation of a minimum dominating set on
planar graphs, for any positive constant~$\epsilon$.

In this paper we study a slight modification of the algorithm of
Lenzen et al.~\cite{lenzen2013distributed} and prove that we can
compute a ... approximation on planar graphs. For $C_4$-free planar
graphs we obtain a ... approximation. Our results are based
on an improvement of a key lemma of Lenzen et
al.~\cite{lenzen2013distributed} and ideas inspired by
recent work of Czygrinow et  al.~\cite{czygrinow2018distributed} and
Kublenz et al.~\cite{kublenz2020distributed}, and finally the
analysis of Wawrzyniak~\cite{wawrzyniak2014strengthened}.

\section{Preliminaries}

In this paper we work with the classic LOCAL model of distributed computing
restricted to planar graphs.
In this model, a distributed system is modeled by an undirected planar
graph~$G$,
in which every vertex represents a computational entity of the network and every edge represents a bidirectional communication channel. The vertices are equipped with unique identifiers.
In a distributed algorithm, initially, the nodes have no knowledge about
the network graph. They must then communicate and coordinate
their actions by passing messages to one another in order to achieve
a common goal, in our case, to compute a dominating set of the
network graph. The complexity of a LOCAL algorithm is the number of communication rounds it needs until it returns its answer.

By the famous theorem of Wagner, planar graphs can be characterized
as those graphs that exclude the complete graph $K_5$ and the
complete bipartite $K_{3,3}$ with partitions of size $3$ as a minor.
A graph~$H$ is a minor of a graph~$G$, written~$H\minor G$, if
there is a set \mbox{$\{G_v :v\in V(H)\}$} of pairwise vertex disjoint and
connected subgraphs
$G_v\subseteq G$ such that if~$\{u,v\}\in E(H)$, then there is an edge
between a vertex of~$G_u$ and a vertex of~$G_v$. We call~$V(G_v)$ the
\emph{branch set} of $v$ and say that it is
\emph{contracted} to the vertex~$v$. By Euler's formula, planar graphs
are sparse: every planar $n$-vertex graph ($n\geq 3$) has
at most $3n-6$ edges. Minors of planar graphs are again planar graphs,
so that this formula applies also to all minors of a planar graph. In
the following we assume that each component of our input graph $G$
has at least $3$ vertices. This is justified by the observation that a
minimum dominating set of a disconnected graph is a union of
minimum dominating sets for each component and that we can
compute optimal dominating sets in components of bounded size
in a constant number of rounds.

For a graph $G$ and $v\in V(G)$ we write $N(v)=\{w~:~\{v,w\}\in E(G)\}$
for the \emph{open neighborhood} of $v$ and $N[v]=N(v)\cup\{v\}$ for
the \emph{closed neighborhood} of~$v$. For a set $A\subseteq V(G)$ let
$N[A]=\bigcup_{v\in A}N[v]$.
%We write $N_r[v]$ for the set
%of vertices at distance at most $r$ from a vertex $v$.
A dominating set in a graph~$G$ is a set
$D\subseteq V(G)$ such that $N[D]=V(G)$. We write $\gamma(G)$ for
the size of a minimum dominating set of $G$. For $W\subseteq V(G)$
we say that a set $Z\subseteq V(G)$ \emph{dominates}
 $W$ if $W\subseteq N[Z]$.

 \begin{tcolorbox}
% \fbox{
% \begin{minipage}{0.9\linewidth}
In the following we fix a planar graph $G$ and a minimum dominating
set $D$ with $|D|=\gamma(G)\eqqcolon \gamma$. We assume
that every component
of $G$ has at least $3$ vertices.
\end{tcolorbox}

\section{Preprocessing}

The algorithm of Lenzen et al.~\cite{lenzen2013distributed} works in two
phases. The first phase is a preprocessing step that leaves us with
only vertices whose neighborhoods can be dominated by a few other
vertices. It is based on the following key combinatorial lemma (which
can be generalized to more general graphs). The lemma states
that there exist only few vertices whose neighborhood cannot be
dominated by at most $6$ other vertices.

\begin{lemma}[Lemma 6.3 of Lenzen et al.~\cite{lenzen2013distributed}]\label{lem:lenzen}
Let $D_1\coloneqq \{v\in V(G) : $ there does not
exist $A\subseteq V(G)\setminus \{v\}$ with $N(v)\subseteq N[A]$
and $|A|\leq 6\}$. Then $|D_1\setminus D|< 3\gamma$.
\end{lemma}

The algorithm in the first phase selects all of vertices of $D_1$ into
a partial dominating set of small size, which leaves only
vertices with the property that their neighborhood can be dominated
by at most $6$ vertices. 

%The further argumentation is crucially
%built on this fact. Our first optimization is based on the fact that
%the size of the dominating sets $A$ of the neighborhoods $N(v)$
%is much more significant for the second phase of the algorithm than
%the fact that $|D_1|<3\gamma$. By optimizing the lemma in this
%direction we can majorly improve the further argumentation.
%
%\begin{lemma}\label{lem:improved-lenzen}
%Let $D_1\coloneqq \{v\in V(G) : $ there does not
%exist $A\subseteq V(G)\setminus \{v\}$ with $N(v)\subseteq N[A]$
%and $|A|\leq 3\}$. Then $|D_1\setminus D|< 3\gamma$.
%\end{lemma}
%\begin{proof}
%Assume $D=\{d_1,\ldots, d_\gamma\}$ and assume towards a
%contradiction that there are pairwise distinct
%vertices $a_1,\ldots, a_{3\gamma}\in V(G)\setminus M$
%such that there does not exist $A\subseteq V(G)\setminus \{v\}$
%with $N(v)\subseteq N[A]$ and $|A|\leq 3$.
%
%We define the branch sets of a minor of $G$ as follows. For
%$1\leq i\leq 3\gamma$ we define a branch set $A_i\coloneqq \{a_i\}$
%and for every $1\leq i\leq \gamma$ we define a branch set
%\[D_i\coloneqq N[d_i]\setminus\big(\{a_i,...,a_{3\gamma}\} \cup \bigcup\limits_{j<i}D_j \cup \bigcup\limits_{j>i}\{d_j\}\big).\]
%By construction, the sets $D_i$ induce pairwise disjoint connected subgraphs
%of $G$, hence, by contracting them we obtain a planar graph $H$. With
%the branch sets in $G$ we associate in the obvious way the vertices
%$a_1',\ldots, a_{3\gamma}', d_1',\ldots d_\gamma'$ of $H$.
%
%Since \{$d_1,...,d_\gamma\}$ is a dominating set of $G$ and by assumption
%on $N(a_i)$, we have that in $H$, every $a'_i$ is connected to at least $4$
%of the $d'_i$.
%We therefore have that $|V_H| = 4\gamma$ and $|E(H)| \geq 4 \cdot 3\gamma$, hence $E(H)\geq 3|V(H)|>3|V(H)|-6$, contradicting Euler's
%formula.
%\end{proof}
%
%In the LOCAL model every vertex $v$ can learn the set of all vertices
%at distance $2$ of $v$ and locally determine whether $N(v)$ can
%be dominated by at most $3$ vertices, hence, compute the set
%$D$. The algorithm removes  $D$ from the graph and marks all
%its neighbors as dominated in one additional round.
%By Lemma~\ref{lem:improved-lenzen}, up to this point we have selected
%at most $3\gamma$ vertices that do not belong to $M$. We may
%hence assume that the neighborhood of every remaining vertex can
%be dominated by at most $3$ other vertices.


\begin{tcolorbox}
% \fbox{
% \begin{minipage}{0.9\linewidth}
In the following we assume that for all $v\in V(G)$ there exists
$A\subseteq V(G)\setminus \{v\}$ such that $N(v)\subseteq N[A]$
and such that $|A|\leq 6$.
% \end{minipage}
% }
\end{tcolorbox}

We can also prove by following the lines of the proof of the above
lemma that almost all vertices $v$ have even better properties,
namely that we can choose the small dominating set~$A$ for $N(v)$ as a
subset of $D$ for almost all vertices.
Note that since $D$ is unknown this set cannot be computed by our algorithm.

\begin{lemma}\label{lem:D-hat}
Let $\hat{D}\coloneqq \{v\in V(G) : $ there does not
exist $A\subseteq D\setminus \{v\}$ with $N(v)\subseteq N[A]$
and $|A|\leq 6\}$. Then $|\hat{D}\setminus D|< 3\gamma$, and
therefore $|\hat{D}|< 4\gamma$.
\end{lemma}


\section{Analyzing the local dominators}

By defining a notion of \emph{pseudo-covers}, Czygrinow et  al.~\cite{czygrinow2018distributed} provided tool to carry out a
fine grained analysis of
vertices that can potentially belong to a set $A$ used to dominate
the neighborhood $N(v)$ of a vertex $v$. This tool can in fact
be applied to much more general graphs than planar graphs, namely,
to all graphs that exclude some complete bipartite graph~$K_{t,t}$.
A refined analysis for classes of bounded expansion was provided
by Kublenz et al.~\cite{kublenz2020distributed}.
We provide an even finer analysis for planar graphs on which we
base a second phase of our distributed algorithm.

%\begin{lemma}
%Let $v\in V(G)$ and let $B_v\coloneqq \{w\in V(G)\setminus \{v\}:
%|N(v)\cap N(w)|\geq 5\}$. Then $|B_v|\leq 6$.
%\end{lemma}

%\begin{lemma}
%Let $\mathcal{P}\coloneqq
%\{(v,w): v\neq w, v,w\not\in M$ and $|N(v)\cap N(w)|\geq 5\}$.
%Then $|\mathcal{P}|\leq \gamma$.
%\end{lemma}


\begin{lemma}\label{lem:small-intersections}
Let $v\in V(G)$ and let $B_v\coloneqq \{w\in V(G)\setminus \{v\}:
|N(v)\cap N[w]|\geq 20\}$. Then $|B_v|\leq 6$ and if $v\not\in \hat{D}$,
then $B_v\subseteq D$.
\end{lemma}
\begin{proof}
By assumption there exists $A\subseteq V(G)\setminus \{v\}$ such 
that $N(v)\subseteq N[A]$ and such that $|A|\leq 6$, say 
$A=\{v_1,\ldots, v_6\}$. We show that the vertices $v_i$ are the only ones 
that can dominate at least $20$ vertices of $N(v)$.

Assume towards a contradiction that there 
exists a vertex $z\in V(G)\setminus \{v,v_1,\ldots, v_6\}$ 
with $|N[z] \cap N(v)| \geq 20$,  hence $|N(z) \cap N(v)| \geq 19$. 
As $v_1, \ldots, v_6$ dominate $N(v)$ there must be~$v_i$, $1\leq i\leq 6$, with 
\mbox{$N(z) \cap N(v) \cap N[v_i] \geq 4$}, hence, 
\mbox{$N(z) \cap N(v) \cap N(v_i) \geq 3$} resulting in a $K_{3,3}$. 
This contradicts the assumption that $G$ is planar. 

%
%
%Remember that the 10 vertices are connected to either $v_1, v_2$ or $v_3$.
%At least 4 of them are connected to one of $v_i$ with:
%$N(z) \cap N(v) \cap N[v_i] \geq 4$ or
%$N(z) \cap N(v) \cap N[v_i] \geq 3$ resulting in a $K_{3,3}$.
%Therefore $z$ has to be one of $v_i$ and $B(v) \leq 3$.

If furthermore $v\not\in \hat{D}$, then we can choose the 
the vertices $v_1,\ldots, v_6$ from $D$ and by arguing as 
above we conclude that $B_v\subseteq D$. 
\end{proof}

We define $D_2\coloneqq \bigcup_{v\in V(G)} B_v$. 

\begin{corollary}
Let $D_2\coloneqq \bigcup_{v\in V(G)} B_v$. Then $|D_2\setminus D|\leq
24\gamma$.
\end{corollary}

Our algorithm now proceeds as follows. Obviously, every
vertex $v$ can locally compute the set $B_v$. The algorithm 
adds the set $D_2=\bigcup_{v\in V(G)} B_v$ to the dominating
set, removes~$D_2$ from the graph and marks all vertices dominated
by $D_2$ as dominated.


\begin{tcolorbox}
% \fbox{
% \begin{minipage}{0.9\linewidth}
In the following we assume that for all $v\in V(G)$ there does not
exist $w\in V(G)\setminus\{v\}$ with $|N(v)\cap N[w]|\geq 20$.
In particular, every vertex has at most $6\cdot 19=114$ non-dominated 
neighbors and $G$ has at most $115\gamma$ non-dominated vertices.
% \end{minipage}
% }
\end{tcolorbox}

\section{Greedy domination in degenerate graphs of maximum 
residual degree}

We define the \emph{residual degree} of a vertex $v$ as the number 
of non-dominated vertices of $v$. 

Let $X$ be the set of non-dominated vertices of $G$ and let $Y\coloneqq
N[X]\setminus X$. For an integer $m$ let $X_{>m}$ be the set of 
vertices of $X$ that have residual degree greater than $m$ and 
similarly, let $Y_{>m}$ be the set of vertices of $Y$ that have residual degree greater than $m$. Let $G'$ be the graph induced by $X$ and $Y$. 
\textcolor{red}{Clean this}

\begin{lemma}\label{lem:num-high-degree}
Let $G'$ be $d$-degenerate and let $m\geq 2d$. Then $|X_{>m}|< ... $
and $|Y_>m|< ...$ and $|X_{>m}\cup Y_{>m}|<2d/(m-d)|X|$.
\end{lemma}
\begin{proof}
The number of edges in $G[X]$ would be greater than $m|X_{>m}|-d|X_{>m}|$. 
This implies $|X_{>m}|<2d/(m-d)|X|$. The number of edges between 
$X$ and $Y$ would be greater than $m|Y_{>m}|$, which implies 
$|Y_{>m}|<2d/m|X|$. The maximum is attained when all edges are hidden
in $X$, so the final inequality follows. 
\end{proof}

\textcolor{red}{Idea: 
We now proceed in $115$ rounds $i=115,114,\ldots, 6$. Let $G_{116}\coloneqq
G$ and let $G_i$ for $6\leq i\leq 115$ be the graph that is obtained
by handling the vertices of residual degree $i$. We get from $G_{i+1}$ to
$G_i$ by picking all vertices of residual degree $i$. Assume that in 
this step we pick $k_i$ vertices of residual degree $i$. Denote by $n_i$
the number of vertices of $G_i$. Then $n_i\leq n_{i+1}-k_i(i-3)/6$. By the 
lemma we have $0\leq k_i\leq 2d/(i-3)n_{i+1}$. The algorithm performs
best when $k_i$ for large $i$ is maximal. 
There is a break-even point for the algorithm where picking $k_i$ 
vertices of degree $i$ has still more effect (by dominating $k_i(i-3)/6$
vertices) than leaving us with vertices of degree $i$ and hence
having a graph with at most $i\gamma$ vertices. }



For planar graphs we want to minimize the function 
$m+6\cdot 115/(m-3)$. This minimum is attained at $m=29$. 
Let $D_3$ be the set of all vertices of residual degree at least $29$. 
Then by Lemma~\ref{lem:num-high-degree} we have $|D_3|\leq 27\gamma$. 


%
%\begin{lemma}
%Let $G$ be a $d$-degenerate graph and let $i$ be a positive integer. 
%Then there are at most
%$n/i$ vertices of degree at least $2di$. 
%\end{lemma}
%\begin{proof}
%Otherwise $G$ has more than $\frac{1}{2}\sum_{v\in V(G)}d(v) \geq
%\frac{1}{2}(n/i)\cdot 2di = dn$ of edges, a contradiction. 
%\end{proof}
%
%We define the auxiliary graph $G'$ from $G$. Let $X$ denote 
%the set of non-dominated vertices and let $Y\coloneqq N[X]\setminus X$. 
%For an integer $m$ let $Y_{>m}$ be the set of vertices of $Y$ that have
%residual degree greater than $m$, that is, with $|N(v)\cap X|> m$. 
%
%\begin{lemma}
%Let $G$ be a $d$-degenerate graph and let $m\geq 2d$. Then 
%$|Y_{>m}|\leq 2d|X|/m$.
%\end{lemma}
%\begin{proof}
%Otherwise we have more than $2d|X|/m\cdot m=2d|X|$ edges between 
%$X$ and $Y$ and therefore the average degree is too high. 
%\end{proof}
%
%
%\textcolor{red}{Actually we need: Let $G'$ be the graph containing the set $X$ of 
%non-dominated vertices and $Y\coloneqq N[X]\setminus X$. 
%It contains all edges between vertices of $X$, all edges between 
%vertices of $X$ and $Y$ and not between vertices of $Y$. }
%
%\begin{lemma}
%Let $G'$ be a $d$-degenerate graph and let $m\geq 4d$ be an 
%integer. Then there are at most 
%$2d|X|/m$ of degree at least $m$. 
%\end{lemma}
%\begin{proof}
%Let $X_{>m}$ be the vertices of degree greater than $m$ 
%\end{proof}
%
%\begin{corollary}
%\end{corollary}

\section{Graphs without $4$-cycles}
% !TEX root = lipics-v2021.tex

Goal : 7 factor approximation


\section{Cleaning up and concluding}

In particular, all remaining vertices of $D$ have at most $30$ non-dominated
neighbors. We conclude that the remaining graph has at most $31\gamma$
vertices. We define the set $D_3$ as the set of all remaining non-dominated
vertices. Then $|D_3\setminus D|\leq 30\gamma$ and $D_1\cup D_2\cup D_3$
is a dominating set of $G$.

\begin{theorem}
There exists a distributed LOCAL algorithm that for every planar graph
$G$ computes in a constant number of rounds a dominating set
of size at most $46\gamma(G)$.
\end{theorem}

%%
%% Bibliography
%%

%% Please use bibtex,

\bibliography{ref}

\appendix

\end{document}
